% language=us

\usemodule[present-boring,abbreviations-logos]

\startdocument
  [title={METAFUN},
   banner={simple fonts},
   location={context\enspace {\bf 2020}\enspace meeting}]

\starttitle[title=Metafonts]

\startitemize

\startitem
    Because \METAPOST\ is based on \METAFONT\ it make sense to use of for making fonts.
\stopitem

\startitem
    Making a font is an art in itself, something that is actually proven by many bad
    looking fonts, but we have plenty of choice nowadays.
\stopitem

\startitem
    We tend to use free fonts and often being made by volunteers we can hardly have any
    demands.
\stopitem

\startitem
    So, instead of complaining (which is not nice anyway) we can try to (at least temporary)
    come up with a solution ourselves.
\stopitem

\startitem
    We're actually talking about missing glyphs here and \METAPOST\ can be of help.
\stopitem

\startitem
    Also keep in mind that we always had this option or variants of it in \CONTEXT, it's
    just that we can make nicer interfaces now.
\stopitem

\startitem
    So, don't expect something spectacular.
\stopitem

\stopitemize

\stoptitle

\starttitle[title=What is is not]

Years ago mechanisms were added to \MKIV\ to come up with more fancy shapes in
for instance math. Actually Alan needed it and I wanted a root symbol to look
like school times.

\startbuffer
\useMPlibrary[mat]

\setupmathradical[color=darkgray,alternative=mp]

% \definemathradical [sqrt] [mp=minifun::math:radical:default]
\stopbuffer

\typebuffer \getbuffer

So:

\startbuffer
\scale[height=2cm]{$ \sqrt {a+b+c+d} $}
\stopbuffer

\typebuffer

Gives:

\startlinecorrection
\getbuffer
\stoplinecorrection

\page

And with:

\startbuffer
\startuniqueMPgraphic{minifun::math:radical:default}
draw
    math_radical_simple(OverlayWidth,OverlayHeight,OverlayDepth,OverlayOffset)
    withpen pencircle
        xscaled (2OverlayLineWidth)
        yscaled (1OverlayLineWidth/4)
        rotated 30
    dashed evenly
    withcolor OverlayLineColor ;
\stopuniqueMPgraphic
\stopbuffer

\typebuffer

We get

\startlinecorrection
\getbuffer
\scale[height=2cm]{$ \sqrt {a+b+c+d} $}
\stoplinecorrection

\page

Also think of stackers:

\startbuffer
\setupmathstackers [both]   [color=darkgray,alternative=mp]
\setupmathstackers [top]    [color=darkgray,alternative=mp]
\setupmathstackers [bottom] [color=darkgray,alternative=mp]
\stopbuffer

\typebuffer \getbuffer

\blank[2*line]

\startbuffer
$
    \overbracket   {a+b+c+d} \quad
    \underbracket  {a+b+c+d} \quad
    \doublebracket {a+b+c+d}
$
\blank
$
    \overparent   {a+b+c+d} \quad
    \underparent  {a+b+c+d} \quad
    \doubleparent {a+b+c+d}
$
\blank
$
    \overbrace   {a+b+c+d} \quad
    \underbrace  {a+b+c+d} \quad
    \doublebrace {a+b+c+d}
$
\blank
$
    \overbar   {a+b+c+d} \quad
    \underbar  {a+b+c+d} \quad
    \doublebar {a+b+c+d}
$
\blank
$
    \overleftarrow  {a+b+c+d} \quad
    \overrightarrow {a+b+c+d}
$
\blank
$
    \underleftarrow  {a+b+c+d} \quad
    \underrightarrow {a+b+c+d}
$
\stopbuffer

\getbuffer

\blank[2*line]

But, these are just overlays and nothing special: we simply don't use the normal
font route not fancy \LUA\ tricks either (in principle \MKII\ could do this). I
might upgrade it some day (no real demand so far, just fun stuff).

\stoptitle

\starttitle[title=Real fonts]

\startitemize

\startitem For text we need an efficient way to define extra shapes. \stopitem
\startitem We don't really want inline graphics every time we use a glyph. \stopitem
\startitem We also want to cut and paste properly. \stopitem
\startitem Basically the fact that we drop in shapes should be hidden. \stopitem

\blank[2*line]

\startitem
    We use the same (generic) subsystem that is also used for color fonts, bitmap
    emoji, \SVG\ fonts, etc.
\stopitem
\startitem
    Shapes end up as \TYPETHREE\ fonts. These have some specific properties and
    limitations, but we can actually make \UNICODE\ fonts.
\stopitem
\startitem
    The system is not burdened by much overhead and most happens at embedding time.
\stopitem

\stopitemize

\page

\startbuffer[font]
\definefont[DemoFontA][Serif*default @ 10pt]
\definefont[DemoFontB][Serif*default @ 12pt]
\definefont[DemoFontC][Serif*default @ 14pt]
\definefont[DemoFontD][SerifBold*default @ 14pt]
\stopbuffer

\typebuffer[font] \getbuffer[font]

\startbuffer[demo]
\startlines
\DemoFontA first\endash second\emdash third\char"2015\relax fourth
\DemoFontB first\endash second\emdash third\char"2015\relax fourth
\DemoFontC first\endash second\emdash third\char"2015\relax fourth
\DemoFontD first\endash second\emdash third\char"2015\relax fourth
\stoplines
\stopbuffer

\typebuffer[demo] \getbuffer[demo]

\page

\startbuffer[mpone]
\startMPcalculation{simplefun}

    vardef QuotationDashOne =
        draw image (
            interim linecap := squared ;
            save l ; l := 0.2 ;
            draw (l/2,3) -- (10-l/2,3) withpen pencircle scaled l ;
        )
    enddef ;

    lmt_registerglyphs [
        name     = "symbolsone",
        units    = 10,
        usecolor = true,
        width    = 10,
        height   = 3.1,
        depth    = 0,
    ] ;

    lmt_registerglyph [
        category = "symbolsone",
        unicode  = "0x2015",
        code     = "QuotationDashOne ;"
    ] ;

\stopMPcalculation
\stopbuffer

\getbuffer[mpone]

\startbuffer[font]
\definefontfeature[exampleone][metapost=symbolsone]

\definefont[DemoFontA][Serif*default,exampleone @ 10pt]
\definefont[DemoFontB][Serif*default,exampleone @ 12pt]
\definefont[DemoFontC][Serif*default,exampleone @ 14pt]
\definefont[DemoFontD][SerifBold*default,exampleone @ 14pt]
\stopbuffer

\typebuffer[font] \getbuffer[font]

\getbuffer[demo]

\page

\typebuffer[mpone][style=\tt\small\small]

\page

\startbuffer[mptwo]
\startMPcalculation{simplefun}

    vardef QuotationDashTwo =
        draw image (
            interim linecap := squared ;
            save l ; l := 0.4 ;
            string weight ; weight := getparameter "mpsfont" "parentdata" "shared" "rawdata" "metadata" "weight" ;
            if     weight = "semibold" : l := l * 2;
            elseif weight = "bold"     : l := l * 3; fi
            draw (l/2,3) -- (10-l/2,3) withpen pencircle scaled l
            withcolor yellow ;
        )
    enddef ;

    lmt_registerglyphs [
        name     = "symbolstwo",
        units    = 10,
        usecolor = false,
        width    = 10,
        height   = 3.1,
        depth    = 0,
    ] ;

    lmt_registerglyph [
        category = "symbolstwo",
        unicode  = "0x2015",
        code     = "QuotationDashTwo ;"
    ] ;

\stopMPcalculation
\stopbuffer

\getbuffer[mptwo]

\startbuffer[font]
\definefontfeature[exampletwo][metapost=symbolstwo]

\definefont[DemoFontA][Serif*default,exampletwo @ 10pt]
\definefont[DemoFontB][Serif*default,exampletwo @ 12pt]
\definefont[DemoFontC][Serif*default,exampletwo @ 14pt]
\definefont[DemoFontD][SerifBold*default,exampletwo @ 14pt]
\stopbuffer

\typebuffer[font] \getbuffer[font]

\getbuffer[demo]

\page

\typebuffer[mptwo][style=\tt\small\small]

\stoptitle

\starttitle[title=More examples]

We give some examples (these are also in the modules). Overloading math
symbols:

\starttyping
meta-imp-kindergarten.mkxl
\stoptyping

Extending fonts with Don Knuths dices and tiles (symbols, ligatures, proper
\UNICODE):

\starttyping
meta-imp-gamesymbols.mkxl
\stoptyping

An implementation of Don Knuths ThirtySix font in various variants (color,
random, shapes):

\starttyping
meta-imp-threesix.mkxl
\stoptyping

\stopdocument
