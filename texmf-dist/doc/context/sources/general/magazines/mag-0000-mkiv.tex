% language=us

% author    : Hans Hagen
% copyright : ConTeXt Development Team
% license   : Creative Commons Attribution ShareAlike 4.0 International
% reference : pragma-ade.nl | contextgarden.net | texlive (related) distributions
% origin    : the ConTeXt distribution
%
% comment   : Because this manual is distributed with TeX distributions it comes with a rather
%             liberal license. We try to adapt these documents to upgrades in the (sub)systems
%             that they describe. Using parts of the content otherwise can therefore conflict
%             with existing functionality and we cannot be held responsible for that. Many of
%             the manuals contain characteristic graphics and personal notes or examples that
%             make no sense when used out-of-context.
%
% comment   : Some chapters might have been published in TugBoat, the NTG Maps, the ConTeXt
%             Group journal or otherwise. Thanks to the editors for corrections. Also thanks
%             to users for testing, feedback and corrections.

\usemodule[mag-01]

\startbuffer[abstract]
    This is the zero issue of a semi periodical. The associated style can be used
    by \CONTEXT\ users to typeset and publish their own issues.
\stopbuffer

\startdocument
  [title={Introduction},
   subtitle={Welcome},
   author={Hans Hagen},
   affiliation=PRAGMA ADE,
   date=Januari 2003,
   number=0 \MKIV]

This is the zero issue of a range of \CONTEXT\ related publications, in most
cases short introductions to new functionality. The style may be used by users
for providing similar documents, but preferably not for other purposes, since it
may confuse readers in their expectations.

We've chosen a layout which is more functional than beautiful. This layout
provides several text areas: headers and footers, margins and edges as well as a
main text area. The surrounding (gray or color) makes the main page (which is
slightly smaller than A4) stand out and is suitable for viewing in spread mode.

The documents produced at \PRAGMA\ are called {\bf This Way}, user documents gets
the title {\bf My Way}. The \PRAGMA\ issues are numbered. We strongly advise you
not to use the \type {mag-} prefix for your issues, since this may lead to
clashes with files distributed by \PRAGMA.

\stopdocument
