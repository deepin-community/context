%D Please don't abuse this style for your own purpose as the look and feel
%D is reserved for our own purpose. Be creative instead.

\setupbodyfont
  [plex,7pt]

\setuppapersize
  [A3,landscape]

\setuplayout
  [width=middle,
   height=middle,
   header=0pt,
   footer=0pt,
   backspace=5mm,
   topspace=5mm]

\definelayer
  [page]
  [width=\textwidth,
   height=\textheight]

\setupbackgrounds
  [page]
  [background=color,
   backgroundcolor=middlegray]

\setuptyping
  [before=,after=]

\usemodule
  [abbreviations-logos]

\startbuffer[style]
  \setupbodyfont
    [plex]
  \setuplayout
    [width=middle,
     height=middle,
     margin=1cm]
  \setupcolors
    [textcolor=white]
  \usemodule
    [abbreviations-logos]
  \setupwhitespace
    [big]
  \setuphead
    [chapter]
    [style=\bfc]
\stopbuffer

\startbuffer[intro]

    \starttitle[title={Using \CONTEXT}]

    The \CONTEXT\ macro package is more than just a \TEX\ processor,
    various input is possible, some we show here. An example of a
    method not shown here is fetching data from a database. The
    various input methods can be combined, so depending on what you
    need you can mix \TEX\ (for typesetting text), \METAPOST\ (for
    producing graphics) or \LUA\ (as language for manipulating
    data.

    All these methods are quite efficient and always have access to
    the full power of the typesetting engine.

    When you use \CONTEXT\ with \LUAMETATEX, you get a reasonable
    small self contained component that can be used in workflows
    that need quality rendering. The ecosystem is rather future
    proof too.

    The \CONTEXT\ macro package has been around for decades and
    evolved from \MKII, to \MKIV\ and now \LMTX. The development
    team has always been involved in the development of engines
    like \PDFTEX, \LUATEX\ and \LUAMETATEX. There is an active
    mailing list and there are \CONTEXT\ meetings.

    \stoptitle

\stopbuffer

\startbuffer[tex]
\starttext

    \starttitle[title={Some \TEX}]

        Just an example.

        \starttabulate[|c|c|]
            \NC first 1 \NC last 1 \NC \NR
            \NC first 2 \NC last 2 \NC \NR
        \stoptabulate

    \stoptitle

\stoptext
\stopbuffer

\startbuffer[xml]
\startbuffer[demo]
<?xml version="1.0"?>
<document>
    <title>Some XML</title>
    <p>Just an example.</p>
    <table>
        <r> <c>first 1</c> <c>last 1</c> </r>
        <r> <c>first 2</c> <c>last 2</c> </r>
    </table>
</document>
\stopbuffer

\startxmlsetups xml:basics
    \xmlsetsetup{#1}{title|p|table}{xml:*}
\stopxmlsetups
\startxmlsetups xml:title
    \title{\xmltext{#1}{.}}
\stopxmlsetups
\startxmlsetups xml:p
    \xmlflush{#1}\par
\stopxmlsetups
\startxmlsetups xml:table
    \starttabulate[|c|c|]
        \xmlfilter{#1}{/r/command(xml:r)}
    \stoptabulate
\stopxmlsetups
\startxmlsetups xml:r
    \NC \xmlfilter{#1}{/c/command(xml:c)} \NR
\stopxmlsetups
\startxmlsetups xml:c
    \xmlflush{#1} \NC
\stopxmlsetups

\xmlregistersetup{xml:basics}

\starttext
    \xmlprocessbuffer{demo}{demo}{}
\stoptext
\stopbuffer

\startbuffer[lua]
\startluacode
    local tmp = {
        { a = "first 1", b = "last 1" },
        { b = "last 2", a = "first 2" },
    }

    -- local tmp = table.load("somefile.lua")

    context.starttext()

        context.starttitle { title = "Some Lua" }

            context("Just an example.") context.par()

            context.starttabulate { "|c|c|" }
                for i=1,#tmp do
                    local t = tmp[i]
                    context.NC()
                        context(t.a) context.NC()
                        context(t.b) context.NC()
                    context.NR()
                end
            context.stoptabulate()

        context.stoptitle()

    context.stoptext()
\stopluacode
\stopbuffer

\startbuffer[mp]
\startMPpage
    draw textext("\bf Some \MetaPost")
        xsized 4cm
        rotated(45)
        withcolor "white" ;

    draw textext("\bs\strut in \ConTeXt")
        xsized 5cm
        shifted (0,-40mm)
        withcolor "white" ;

    draw fullcircle
        scaled 6cm
        dashed evenly
        withcolor "gray" ;
\stopMPpage
\stopbuffer

\startbuffer[csv]
\startluacode
    local tmp = [[
        first,second
        first 1,last 1
        first 2,last 2
    ]]

    -- local tmp = io.loaddata("somefile.csv")

    local mycsvsplitter = utilities.parsers.rfc4180splitter()
    local list, names = mycsvsplitter(tmp,true)

    context.starttext()

        context.starttitle { title = "Some CSV" }

            context("Just an example.") context.par()

            context.starttabulate { "|c|c|" }
                for i=1,#list do
                    local l = list[i]
                    context.NC()
                        context(l[1]) context.NC()
                        context(l[2]) context.NC()
                    context.NR()
                end
            context.stoptabulate()

        context.stoptitle()

    context.stoptext()
\stopluacode
\stopbuffer

\startbuffer[json]
\startluacode
    require("util-jsn")

    -- local str = io.loaddata("somefile.json")

    local str = [[ {
        "title": "Some JSON",
        "text" : "Just an example.",
        "data" : [
            { "a" : "first 1", "b" : "last 1" },
            { "b" : "last 2", "a" : "first 2" }
        ]
    } ]]

    local tmp = utilities.json.tolua(str)

    context.starttext()

        context.starttitle { title = tmp.title }

            context(tmp.text) context.par()

            context.starttabulate { "|c|c|" }
                for i=1,#tmp.data do
                    local d = tmp.data[i]
                    context.NC()
                        context(d.a) context.NC()
                        context(d.b) context.NC()
                    context.NR()
                end
            context.stoptabulate()

        context.stoptitle()

    context.stoptext()
\stopluacode
\stopbuffer

\startbuffer[mkxi]
% normally there is already a file:

\startbuffer[demo]
\starttext
  \starttitle[title={Some template}]

  Just an example. \blank

  \startlinecorrection
    \bTABLE
      <?lua for i=1,20 do ?>
        \bTR
          <?lua for j=1,5 do ?>
            \bTD
              cell (<?lua inject(i) ?>,<?lua inject(j) ?>)
              is <?lua inject(variables.text or "unset") ?>
            \eTD
          <?lua end ?>
        \eTR
      <?lua end ?>
    \eTABLE
  \stoplinecorrection

  \stoptitle
\stoptext

\stopbuffer

\savebuffer[file=demo.mkxi,prefix=no,list=demo]

% the action:

\startluacode
    document.variables.text = "set"
\stopluacode

\input{demo.mkxi}
\stopbuffer

\definemeasure[blobwidth] [\textwidth/4-3mm]
\definemeasure[blobscale] [\textwidth/4-3mm-4mm]
\definemeasure[blobheight][\textheight/2-2mm]

\startbuffer[everything]

\setlayerframed
  [page]
  [preset=lefttop]
  [align=normal,
   offset=2mm,
   strut=no,
   frame=off,
   height=\measure{blobheight},
   width=\measure{blobwidth},
   background=color,
   backgroundcolor=darkgray,
   foregroundcolor=white]
  {\doifelsemode{verbose}{\typebuffer[intro]}{\typesetbuffer[style,intro][frame=on,width=\measure{blobscale}]}}

\setlayerframed
  [page]
  [preset=lefttop,
   hoffset=4mm,
   x=\measure{blobwidth}]
  [align=normal,
   offset=2mm,
   strut=no,
   frame=off,
   height=\measure{blobheight},
   width=\measure{blobwidth},
   background=color,
   backgroundcolor=darkred,
   foregroundcolor=white]
  {\doifelsemode{verbose}{\typebuffer[tex]}{\typesetbuffer[style,tex][frame=on,width=\measure{blobscale}]}}

\setlayerframed
  [page]
  [preset=righttop,
   hoffset=4mm,
   x=\measure{blobwidth}]
  [align=normal,
   offset=2mm,
   strut=no,
   frame=off,
   height=\measure{blobheight},
   width=\measure{blobwidth},
   background=color,
   backgroundcolor=darkblue,
   foregroundcolor=white]
  {\doifelsemode{verbose}{\typebuffer[xml]}{\typesetbuffer[style,xml][frame=on,width=\measure{blobscale}]}}

\setlayerframed
  [page]
  [preset=righttop]
  [align=normal,
   offset=2mm,
   strut=no,
   frame=off,
   height=\measure{blobheight},
   width=\measure{blobwidth},
   background=color,
   backgroundcolor=darkgreen,
   foregroundcolor=white]
  {\doifelsemode{verbose}{\typebuffer[lua]}{\typesetbuffer[style,lua][frame=on,width=\measure{blobscale}]}}

\setlayerframed
  [page]
  [preset=lefttop,
   voffset=4mm,
   y=\measure{blobheight}]
  [align=normal,
   offset=2mm,
   strut=no,
   frame=off,
   height=\measure{blobheight},
   width=\measure{blobwidth},
   background=color,
   backgroundcolor=darkcyan,
   foregroundcolor=white]
  {\doifelsemode{verbose}{\typebuffer[mp]}{\typesetbuffer[style,mp][frame=on,width=\measure{blobscale}]}}

\setlayerframed
  [page]
  [preset=lefttop,
   hoffset=4mm,
   x=\measure{blobwidth},
   voffset=4mm,
   y=\measure{blobheight}]
  [align=normal,
   offset=2mm,
   strut=no,
   frame=off,
   height=\measure{blobheight},
   width=\measure{blobwidth},
   background=color,
   backgroundcolor=darkmagenta,
   foregroundcolor=white]
  {\doifelsemode{verbose}{\typebuffer[csv]}{\typesetbuffer[style,csv][frame=on,width=\measure{blobscale}]}}

\setlayerframed
  [page]
  [preset=righttop,
   hoffset=4mm,
   x=\measure{blobwidth},
   voffset=4mm,
   y=\measure{blobheight}]
  [align=normal,
   offset=2mm,
   strut=no,
   frame=off,
   height=\measure{blobheight},
   width=\measure{blobwidth},
   background=color,
   backgroundcolor=darkyellow,
   foregroundcolor=white]
  {\doifelsemode{verbose}{\typebuffer[json]}{\typesetbuffer[style,json][frame=on,width=\measure{blobscale}]}}

\setlayerframed
  [page]
  [preset=righttop,
   voffset=4mm,
   y=\measure{blobheight}]
  [align=normal,
   offset=2mm,
   strut=no,
   frame=off,
   height=\measure{blobheight},
   width=\measure{blobwidth},
   background=color,
   backgroundcolor=darkorange,
   foregroundcolor=white]
  {\doifelsemode{verbose}{\typebuffer[mkxi]}{\typesetbuffer[style,mkxi][frame=on,width=\measure{blobscale}]}}

\startstandardmakeup
    \tightlayer[page]
\stopstandardmakeup

\stopbuffer

\starttext

{\enablemode[verbose] \getbuffer[everything]}
                      \getbuffer[everything]

\stoptext

