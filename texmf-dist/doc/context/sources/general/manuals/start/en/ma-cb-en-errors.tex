\startcomponent ma-cb-en-errors

\enablemode[**en-us]

\project ma-cb

\startchapter[reference=trouble,title=Problems during processing]

\index{errors}
\index{problems}

If processing is not succesful ---for example because you typed \type{\stptext}
instead of \type{\stoptext}--- \CONTEXT\ produces a \type{ ? } on your screen
and tells you it has just processed an error. It will give you some basic
information on the type of error and the line number where the error becomes
effective.

At the instant of \type{ ? } you can type:

\starttabulate[|||]
\NC \type{H} \NC for help information on your error \NC\NR
\NC \type{I} \NC for inserting the correct \CONTEXT\ command \NC\NR
\NC \type{Q} \NC for quiting and entering batch mode \NC\NR
\NC \type{X} \NC for exiting the running mode \NC\NR
\NC \Enter   \NC for ignoring the error \NC\NR
\stoptabulate

Most of the time you will type \Enter\ and processing will continue. Then you can
edit the input file and fix the error.

Some errors will produce a~\type{ * } on your screen and processing will stop.
This error is due to a fatal error in your input file. You can't ignore this
error and the only option you have is to type \type{\stop} or {\sc Ctrl}~Z. The
program will be halted and you can fix the error in your text editor.

\startframedtext[width=\hsize]
A well known error is:

\starttyping
! I can't write on file 'myfile.pdf'.
Please type another filename for output:
\stoptyping

This error is due to the fact that the file \type{myfile.pdf} is stil open
in \READER.

\blank

The best way to proceed is:

\startitemize[packed]
\item close the file in \READER
\item type \Enter\ at the console
\stopitemize
\stopframedtext

Sometimes the error messages are very obscure. Finding the location of the error
in an extensive document can then be a tedious job. You could try to isolate
the error:

\startitemize[packed]
\item open the file in your text editor
\item save a copy of your file (to be on the safe side)
\item isolate the error
      \startitemize[n,packed]
      \item place a \type{\stoptext} command higher up in your text
      \item process the file
      \item repeat step 1 and 2 until the file processes correctly
      \stopitemize
\item study the content that produces the error
\item fix the error
\item place the \type{\stoptext} command after the corrected error
\item process your file
\item etc.
\stopitemize

\stopchapter

\stopcomponent
