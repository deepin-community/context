% language=us

\startcomponent hybrid-parbuilder

\startbuffer[MyAbstract]
\StartAbstract
    In this article I will summarize some experiences with converting the \TEX\
    par builder to \LUA. In due time there will be a plugin mechanism in
    \CONTEXT, and this is a prelude to that.
\StopAbstract
\stopbuffer

\doifmodeelse {tugboat} {
    \usemodule[tug-01,abr-01]
    \setvariables
      [tugboat]
      [columns=yes]
    \setvariables
      [tugboat]
      [year=2010,
       volume=99,
       number=9,
       page=99]
    \setvariables
      [tugboat]
      [title=Building paragraphs,
       subtitle=,
       keywords=,
       author=Hans Hagen,
       address=PRAGMA ADE\\Ridderstraat 27\\8061GH Hasselt NL,
       email=pragma@xs4all.nl]
    %
    % we use a buffer as abstract themselves are buffers and
    % inside macros we loose line endings and such
    \getbuffer[MyAbstract]
    %
    \StartArticle
} {
    \environment hybrid-environment
    \startchapter[title={Building paragraphs}]
}

\startsection [title={Introduction}]

You enter the den of the Lion when you start messing around with the parbuilder.
Actually, as \TEX\ does a pretty good job on breaking paragraphs into lines I
never really looked into the code that does it all. However, the Oriental \TEX\
project kind of forced it upon me. In the chapter about font goodies an optimizer
is described that works per line. This method is somewhat similar to expansion
level~one support (hz) in the sense that it acts independent of the par builder:
the split off (best) lines are postprocessed. Where expansion involves horizontal
scaling, the goodies approach does with (Arabic) words what the original HZ
approach does with glyphs.

It would be quite some challenge (at least for me) to come up with solutions that
look at the whole paragraph and as the per-line approach works quite well, there
is no real need for an alternative. However, in September 2008, when we were
exploring solutions for Arabic par building, Taco converted the parbuilder into
\LUA\ code and stripped away all code related to hyphenation, protrusion,
expansion, last line fitting, and some more. As we had enough on our plate at
that time, we never came to really testing it. There was even less reason to
explore this route because in the Oriental \TEX\ project we decided to follow the
\quotation {use advanced \OPENTYPE\ features} route which in turn lead to the
\quote {replace words in lines by narrower of wider variants} approach.

However, as the code was laying around and as we want to explore further I
decided to pick up the parbuilder thread. In this chapter some experiences will
be discussed. The following story is as much Taco's as mine.

\stopsection

\startsection [title={Cleaning up}]

In retrospect, we should not have been too surprised that the first approximation
was broken in many places, and for good reason. The first version of the code was
a conversion of the \CCODE\ code that in turn was a conversion from the original
interwoven \PASCAL\ code. That first conversion still looked quite \CCODE||ish
and carried interesting bit and pieces of \CCODE||macros, \CCODE||like pointer
tests, interesting magic constants and more.

When I took the code and \LUA-fied it nearly every line was changed and it took
Taco and me a bit of reverse engineering to sort out all problems (thank you
Skype). Why was it not an easy task? There are good reasons for this.

\startitemize

\startitem The parbuilder (and related hpacking) code is derived from traditional
\TEX\ and has bits of \PDFTEX, \ALEPH\ (\OMEGA), and of course \LUATEX. \stopitem

\startitem The advocated approach to extending \TEX\ has been to use change files
which means that a coder does not see the whole picture. \stopitem

\startitem Originally the code is programmed in the literate way which means that
the resulting functions are build stepwise. However, the final functions can (and
have) become quite large. Because \LUATEX\ uses the woven (merged) code indeed we
have large functions. Of course this relates to the fact that succesive \TEX\
engines have added functionality. Eventually the source will be webbed again, but
in a more sequential way. \stopitem

\startitem This is normally no big deal, but the \ALEPH\ (\OMEGA) code has added
a level of complexity due to directional processing and additional begin and end
related boxes. \stopitem

\startitem Also the \ETEX\ extension that deals with last line fitting is
interwoven and uses goto's for the control flow. Fortunately the extensions are
driven by parameters which make the related code sections easy to recognize.
\stopitem

\startitem The \PDFTEX\ protrusion extension adds code to glyph handling and
discretionary handling. The expansion feature does that too and in addition also
messes around with kerns. Extra parameters are introduced (and adapted) that
influence the decisions for breaking lines. There is also code originating in
\PDFTEX\ which deals with poor mans grid snapping although that is quite isolated
and not interwoven. \stopitem

\startitem Because it uses a slightly different way to deal with hyphenation,
\LUATEX\ itself also adds some code. \stopitem

\startitem Tracing is sort of interwoven in the code. As it uses goto's to share
code instead of functions, one needs to keep a good eye on what gets skipped or
not. \stopitem

\stopitemize

I'm pretty sure that the code that we started with looks quite different from the
original \TEX\ code if it had been translated into \CCODE. Actually in modern
\TEX\ compiling involves a translation into \CCODE\ first but the intermediate
form is not meant for human eyes. As the \LUATEX\ project started from that
merged code, Taco and Hartmut already spent quite some time on making it more
readable. Of course the original comments are still there.

Cleaning up such code takes a while. Because both languages are similar but also
quite different it took some time to get compatible output. Because the \CCODE\
code uses macros, careful checking was needed. Of course \LUA's table model and
local variables brought some work as well. And still the code looks a bit
\CCODE||ish. We could not divert too much from the original model simply because
it's well documented.

When moving around code redundant tests and orphan code has been removed. Future
versions (or variants) might as well look much different as I want more hooks,
clearly split stages, and convert some linked list based mechanism to \LUA\
tables. On the other hand, as already much code has been written for \CONTEXT\
\MKIV, making it all reasonable fast was no big deal.

\stopsection

\startsection [title={Expansion}]

The original \CCODE||code related to protrusion and expansion is not that
efficient as many (redundant) function calls take place in the linebreaker and
packer. As most work related to fonts is done in the backend, we can simply stick
to width calculations here. Also, it is no problem at all that we use floating
point calculations (as \LUA\ has only floats). The final result will look okay as
the original hpack routine will nicely compensate for rounding errors as it will
normally distribute the content well enough. We are currently compatible with the
regular par builder and protrusion code, but expansion gives different results
(actually not worse).

The \LUA\ hpacker follows a different approach. And let's admit it: most \TEX ies
won't see the difference anyway. As long as we're cross platform compatible it's
fine.

It is a well known fact that character expansion slows down the parbuilder. There
are good reasons for this in the \PDFTEX\ approach. Each glyph and intercharacter
kern is checked a few times for stretch or shrink using a function call. Also
each font reference is checked. This is a side effect of the way \PDFTEX\ backend
works as there each variant has its own font. However, in \LUATEX, we scale
inline and therefore don't really need the fonts. Even better, we can get rid of
all that testing and only need to pass the eventual \type {expansion_ratio} so
that the backend can do the right scaling. We will prototype this in the \LUA\
version \footnote {For this Hartmuts has adapted the backend code has to honour
this field in the glyph and kern nodes.} and we feel confident about this
approach it will be backported into the \CCODE\ code base. So eventually the
\CCODE\ might become a bit more readable and efficient.

Intercharacter kerning is dealt with in a somewhat strange way. If a kern of
subtype zero is seen, and if it's neighbours are glyphs from the same font, the
kern gets replaced by a scaled one looked up in the font's kerning table. In the
parbuilder no real replacement takes place but as each line ends up in the hpack
routine (where all work is simply duplicated and done again) it really gets
replaced there. When discussing the current aproach we decided, that manipulating
intercharacter kerns while leaving regular spacing untouched, is not really a
good idea so there will be an extra level of configuration added to \LUATEX:
\footnote {As I more and more run into books typeset (not by \TEX) with a
combination of character expansion and additional intercharacter kerning I've
been seriously thinking of removing support for expansion from \CONTEXT\ \MKIV.
Not all is progress especially if it can be abused.}

\starttabulate
\NC 0 \NC no character and kern expansion \NC \NR
\NC 1 \NC character and kern expansion applied to complete lines \NC \NR
\NC 2 \NC character and kern expansion as part of the par builder \NC \NR
\NC 3 \NC only character expansion as part of the par builder (new) \NC \NR
\stoptabulate

You might wonder what happens when you unbox such a list: the original font
references have been replaced as were the kerns. However, when repackaged again,
the kerns are replaced again. In traditional \TEX, indeed rekerning might happen
when a paragraph is repackaged (as different hyphenation points might be chosen
and ligature rebuilding etc.\ has taken place) but in \LUATEX\ we have clearly
separated stages. An interesting side effect of the conversion is that we really
have to wonder what certain code does and if it's still needed.

\stopsection

\startsection [title={Performance}]

% timeit context ...

We had already noticed that the \LUA\ variant was not that slow. So after the
first cleanup it was time to do some tests. We used our regular \type {tufte.tex}
test file. This happens to be a worst case example because each broken line ends
with a comma or hyphen and these will hang into the margin when protruding is
enabled. So the solution space is rather large (an example will be shown later).

Here are some timings of the March 26, 2010 version. The test is typeset in a box
so no shipout takes place. We're talking of 1000 typeset paragraphs. The times
are in seconds an between parentheses the speed relative to the regular
parbuilder is mentioned.

\startmode[mkiv]

\startluacode
    local times = {
        { 1.6,  8.4,  9.8 }, --  6.7 reported in statistics
        { 1.7, 14.2, 15.6 }, -- 13.4
        { 2.3, 11.4, 13.3 }, --  9.5
        { 2.9, 19.1, 21.5 }, -- 18.2
    }

    local NC, NR, b, format = context.NC, context.NR, context.bold, string.format

    local function v(i,j)
        if times[i][j]<10 then   -- This is a hack. The font that we use has no table
            context.dummydigit() -- digits (tnum) so we need this hack. Not nice anyway.
        end
        context.equaldigits(format("%0.01f",times[i][j]))
        if j > 1 then
            context.enspace()
            context.equaldigits(format("(%0.01f)",times[i][j]/times[i][1]))
        end
    end

    context.starttabulate { "|l|c|c|c|" }
        NC()                 NC() b("native") NC() b("lua") NC() b("lua + hpack") NC() NR()
        NC() b("normal")     NC() v(1,1)      NC() v(1,2)   NC() v(1,3)           NC() NR()
        NC() b("protruding") NC() v(2,1)      NC() v(2,2)   NC() v(2,3)           NC() NR()
        NC() b("expansion")  NC() v(3,1)      NC() v(3,2)   NC() v(3,3)           NC() NR()
        NC() b("both")       NC() v(4,1)      NC() v(4,2)   NC() v(4,3)           NC() NR()
    context.stoptabulate()
\stopluacode

\stopmode

\startnotmode[mkiv]

% for the tugboat article

\starttabulate[|l|c|c|c|]
\NC                \NC \bf native \NC \bf lua    \NC \bf lua + hpack \NC \NR
\NC \bf normal     \NC 1.6        \NC  8.4 (5.3) \NC  9.8 (6.1)      \NC \NR
\NC \bf protruding \NC 1.7        \NC 14.2 (8.4) \NC 15.6 (9.2)      \NC \NR
\NC \bf expansion  \NC 2.3        \NC 11.4 (5.0) \NC 13.3 (5.8)      \NC \NR
\NC \bf both       \NC 2.9        \NC 19.1 (6.6) \NC 21.5 (7.4)      \NC \NR
\stoptabulate

\stopnotmode

For a regular paragraph the \LUA\ variant (currently) is 5~times slower and about
6~times when we use the \LUA\ hpacker, which is not that bad given that it's
interpreted code and that each access to a field in a node involves a function
call. Actually, we can make a dedicated hpacker as some code can be omitted, The
reason why the protruding is relatively slow is, that we have quite some
protruding characters in the test text (many commas and potential hyphens) and
therefore we have quite some lookups and calculations. In the \CCODE\ variant
much of that is inlined by macros.

Will things get faster? I'm sure that I can boost the protrusion code and
probably the rest as well but it will always be slower than the built in
function. This is no problem as we will only use the \LUA\ variant for
experiments and special purposes. For that reason more \MKIV\ like tracing will
be added (some is already present) and more hooks will be provided once the
builder is more compartimized. Also, future versions of \LUATEX\ will pass around
paragrapgh related parameters differently so that will have impact on the code as
well.

\stopsection

\startsection[title=Usage]

The basic parbuilder is enabled and disabled as follows:\footnote {I'm not
sure yet if the parbuilder has to do automatic grouping.}

\startbuffer[example]
\definefontfeature[example][default][protrusion=pure]
\definedfont[Serif*example]
\setupalign[hanging]

\startparbuilder[basic]
    \startcolor[blue]
        \input tufte
    \stopcolor
\stopparbuilder

\stopbuffer

\typebuffer[example]

\startmode[mkiv]
    This results in: \par \getbuffer[example]
\stopmode

There are a few tracing options in the \type {parbuilders} namespace but these
are not stable yet.

\stopsection

\startsection[title=Conclusion]

The module started working quite well around the time that Peter Gabriels
\quotation {Scratch My Back} ended up in my Squeezecenter: modern classical
interpretations of some of his favourite songs. I must admit that I scratched the
back of my head a couple of times when looking at the code below. It made me
realize that a new implementation of a known problem indeed can come out quite
different but at the same time has much in common. As with music it's a matter of
taste which variant a user likes most.

At the time of this writing there is still work to be done. For instance the
large functions need to be broken into smaller steps. And of course more testing
is needed.

\stopsection

\doifmodeelse {tugboat} {
    \StopArticle
} {
    \stopchapter
}

\stopcomponent
