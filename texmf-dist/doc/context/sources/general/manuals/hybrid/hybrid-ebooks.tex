% language=us

\startcomponent hybrid-ebooks

\environment hybrid-environment

\startchapter[title={E-books: Old wine in new bottles}]

\startsection [title={Introduction}]

When Dave Walden asked me if \TEX\ (or \CONTEXT) can generate ebooks we exchanged
a bit of mail on the topic. Although I had promised myself never to fall into the
trap of making examples for the sake of proving something I decided to pick up an
experiment that I had been doing with a manual in progress and look into the
\HTML\ side of that story. After all, occasionally on the \CONTEXT\ list similar
questions are asked, like \quotation {Can \CONTEXT\ produce \HTML ?}. \footnote
{This text appeared in the \EUROTEX\ 2011 proceedings and TUGBoat 101. Thanks to
Karl Berry for correcting it.}

\stopsection

\startsection [title={Nothing new}]

When you look at what nowadays is presented as an ebook document, there is not
much new going on. Of course there are very advanced and interactive documents,
using techniques only possible with recent hardware and programs, but the average
ebook is pretty basic. This is no surprise. When you take a novel, apart from
maybe a cover or an occasional special formatting of section titles, the
typesetting of the content is pretty straightforward. In fact, given that
formatters like \TEX\ have been around that can do such jobs without much
intervention, it takes quite some effort to get that job done badly. It was a bit
shocking to notice that on one of the first e-ink devices that became available
the viewing was quite good, but the help document was just some word processor
output turned into bad|-|looking \PDF. The availability of proper hardware does
not automatically trigger proper usage.

I can come up with several reasons why a novel published as an ebook does not
look much more interesting and in many cases looks worse. First of all it has to
be produced cheaply, because there is also a printed version and because the
vendor of some devices also want to make money on it (or even lock you into their
technology or shop). Then, it has to be rendered on various devices so the least
sophisticated one sets the standard. As soon as it gets rendered, the resolution
is much worse than what can be achieved in print, although nowadays I've seen
publishers go for quick and dirty printing, especially for reprints.

Over a decade ago, we did some experiments with touch screen computers. They had
a miserable battery life, a slow processor and not much memory, but the
resolution was the same as on the now fashionable devices. They were quite
suitable for reading but even in environments where that made sense (for instance
to replace carrying around huge manuals), such devices never took off. Nowadays
we have wireless access and \USB\ sticks and memory cards to move files around,
which helps a lot. And getting a quality comparable to what can be done today was
no big deal, at least from the formatting point of view.

In the \CONTEXT\ distribution you will find several presentation styles that can
serve as bases for an ebook style. Also some of the \CONTEXT\ manuals come with
two versions: one for printing and one for viewing on the screen. A nice example
is the \METAFUN\ manual (see \in {figure} [fig:metafun]) where each page has a
different look.

\placefigure
  [here]
  [fig:metafun]
  {A page from the \METAFUN\ manual.}
  {\externalfigure[ebook-metafun-2.png][width=\textwidth]}

It must be said that the (currently only black and white) devices that use
electronic ink have a perceived resolution that is higher than their
specifications, due to the semi|-|analog way the \quote {ink} behaves. In a
similar fashion clever anti|-|aliasing can do wonders on \LCD\ screens. On the
other hand they are somewhat slow and a display refresh is not that convenient.
Their liquid crystal counterparts are much faster but they can be tiresome to
look at for a long time and reading a book on it sitting in the sun is a no|-|go.
Eventually we will get there and I'm really looking forward to seeing the first
device that will use a high resolution electrowetting \CMYK\ display. \footnote
{http://www.liquavista.com/files/LQV0905291LL5-15.pdf} But no matter what device
is used, formatting something for it is not the most complex task at hand.

\stopsection

\startsection[title={Impact}]

Just as with phones and portable audio devices, the market for tablets and
ebook|-|only devices is evolving rapidly. While writing this, at work I have one
ebook device and one tablet. The ebook device is sort of obsolete because the
e-ink screen has deteriorated even without using it and it's just too slow to be
used for reference manuals. The tablet is nice, but not that suitable for all
circumstances: in the sun it is unreadable and at night the backlight is rather
harsh. But, as I mentioned in the previous section, I expect this to change.

If we look at the investment, one needs good arguments to buy hardware that is
seldom used and after a few years is obsolete. Imagine that a family of four has
to buy an ebook device for each member. Add to that the cost of the books and you
quickly can end up with a larger budget than for books. Now, imagine that you
want to share a book with a friend: will you give him or her the device? It might
be that you need a few more devices then. Of course there is also some data
management needed: how many copies of a file are allowed to be made and do we
need special programs for that? And if no copy can be made, do we end up swapping
devices? It is hard to predict how the situation will be in a few years from now,
but I'm sure that not everyone can afford this rapid upgrading and redundant
device approach.

A friend of mine bought ebook devices for his children but they are back to paper
books now because the devices were not kid|-|proof enough: you can sit on a book
but not on an ebook reader.

The more general devices (pads) have similar problems. I was surprised to see
that an iPad is a single user device. One can hide some options behind passwords
but I'm not sure if parents want children to read their mail, change preferences,
install any application they like, etc. This makes pads not that family friendly
and suggests that such a personal device has to be bought for each member. In
which case it suddenly becomes a real expensive adventure. So, unless the prices
drop drastically, pads are not a valid large scale alternative for books yet.

It might sound like I'm not that willing to progress, but that's not true. For
instance, I'm already an enthusiastic user of a media player infrastructure.
\footnote {The software and hardware was developed by SlimDevices and currently
is available as Logitech Squeezeserver. Incidentally I can use the iPad as an
advanced remote control.} The software is public, pretty usable, and has no
vendor lock|-|in. Now, it would make sense to get rid of traditional audio media
then, but this is not true. I still buy \CD{}s if only because I then can rip
them to a proper lossless audio format (\FLAC). The few \FLAC s that I bought via
the Internet were from self|-|publishing performers. After the download I still
got the \CD{}s which was nice because the booklets are among the nicest that I've
ever seen.

Of course it makes no sense to scan books for ebook devices so for that we depend
on a publishing network. I expect that at some point there will be proper tools
for managing your own electronic books and in most cases a simple file server
will do. And the most open device with a proper screen will become my favourite.
Also, I would not be surprised if ten years from now, many authors will publish
themselves in open formats and hopefully users will be honest enough to pay for
it. I'm not too optimistic about the latter, if only because I observe that
younger family members fetch everything possible from the Internet and don't
bother about rights, so we definitely need to educate them. To some extent
publishers of content deserve this behaviour because more than I like I find
myself in situations where I've paid some 20 euro for a \CD\ only to see that
half a year later you can get it for half the price (sometimes it also happens
with books).

Given that eventually the abovementioned problems and disadvantages will be dealt
with, we can assume that ebooks are here and will stay forever. So let's move on
to the next section and discuss their look and feel.

\stopsection

\startsection[title={Interactivity}]

The nice thing about a paper book is that it is content and interface at the same
time. It is clear where it starts and ends and going from one page to another is
well standardized. Putting a bookmark in it is easy as you can fall back on any
scrap of paper lying around. While reading you know how far you came and how much
there is to come. Just as a desktop on a desktop computer does not resemble the
average desktop, an ebook is not a book. It is a device that can render content
in either a given or more free|-|form way.

However, an electronic book needs an interface and this is also where at the
moment it gets less interesting. Of course the Internet is a great place to
wander around and a natural place to look for electronic content. But there are
some arguments for buying them at a bookshop, one being that you see a lot of
(potentially) new books, often organized in topics in one glance. It's a
different way of selecting. I'm not arguing that the Internet is a worse place,
but there is definitely a difference: more aggressive advertisements, unwanted
profiling that can narrow what is presented to a few choices.

Would I enter a bookshop if on the display tables there were stacks of (current)
ebook devices showing the latest greatest books? I can imagine that at some point
we will have ebook devices that have screens that run from edge to edge and then
we get back some of the appeal of book designs. It is that kind of future devices
that we need to keep in mind when we design electronic documents, especially when
after some decades we want them to be as interesting as old books can be. Of
course this is only true for documents that carry the look and feel of a certain
time and place and many documents are thrown away. Most books have a short
lifespan due to the quality of the paper and binding so we should not become too
sentimental about the transition to another medium.

Once you're in the process of reading a book not much interfacing is needed.
Simple gestures or touching indicated areas on the page are best. For more
complex documents the navigation could be part of the design and no screen real
estate has to be wasted by the device itself. Recently I visited a
school|-|related exhibition and I was puzzled by the fact that on an electronic
schoolboard so much space was wasted on colorful nonsense. Taking some 20\% off
each side of such a device brings down the effective resolution to 600 pixels so
we end up with 10 pixels or less per character (shown at about 1 cm width). At
the same exhibition there were a lot of compensation programs for dyslexia
advertised, and there might be a relationship.

\stopsection

\startsection[title={Formatting}]

So how important is the formatting? Do we prefer reflow on demand or is a more
frozen design that suits the content and expresses the wish of the author more
appropriate? In the first case \HTML\ is a logical choice, and in the second one
\PDF\ makes sense. You design a nice \HTML\ document but at some point the reflow
gets in the way. And yes, you can reflow a \PDF\ file but it's mostly a joke.
Alternatively one can provide both which is rather trivial when the source code
is encoded in a systematic way so that multiple output is a valid option. Again,
this is not new and mostly a matter of a publisher's policy. It won't cost more
to store in neutral formats and it has already been done cheaply for a long time.

Somewhat interfering in this matter is digital rights management. While it is
rather customary to buy a book and let friends or family read the same book, it
can get complicated when content is bound to one (or a few) devices. Not much
sharing there, and in the worst case, no way to move your books to a better
device. Each year in the Netherlands we have a book fair and bookshops give away
a book specially written for the occasion. This year the book was also available
as an ebook, but only via a special code that came with the book. I decided to
give it a try and ended up installing a broken application, i.e.\ I could not get
it to load the book from the Internet, and believe me, I have a decent machine
and the professional \PDF\ viewer software that was a prerequisite.

\stopsection

\startsection[title={Using \TEX}]

So, back to Dave's question: if \CONTEXT\ can generate ebooks in the \EPUB\
format. Equally interesting is the question if \TEX\ can format an \EPUB\ file
into a (say) \PDF\ file. As with much office software, an \EPUB\ file is nothing
more than a zip file with a special suffix in which several resources are
combined. The layout of the archive is prescribed. However, by demanding that the
content itself is in \HTML\ and by providing a stylesheet to control the
renderer, we don't automatically get properly tagged and organized content. When
I first looked into \EPUB, I naively assumed that there was some well|-|defined
structure in the content; turns out this is not the case.

Let's start by answering the second question. Yes, \CONTEXT\ can be used to
convert an \EPUB\ file into a \PDF\ file. The natural followup question is if it
can be done automatically, and then some more nuance is needed: it depends. If
you download the \EPUB\ for \quotation {A tale of two cities} from Charles
Dickens from the Gutenberg Project website and look into a chapter you will see
this:

\starttyping
<h1 id="pgepubid00000">A TALE OF TWO CITIES</h1>
<h2 id="pgepubid00001">A STORY OF THE FRENCH REVOLUTION</h2>
<p><br/></p>
<h2>By Charles Dickens</h2>
<p><br/>
<br/></p>
<hr/>
<p><br/>
<br/></p>
<h2 id="pgepubid00002">Contents</h2>
\stoptyping

What follows is a table of contents formatted using \HTML\ tables
and after that

\starttyping
<h2 id="pgepubid00004">I. The Period</h2>
\stoptyping

So, a level two header is used for the subtitle of the book as well as a regular
chapter. I must admit that I had to go on the Internet to find this snippet as I
wanted to check its location. On my disk I had a similar file from a year ago
when I first looked into \EPUB. There I have:

\starttyping
<html xmlns="http://www.w3.org/1999/xhtml" xml:lang="en">
  <head>
    <title>I | A Tale of Two Cities</title>
    ....
  </head>
  <body>
    <div class="body">
      <div class="chapter">
        <h3 class="chapter-title">I</h3>
        <h4 class="chapter-subtitle">The Period</h4>
\stoptyping

I also wanted to make sure if the interesting combination of third and fourth
level head usage was still there but it seems that there are several variants
available. It is not my intention to criticize the coding, after all it is valid
\HTML\ and can be rendered as intended. Nevertheless, the first snippet
definitely looks worse, as it uses breaks instead of \CSS\ spacing directives and
the second wins on obscurity due to the abuse of the head element.

These examples answer the question about formatting an arbitrary \EPUB\ file:
\quotation {no}. We can of course map the tagging to \CONTEXT\ and get pretty
good results but we do need to look at the coding.

As such books are rather predictable it makes sense to code them in a more
generic way. That way generic stylesheets can be used to render the book directly
in a viewer and generic \CONTEXT\ styles can be used to format it differently,
e.g.\ as \PDF.

Of course, if I were asked to set up a workflow for formatting ebooks, that would
be relatively easy. For instance the Gutenberg books are available as raw text
and that can be parsed to some intermediate format or (with \MKIV) interpreted
directly.

Making a style for a specific instance, like the Dickens book, is not that
complex either. After all, the amount of encoding is rather minimal and special
bits and pieces like a title page need special design anyway. The zipped file can
be processed directly by \CONTEXT, but this is mostly just a convenience.

As \EPUB\ is just a wrapper, the next question is if \CONTEXT\ can produce some
kind of \HTML\ and the answer to that question is positive. Of course this only
makes sense when the input is a \TEX\ source, and we have argued before that when
multiple output is needed the user might consider a different starting point.
After all, \CONTEXT\ can deal with \XML\ directly.

The main advantage of coding in \TEX\ is that the source remains readable and for
some documents it's certainly more convenient, like manuals about \TEX. In the
reference manual \quote {\CONTEXT\ \LUA\ Documents} (\CLD) there are the
following commands:

\starttyping
\setupbackend
  [export=yes]

\setupinteraction
  [title=Context Lua Documents,
   subtitle=preliminary version,
   author=Hans Hagen]
\stoptyping

At the cost of at most 10\% extra runtime an \XML\ export is generated in
addition to the regular \PDF\ file. Given that you have a structured \TEX\ source
the exported file will have a decent structure as well and you can therefore
transform the file into something else, for instance \HTML. But, as we already
have a good|-|looking \PDF\ file, the only reason to have \HTML\ as well is for
reflowing. Of course wrapping up the \HTML\ into an \EPUB\ structure is not that
hard. We can probably even get away from wrapping because we have a single
self|-|contained file.

\placefigure
  {A page from the \CLD\ manual in \PDF.}
  {\externalfigure[ebook-pdf.png][width=\textwidth]}

The \type {\setupbackend} command used in the \CLD\ manual has a few
more options:

\starttyping
\setupbackend
  [export=cld-mkiv-export.xml,
   xhtml=cld-mkiv-export.xhtml,
   css={cld-mkiv-export.css,mathml.css}]
\stoptyping

We explicitly name the export file and in addition specify a stylesheet and an
alternative \XHTML\ file. If you can live without hyperlinks the \XML\ file
combined with the cascading style sheet will do a decent job of controlling the
formatting.

In the \CLD\ manual chapters are coded like this:

\starttyping
\startchapter[title=A bit of Lua]

\startsection[title=The language]
\stoptyping

The \XML\ output of this

\starttyping
<division detail='bodypart'>
  <section detail='chapter' location='aut:3'>
    <sectionnumber>1</sectionnumber>
    <sectiontitle>A bit of Lua</sectiontitle>
    <sectioncontent>
      <section detail='section'>
        <sectionnumber>1.1</sectionnumber>
        <sectiontitle>The language</sectiontitle>
        <sectioncontent>
\stoptyping

The \HTML\ version has some extra elements:

\starttyping
<xhtml:a name="aut_3">
  <section location="aut:3" detail="chapter">
\stoptyping

The table of contents and cross references have \type {xhtml:a} elements too but
with the \type {href} attribute. It's interesting to search the web for ways to
avoid this, but so far no standardized solution for mapping \XML\ elements onto
hyperlinks has been agreed upon. In fact, getting the \CSS\ mapping done was not
that much work but arriving at the conclusion that (in 2011) these links could
only be done in a robust way using \HTML\ tags took more time. \footnote {In this
example we see the reference \type {aut:3} turned into \type {aut_1}. This is
done because some browsers like to interpret this colon as a url.} Apart from
this the \CSS\ has enough on board to map the export onto something presentable.
For instance:

\starttyping
sectioncontent {
    display: block ;
    margin-top: 1em ;
    margin-bottom: 1em ;
}

section[detail=chapter], section[detail=title] {
    margin-top: 3em ;
    margin-bottom: 2em ;
}

section[detail=chapter]>sectionnumber {
    display: inline-block ;
    margin-right: 1em ;
    font-size: 3em ;
    font-weight: bold ;
}
\stoptyping

As always, dealing with verbatim is somewhat special. The following code does the
trick:

\starttyping
verbatimblock {
    background-color: #9999FF ;
    display: block ;
    padding: 1em ;
    margin-bottom: 1em ;
    margin-top: 1em ;
    font-family: "Lucida Console", "DejaVu Sans Mono", monospace ;
}

verbatimline {
    display: block ;
    white-space: pre-wrap ;
}

verbatim {
    white-space: pre-wrap ;
    color: #666600 ;
    font-family: "Lucida Console", "DejaVu Sans Mono", monospace ;
}
\stoptyping

The spacing before the first and after the last one differs from the spacing
between lines, so we need some extra directives:

\starttyping
verbatimlines+verbatimlines {
    display: block ;
    margin-top: 1em ;
}
\stoptyping

This will format code like the following with a bluish background and inline
verbatim with its complement:

\starttyping
<verbatimblock detail='typing'>
  <verbatimlines>
    <verbatimline>function sum(a,b)</verbatimline>
    <verbatimline>  print(a, b, a + b)</verbatimline>
    <verbatimline>end</verbatimline>
  </verbatimlines>
</verbatimblock>
\stoptyping

The hyperlinks need some attention. We need to make sure that only the links and
not the anchors get special formatting. After some experimenting I arrived at
this:

\starttyping
a[href] {
    text-decoration: none ;
    color: inherit ;
}

a[href]:hover {
    color: #770000 ;
    text-decoration: underline ;
}
\stoptyping

Tables are relatively easy to control. We have tabulate (nicer for text) and
natural tables (similar to the \HTML\ model). Both get mapped into \HTML\ tables
with \CSS\ directives. There is some detail available so we see things like this:

\starttyping
tablecell[align=flushleft] {
    display: table-cell ;
    text-align: left ;
    padding: .1em ;
}
\stoptyping

It is not hard to support more variants or detail in the export but that will
probably only happen when I find a good reason (a project), have some personal
need, or when a user asks for it. For instance images will need some special
attention (conversion, etc.). Also, because we use \METAPOST\ all over the place
that needs special care as well, but a regular (novel|-|like) ebook will not have
such resources.

\placefigure
  {A page from \CLD\ manual in \HTML.}
  {\externalfigure[ebook-xhtml.png][width=\textwidth]}

As an extra, a template file is generated that mentions all
elements used, like this:

\starttyping
section[detail=summary] {
    display: block ;
}
\stoptyping

with the inline and display properties already filled in. That way I could see
that I still had to add a couple of directives to the final \CSS\ file. It also
became clear that in the \CLD\ manual some math is used that gets tagged as
\MATHML, so that needs to be covered as well. \footnote {Some more advanced
\MATHML\ output will be available when the matrix|-|related core commands have
been upgraded to \MKIV\ and extended to suit today's needs.} Here we need to make
some decisions as we export \UNICODE\ and need to consider support for less
sophisticated fonts. On the other hand, the \WTHREEC\ consortium has published
\CSS\ for this purpose so we can use these as a starting point. It might be that
eventually more tuning will be delegated to the \XHTML\ variant. This is not much
extra work as we have the (then intermediate) \XML\ tree available. Thinking of
it, we could eventually end up with some kind of \CSS\ support in \CONTEXT\
itself.

It will take some experimenting and feedback from users to get the export right,
especially to provide a convenient way to produce so|-|called \EPUB\ files
directly. There is already some support for this container format. If you have
enabled \XHTML\ export, you can produce an \EPUB\ archive afterwards with:

\starttyping
mtxrun --script epub yourfile
\stoptyping

For testing the results, open source programs like \type {calibre} are quite
useful. It will probably take a while to figure out to what extent we need to
support formats like \EPUB, if only because such formats are adapted on a regular
basis.

\stopsection

\startsection[title=The future]

It is hard to predict the future. I can imagine that given the user interface
that has evolved over ages paper books will not disappear soon. Probably there
will be a distinction between read|-|once and throw|-|away books and those that
you carry with you your whole life as visible proof of that life. I can also
imagine that (if only for environmental reasons) ebooks (hopefully with stable
devices) will dominate. In that case traditional bookshops will disappear and
with them the need for publishers that supply them. Self|-|publishing will then
be most interesting for authors and maybe some of them (or their helpful friends)
will be charmed by \TEX\ and tinkering with the layout using the macro language.
I can also imagine that at some point new media (and I don't consider an ebook a
new medium) will dominate. And how about science fiction becoming true:
downloading stories and information directly into our brains.

It reminds me of something I need to do some day soon: get rid of old journals
that I planned to read but never will. I would love to keep them electronically
but it is quite unlikely that they are available and if so, it's unlikely that I
want to pay for them again. This is typically an area where I'd consider using an
ebook device, even if it's suboptimal. On the other hand, I don't consider
dropping my newspaper subscription yet as I don't see a replacement for the
regular coffeestop at the table where it sits and where we discuss the latest
news.

The nice thing about an analogue camera is that the image carrier has been
standardized and you can buy batteries everywhere. Compare this with their
digital cousins: all have different batteries, there are all kinds of memory
cards, and only recently has some standardization in lenses shown up. There is a
wide range of resolutions and aspect ratios. Other examples of standardization
are nuts and bolts used in cars, although it took some time for the metric system
to catch on. Books have different dimensions but it's not hard to deal with that
property. Where desktop hardware is rather uniform everything portable is
different. For some brands you need a special toolkit with every new device.
Batteries cannot be exchanged and there are quite some data carriers. On the
other hand, we're dealing with software and if we want we can support data
formats forever. The \MICROSOFT\ operating systems have demonstrated that
programs written years ago can still run on updates. In addition \LINUX\
demonstrates that users can take and keep control and create an independence from
vendors. So, given that we can still read document sources and given that they
are well structured, we can create history|-|proof solutions. I don't expect that
the traditional publishers will play a big role in this if only because of their
short term agendas and because changing ownerships works against long term views.
And programs like \TEX\ have already demonstrated having a long life span,
although it must be said that in today's rapid upgrade cycles it takes some
courage to stay with it and its descendants. But downward compatibility is high
on the agenda of its users and user groups which is good in the perspective of
discussing stable ebooks.

Let's finish with an observation. Books often make a nice (birthday) present and
finding one that suits is part of the gift. Currently a visible book has some
advantages: when unwrapped it can be looked at and passed around. It also can be
a topic of discussion and it has a visible personal touch. I'm not so sure if
vouchers for an ebook have the same properties. It probably feels a bit like
giving synthetic flowers. I don't know what percentage of books is given as
presents but this aspect cannot be neglected. Anyway, I wonder when I will buy my
first ebook and for who. Before that happens I'll probably have generated lots of
them.

\stopsection

\stopcomponent
