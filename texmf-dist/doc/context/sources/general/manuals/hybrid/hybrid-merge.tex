% language=us

\startcomponent hybrid-merge

\environment hybrid-environment

\startchapter[title={Including pages}]

\startsection [title={Introduction}]

It is tempting to add more and more features to the backend code
of the engine but it is not really needed. Of course there are
features that can best be supported natively, like including
images. In order to include \PDF\ images in \LUATEX\ the backend
uses a library (xpdf or poppler) that can load an page from a file
and embed that page into the final \PDF, including all relevant
(indirect) objects needed for rendering. In \LUATEX\ an
experimental interface to this library is included, tagged as
\type {epdf}. In this chapter I will spend a few words on my first
attempt to use this new library.

\stopsection

\startsection [title={The library}]

The interface is rather low level. I got the following example
from Hartmut (who is responsible for the \LUATEX\ backend code and
this library).

\starttyping
local doc = epdf.open("luatexref-t.pdf")
local cat = doc:getCatalog()
local pag = cat:getPage(3)
local box = pag:getMediaBox()

local w = pag:getMediaWidth()
local h = pag:getMediaHeight()
local n = cat:getNumPages()
local m = cat:readMetadata()

print("nofpages: ", n)
print("metadata: ", m)
print("pagesize: ", w .. " * " .. h)
print("mediabox: ", box.x1, box.x2, box.y1, box.y2)
\stoptyping

As you see, there are accessors for each interesting property
of the file. Of course such an interface needs to be extended
when the \PDF\ standard evolves. However, once we have access to
the so called catalog, we can use regular accessors to the
dictionaries, arrays and other data structures. So, in fact we
don't need a full interface and can draw the line somewhere.

There are a couple of things that you normally do not want to
deal with. A \PDF\ file is in fact just a collection of objects
that form a tree and each object can be reached by an index using
a table that links the index to a position in the file. You don't
want to be bothered with that kind of housekeeping indeed. Some data
in the file, like page objects and annotations are organized in a
tree form that one does not want to access in that form, so again
we have something that benefits from an interface. But the
majority of the objects are simple dictionaries and arrays.
Streams (these hold the document content, image data, etc.) are
normally not of much interest, but the library provides an
interface as you can bet on needing it someday. The library also
provides ways to extend the loaded \PDF\ file. I will not discuss
that here.

Because in \CONTEXT\ we already have the \type {lpdf} library for
creating \PDF\ structures, it makes sense to define a similar
interface for accessing \PDF. For that I wrote a wrapper that will
be extended in due time (read: depending on needs). The previous
code now looks as follows:

\starttyping
local doc = epdf.open("luatexref-t.pdf")
local cat = doc.Catalog
local pag = cat.Pages[3]
local box = pag.MediaBox

local llx, lly, urx, ury = box[1], box[2] box[3], box[4]

local w = urx - llx -- or: box.width
local h = ury - lly -- or: box.height
local n = cat.Pages.size
local m = cat.Metadata.stream

print("nofpages: ", n)
print("metadata: ", m)
print("pagesize: ", w .. " * " .. h)
print("mediabox: ", llx, lly, urx, ury)
\stoptyping

If we write code this way we are less dependent on the exact \API,
especially because the \type {epdf} library uses methods to access
the data and we cannot easily overload method names in there. When
you look at the \type {box}, you will see that the natural way to
access entries is using a number. As a bonus we also provide the
\type {width} and \type {height} entries.

\stopsection

\startsection [title={Merging links}]

It has always been on my agenda to add the possibility to carry
the (link) annotations with an included page from a document. This
is not that much needed in a regular document, but it can be handy
when you use \CONTEXT\ to assemble documents. In any case, such a
merge has to happen in such a way that it does not interfere with
other links in the parent document. Supporting this in the engine
is no option as each macro package follows its own approach to
referencing and interactivity. Also, demands might differ and one
would end up with a lot of (error prone) configurability. Of course
we want scaled pages to behave well too.

Implementing the merge took about a day and most of that time was
spent on experimenting with the \type {epdf} library and making
the first version of the wrapper. I definitely had expected to
waste more time on it. So, this is yet another example of
extensions that are quite doable in the \LUA|-|\TEX\ mix. Of
course it helps that the \CONTEXT\ graphic inclusion code provides
enough information to integrate such a feature. The merge is
controlled by the interaction key, as shown here:

\starttyping
\externalfigure[somefile.pdf][page=1,scale=700,interaction=yes]
\externalfigure[somefile.pdf][page=2,scale=600,interaction=yes]
\externalfigure[somefile.pdf][page=3,scale=500,interaction=yes]
\stoptyping

You can finetune the merge by providing a list of options to the
interaction key but that's still somewhat experimental. As a start
the following links are supported.

\startitemize[packed]
\startitem internal references by name (often structure related) \stopitem
\startitem internal references by page (e.g.\ table of contents) \stopitem
\startitem external references by file (optionally by name and page) \stopitem
\startitem references to uri's (normally used for webpages) \stopitem
\stopitemize

When users like this functionality (or when I really need it
myself) more types of annotations can be added although support
for \JAVASCRIPT\ and widgets doesn't make much sense. On the other
hand, support for destinations is currently somewhat simplified
but at some point we will support the relevant zoom options.

The implementation is not that complex:

\startitemize[packed]
\startitem check if the included page has annotations \stopitem
\startitem loop over the list of annotations and determine if
           an annotation is supported (currently links) \stopitem
\startitem analyze the annotation and overlay a button using the
           destination that belongs to the annotation \stopitem
\stopitemize

Now, the reason why we can keep the implementation so simple is that
we just map onto existing \CONTEXT\ functionality. And, as we have
a rather integrated support for interactive actions, only a few
basic commands are involved. Although we could do that all in
\LUA, we delegate this to \TEX. We create a layer which we put on top
of the image. Links are put onto this layer using the equivalent of:

\starttyping
\setlayer
  [epdflinks]
  [x=...,y=...,preset=leftbottom]
  {\button
     [width=...,height=...,offset=overlay,frame=off]
     {}% no content
     [...]}}
\stoptyping

The \type {\button} command is one of those interaction related
commands that accepts any action related directive. In this first
implementation we see the following destinations show up:

\starttyping
somelocation
url(http://www.pragma-ade.com)
file(somefile)
somefile::somelocation
somefile::page(10)
\stoptyping

References to pages become named destinations and are later
resolved to page destinations again, depending on the
configuration of the main document. The links within an included
file get their own namespace so (hopefully) they will not clash
with other links.

We could use lower level code which is faster but we're not
talking of time critical code here. At some point I might optimize
the code a bit but for the moment this variant gives us some
tracing options for free. Now, the nice thing about using this
approach is that the already existing cross referencing mechanisms
deal with the details. Each included page gets a unique reference
so references to not included pages are ignored simply because
they cannot be resolved. We can even consider overloading certain
types of links or ignoring named destinations that match a
specific pattern. Nothing is hard coded in the engine so we have
complete freedom of doing that.

\stopsection

\startsection [title={Merging layers}]

When including graphics from other applications it might be that
they have their content organized in layers (that then can be
turned on or off). So it will be no surprise that on the agenda is
merging layer information: first a straightforward inclusion of
optional content dictionaries, but it might make sense to parse
the content stream and replace references to layers by those that
are relevant in the main document. Especially when graphics come
from different sources and layer names are inconsistent some
manipulation might be needed so maybe we need more detailed
control. Implementing this is is no big deal and mostly a matter
of figuring out a clean and simple user interface.

\stopsection

\stopchapter

\stopcomponent
