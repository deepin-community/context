% language=us runpath=texruns:manuals/followingup

\startcomponent followingup-mp

\environment followingup-style

\startchapter[title={\METAPOST}]

\startsection[title={Introduction}]

Relatively late in the followup I started wondering about what to do with \MPLIB.
Alan Braslau is working on the \type {luapost} module and we discuss handy
extensions written in \LUA\ and \METAPOST\ code but who knows what more is
needed. Some ideas were put on delay but it looked like a good moment to pick up
on them. One problem is that when we play with the \MPLIB\ code itself in
\LUAMETATEX, the question is how to keep in sync with the official library. In
this chapter I'll discuss both: keeping up with the official code, and keeping
ahead with ideas.

\stopsection

\startsection[title={The code base}]

The \MPLIB\ code is written in \CWEB\ and lives in files with the suffix \type
{w}. These files need to be converted to \type {c} and \type {h} files, something
that is done with the \type {ctangle} program. To avoid that dependency I just
took the \CCODE\ files from \LUATEX, but I had to apply a few patches (to get rid
of dependencies). Now, it is a fact that \METAPOST\ doesn't really develop fast and
in principle a diff could identify the changes easily. So, why shouldn't I also
start experimenting with \MPLIB\ itself in the follow up? It's easy to merge
future changes (in both directions).

The first thing I wrote was a \type {w-to-c} script. This was not that hard given
that I already had written lexers. After a first prototype worked out well, I
redid the code a bit (so that in the future I can also implement support for
change files for instance). A complication was that I found out that the regular
\CWEB\ converter messes around a bit with the code. So, I had to write another
script to mimmick that to the level that I could compare the results. For
example, spaces are removed before and after operators and all leading space gets
removed too. When I got the same output I could get rid of that code and output
what I want. For instance I'd like to keep the spacing the same because compilers
can warn about some issues, like missing \type {;} and misleading indentation in
simple \type {if} and \type {while} constructs where braces are omitted.
\footnote {This is no problem in for instance \PASCAL\ where we always have a
\type {begin} and \type {end}.} One can argue that this is not important, but if
not, then why enable warnings at all. I had to fix half a dozen places in the
\type {w} file to make the compiler happy, so the price was small.

Once I had a more or less instantaneous conversion \footnote {Conversion of the
\type {w} files involved took just over half a second at that time, currently it
takes just over a quarter of a second, on a relatively old machine that is.} I
got the same feeling as with the rest of the code: experimenting became
convenient due to the fast edit|-|compile cycle. So, with al this covered I could
do what I always had wanted to do: remove traces of the backends (including the
full \POSTSCRIPT\ one), because they are actually to be plug|-|ins, and also get
rid of internal font handling, which is bound to \TYPEONE\ (rendering) and small
size \TFM\ (generating). With respect to that export: I wonder if anyone used
that these days because even the Gust font project always had their own tool
chain alongside \METAPOST. I could also void the hacks needed to trick the
library in not being dependent of \type {png.h} and \type {zlib.h} headers, for
which I had to use dummies. \footnote {The converter can load a file with patches
to be applied but by now there are no patches.}

It took a few days scripting the converter (most time went into getting identical
output in order to check the converter which was later dropped), a few days
stripping unused code, another day cleaning up the remaining code and then I
could start playing with some new extensions. The binary has shrunk with 200KB
and the whole \LUAMETATEX\ code base in compressed \type {tar.xz} format is now
below 1.8MB while before it was above 2MB. Not that it matters much, but it was
an nice side effect. \footnote {Size matters as we want to code to end up in the
\CONTEXT\ distribution. It might grow a bit as side effect of adding some more
features to \MPLIB.}

What new extensions would show up was still open. Because Alan and I play with
scanners it made sense to look into that. Error handling and logging has also
been on my radar for a while. In the process some more code might be dropped, but
actually the current version is still useable as library for a stand alone
program, given that one reconstructs the \POSTSCRIPT\ driver from the dropped
code (not that much work). Some configuration options are missing then but that
could be provided as extensions (after all we can have change files.) On the
other hand, wrapping code in \CONTEXT, like:

\starttyping
\starttext
\startMPpage
    ........
\stopMPpage
\startMPpage
    ........
\stopMPpage
\stoptext
\stoptyping

will give a \PDF\ file that can be converted to all kinds of formats, and the
advantage is that one has full font support. There is already a script in the
distribution that does this anyway.

\stopsection

\startsection[title={Communication}]

The first experiment concerns a change in the interfacing between the \METAPOST\
and \LUA\ end. In the original library all file \IO\ is handled by the library
itself. The filenames can be resolved via a callback. Once an instance is
initialized, snippets of code are passed to the instance via the \type {execute}
call. Log, terminal and error information is collected and returned as part of
the return value (a table). This means that reporting back to the user has a
delay: it can be shown {\em after} all code in the buffer has been processed. The
code given as argument to \type {execute} is passed to the engine as (fake)
terminal input, which nicely fits in the concept of interactive input, which
already is part of the \METAPOST\ concept.

In our follow up variant all file \IO\ goes via \LUA. This means that we have a
bit more control over matters. In \CONTEXT\ we now can use the usual file
handling code. One defines an \type {open_file} callback that returns a table
with possible methods \type {close}, \type {reader} and \type {writer}, as in
similar \LUATEX\ callbacks. A special file, with the name \type {terminal} is
used for terminal communication. Now, when the \type {execute} command is
handled, the string that gets passed ends up in the terminal, so the file handler
has to deal with it: the string gets written to the handle, and the handle has to
return it as lines on request. In \CONTEXT\ we directly feed the to be executed
code into the terminal cache.

It's all experimental and subject to changes but as we keep \CONTEXT\ \LMTX\ and
\LUAMETATEX\ in sync, this is no problem. Users will not use these low level
interfaces directly. It might take a few years to settle on this.

The reports that come from the \METAPOST\ engine are now passed on to the \type
{run_logger} callback. That one gets a target and a string passed. Where the
original library can output stuff twice, once for the log and once for the
console, in the new situation it gets output once, with the target being
terminal, log file or both. The nice thing about this callback is that there is no
delay: the messages come as the code is processed.

We combine this logging with the new \type {halt_on_error} flag, which makes the
engine abort after one error. This mechanism will be improved as we go. The
interaction option \type {silent} hides some of the less useful messages.

The overall efficiency of the library doesn't suffer from these changes, and in
some cases it can perform even better. Anyhow, the user experience is much better
with synchronous reports.

Although not strictly related to \IO, we already has extended the library with
the option to support \UTF-8, which is handy for special symbols, as for instance
used in the \type {luapost} library.

\stopsection

\startsection[title={Scanning}]

Another extension is more fundamental in the sense that it can affect the way
users see \METAFUN: extending the user interface. It is again an example of why
is having an independent code base has benefits: we can do such experiments for a
long time, before we decide that (and how) it can end up in the parent (of course
the same is true for the mentioned \IO\ features). I will not discuss these
features here. For now it is enough to know that it gets applied in \CONTEXT\ and
will provide a convenient additional interface. Once it is stable I'll wrap it up
in writing.

\stopsection

\stopchapter

\stopcomponent
