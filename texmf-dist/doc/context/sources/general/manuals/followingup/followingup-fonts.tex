% language=us runpath=texruns:manuals/followingup

\startcomponent followingup-fonts

\environment followingup-style

\startchapter[title={Scaled fonts}]

\startsection[title={History}]

The infrastructure for fonts makes up a large part of the code of any \TEX\ macro
package. We have to go back in time to understand why. When \TEX\ showed up,
fonts were collections of bitmaps and measures. There were at most 256 glyphs in
a font and in order to do its job, \TEX\ needed to know (and still needs to know)
the width, height and depth of glyphs. If you want ligatures it also needs to
know how to construct them from the input and when you want kerning there has to
be additional information about what neighboring glyphs need a kern in between.
Math is yet another subtask that demands extra information, like chains of glyphs
that grow in size and if needed even recipes of how to construct large shapes
from smaller ones.

Fonts come in sizes. Latin Modern and the original Computer Modern, for instance,
have quite a few variants where the shapes are adapted to the size. This means
that when you need a 9pt regular shape alongside a 12pt one, two fonts have to be
loaded. This is quite visible in math where we have three related sizes: text,
script and scriptscript, grouped in so called families. When we scale the digit~2
to the same height you will notice that the text, script and scriptscript sizes
look different (the last three are unscaled):

\startlinecorrection
\dontleavehmode\scale[frame=on,height=5ex]{$\textstyle         2$}\quad
\dontleavehmode\scale[frame=on,height=5ex]{$\scriptstyle       2$}\quad
\dontleavehmode\scale[frame=on,height=5ex]{$\scriptscriptstyle 2$}\quad\quad
\dontleavehmode\scale[frame=on,height=2ex]{$\textstyle         2$}\quad
\dontleavehmode\scale[frame=on,height=2ex]{$\scriptstyle       2$}\quad
\dontleavehmode\scale[frame=on,height=2ex]{$\scriptscriptstyle 2$}\quad\quad
\dontleavehmode\scale[frame=on,width=4em]{\colored[r=.6,a=1,t=.5]{$\textstyle         2$}}\hskip-4em
\dontleavehmode\scale[frame=on,width=4em]{\colored[g=.6,a=1,t=.5]{$\scriptstyle       2$}}\hskip-4em
\dontleavehmode\scale[frame=on,width=4em]{\colored[b=.6,a=1,t=.5]{$\scriptscriptstyle 2$}}\quad\quad
\dontleavehmode{$\textstyle         2$}\quad
\dontleavehmode{$\scriptstyle       2$}\quad
\dontleavehmode{$\scriptscriptstyle 2$}\quad
\stoplinecorrection

Plenty has been written (in various documents that come with \CONTEXT) about how
this all works together and how it impacts the design of the system, so here I
just give a short summary of what a font system has to deal with.

\startitemize
\startitem
    In a bodyfont setup different sizes (9pt, 10pt, 12pt) can have their own
    specific set of fonts. This can result in quite a number of definitions that
    relate to the style, like regular, bold, italic, bold italic, slanted, bold
    slanted, etc. When possible loading the fonts is delayed. In \CONTEXT\ often
    the number of fonts that are actually loaded is not that large.
\stopitem
\startitem
    Some font designs have different shapes per bodyfont size. A minor
    complication is that when one is missing some heuristic best|-|match choice
    might be needed. Okay, in practice only Latin Modern falls into this category
    for \CONTEXT. Maybe \OPENTYPE\ variable fonts can be seen this way, but,
    although we supported that right from the start, I haven't noticed much
    interest in the \TEX\ community.
\stopitem
\startitem
    Within a bodyfont size we distinguish size variants. We can go smaller (x and
    xx), for instance when we use sub- and superscripts in text, or we can go
    larger, for instance in titles (a, b, c, d, \unknown). Fortunately most of
    the loading of these can be delayed too.
\stopitem
\startitem
    When instances are not available, scaling can be used, as happens for
    instance with 11pt in Computer Modern. Actually, this is why in \CONTEXT\ we
    default to 12pt, because the scaled versions didn't look as nice as the
    others (keep in mind that we started in the age of bitmaps).
\stopitem
\startitem
    Special features, such as smallcaps or oldstyle numerals, can demand their own
    definitions. More loading and automatic definitions can be triggered by sizes
    needed in, e.g., scripts and titles.
\stopitem
\startitem
    A document can have a mixed setup, that is: using different font
    designs within one
    document, so some kind of namespace subsystem is needed.
\stopitem
\startitem
    In an eight|-|bit font world, we not only have text fonts but also collections
    of symbols, and even in math there are additional symbol collections. In
    \OPENTYPE\ symbols end up in text fonts, but there we have tons of emojis
    and color fonts. All has to be dealt with in an integrated way. And we're
    not even talking of virtual fonts, (runtime) \METAPOST\ generated fonts, and
    so on.
\stopitem
\startitem
    In traditional eight|-|bit engines, hyphenation depends on a font's encoding,
    which can require loading a font multiple times in different encodings. This
    depends on the language mix used. A side point is that defining a European
    encoding covering most Latin languages was not that hard, especially when one
    keeps in mind that many eight|-|bit encodings waste slots on seldom used
    symbols, but by that time \OPENTYPE\ and \UNICODE\ input started to dominate.
\stopitem
\startitem
    In the more modern \OPENTYPE\ fonts combinations of features can demand
    additional instances: one can think of language|/|script combinations,
    substitutions in base mode, special effects like emboldening, color fonts,
    etc.
\stopitem
\startitem
    Math is complicated by the fact that in traditional \TEX, alphabets come from
    different fonts, which is why we have many so|-|called families; a font can
    have several alphabets which means that some mapping can be needed. Operating
    on the size, shape, encoding and style axes puts some demands on the font
    system. Add to this the (often) partial (due to lack of fonts) bold support
    and it gets even more complicated. In \OPENTYPE\ all the alphabets come from
    one font.
\stopitem
\startitem
    There is additional math auto|-|definition and loading code for the sizes
    used in text scripts and titles.
\stopitem
\stopitemize

All this has resulted in a pretty complex subsystem. Although going \OPENTYPE\
(and emulated \OPENTYPE\ with \TYPEONE\ fonts as we do in \MKIV) removes some
complications, like encodings, it also adds complexity because of the many
possible font features, either dependent or not on script and language. Text as
well as math got simpler in the \TEX\ code, though that was traded for quite a
bit of \LUA\ code to deal with new features.

So, in order to let the font subsystem not impact performance too much, let alone
extensive memory usage, the \CONTEXT\ font subsystem is rather optimized. The
biggest burden comes from fonts that have a dynamic (adaptive) definition because
then we need to do quite a bit of testing per font switch, but even that has
always been rather fast.

\stopsection

\startsection[title={Reality}]

In \MKIV\ and therefore also in \LUAMETATEX\ (\LMTX) more font magic happens. The
initial node lists that make up a box or paragraph can get manipulated in several
ways and often fonts are involved. The font features (smallcaps, oldstyle,
alternates, etc.)\ can be defined as static (part of the definition) or as
dynamic (resolved on the spot at the cost of some overhead). Characters can be
remapped, fonts can be replaced. The math subsystem in \MKIV\ was different right
from the start: we use a limited number of families (regular, bold, l2r and r2l),
and stay abstract till the moment we need to deal with the specific alphabets.
But still, in \MKIV, we have the families with three fonts.

In the \LUAMETATEX\ manual we show some math magic for different fonts. As a side
effect, we set up half a dozen bodyfont collections: Lucida, Pagella, Latin
Modern, Dejavu, the math standard Cambria, etc. Even with delayed and shared font
loading, we end up with 158 instances but quite a few of them are math fonts, at
least six per bodyfont size: regular and bold (emboldened) text, script and
scriptscript. Of course most are just copies with different scaling that reuse
already loaded resources. In the final \PDF\ we have 21 subsetted fonts.

If we look at the math fonts that we use today, there is however quite some
overlap. It starts with a text font. From that, script and scriptscript variants
are derived, but often these variants use many text size related shapes too. Some
shapes get alternatives (from the \type {ssty} feature), and the whole clone gets
scaled. But, much of the logic of, for instance, extensibles is the same.

A similar situation happens with large \CJK\ fonts: there are hardly any advanced
features involved there, so any size is basically a copy with scaled dimensions,
and these fonts can be truly huge!

When we talk about features, in many cases in \CONTEXT\ you don't define them as
part of the font. For instance small caps can best be triggered by using a
dynamic feature: applied to a specific stretch of text. In fact, often features
like superiors of fractions only work well on characters that fit the bill and
produce weird side effects otherwise (a matter of design completeness). When the
font handler does its work there are actually four cases: no features get applied
(something that happens with, for instance, most monospaced fonts); base mode is
used (which means that the \TEX\ machinery takes care of constructing ligatures
and injecting kerns); and node mode (where \LUA\ handles the features). The
fourth case is a special case of node mode where a different feature set is
applied. \footnote {We also have so|-|called plug mode where an external renderer
can do the work but that one is only around due to some experiments during Idris
Hamid's font development.} At the cost of some extra overhead (for each node mode
run) dynamic features are quite powerful and save quite a lot of memory and
definitions. \footnote {The generic font handler that is derived from the
\CONTEXT\ one doesn't implement this, so it runs a little faster.} The overhead
comes from much more testing regarding the font we deal with because suddenly the
same font can demand different treatments, depending on what dynamic features are
active. \footnote {Originally this model was introduced for a dynamic paragraph
optimization subsystem for Arabic but in practice no one uses it because there
are no suitable fonts.}

Although the font handling is responsible for much of the time spent in \LUA, it
is still reasonable given what has to be done. Because we have an extensible
system, it's often the extensions that takes additional runtime. Flexibility
comes at a price.

\stopsection

\startsection[title={Progress}]

At some point I started playing with realtime glyph scaling. Here realtime means
that it doesn't depend on the font definition. To get an idea, here is an example
(all examples are additionally scaled for \TUGBOAT):

\startbuffer
test {\glyphxscale 2500 test} test
\stopbuffer

\typebuffer

\getbuffer

The glyphs in the current font get scaled horizontally without the need for an
extra font instance. Now, this kind of trickery puts some constraints on the font
handling, as is demonstrated in the next example. We use Latin Modern because
that font has all these ligatures:

\startbuffer
\definedfont[lmroman10-regular*default]%
e{\glyphxscale 2500 ff}icient
ef{\glyphxscale 2500 f}icient
ef{\glyphxscale 2500 fi}cient
e{\glyphxscale 2500 ffi}cient
\stopbuffer

\typebuffer

{\getbuffer}

In order to deal with this kind of scaling, we now operate not only on the font
(id) and dynamic feature axes, but also on the scales, of which we have three
variants: glyph scale, glyph xscale and glyph yscale. There is actually also a
state dimension but we omit that for now (think of flagging glyphs as initial or
final). This brings the number of axes to six. It is important to stress that in
these examples the same font instance is used!

Just for the record: several approaches to switching fonts are possible but for
now we stick to a simple font id switch plus glyph scale settings at the \TEX\
end. A variant would be to introduce a new mechanism where id's and scales go
together but for now I see no real gain in that.

\stopsection

\startsection[title={Math}]

Given what has been discussed in the previous sections, a logical question would
be \quotation {Can we apply scaling to math?} and the answer is \quotation {Yes,
we can!}. We can even go a bit further and that is partly due to some other
properties of the engine.

From \PDFTEX\ the \LUATEX\ engines inherited character protrusion and glyph
expansions, aka hz. However, where in \PDFTEX\ copies of the font are made that
carry the expanded dimensions, in \LUATEX\ at some point this was replaced by an
expansion field in the glyph and kern nodes. So, instead of changing the font id
of expanded glyphs, the same id is used but with the applied expansion factor set
in the glyph. A side effect was that in places where dimensions are needed, we
call functions that calculate the expanded widths on request (as these can change
during linebreak calculations) in combination with accessing font dimensions
directly. This level of abstraction is even more present in \LUAMETATEX. This
means that we have an uniform interface to fonts and as a side effect scaling
need be dealt with in only a few places in the code.

Now, in math we have a few more complications. First of all, we have three sizes
to consider and we also have lots of parameters that depend on the size. But, as
I wanted to be able to apply scaling to math, the whole machinery was also
abstracted in a way that, at the cost of some extra overhead, made it easier to
work with scaled glyph properties. This means that we can stick to loading only
one bodyfont size of math (note that each math family has three sizes, where the
script and script sizes can have different, fine tuned, shapes) and just scale
that on demand.

Once all that was in place it was a logical next step to see if we could stick to
just a single instance. Because in \LUAMETATEX\ we try to load fonts efficiently
we store only the minimally needed information at the \TEX\ end. A font with no
math therefore has less data per glyph. Again, this brings some abstraction that
helped to implement the one instance mechanism. A math glyph has optional lists
of increasing sizes and vertical or horizontal extensibles. So what got added was
an optional chain of smaller sizes. If a character has three different glyphs for
the three sizes, the text glyph has a pointer to the script glyph which in turn
has a pointer to the scriptscript glyph. This means that when the math engine
needs a specific character at a given size (text, script, scriptscript) we just
follow that chain.

In an \OPENTYPE\ math font the script and scriptscript sizes are specified as
percentages of the text size. When the dimensions of a glyph are needed, we just
scale on the fly. Again this adds some overhead but I'm pretty sure that no user
will notice.

So, to summarize: if we need a character at scriptscript size, we access the text
size glyph, check for a pointer to a script size, go there, and again check for a
smaller size. We use only what fits the bill. And, when we need dimensions we
just scale. In order to scale we need the relative size, so we need to set that
up when we load the font. Because in \CONTEXT\ we also can assemble a virtual
\OPENTYPE\ font from \TYPEONE\ fonts, it was actually that (old) compatibility
feature, the one that implements \TYPEONE\ based on \OPENTYPE\ math, that took
the most time to adapt, not so much because it is complicated but because in
\LMTX\ we have to bypass some advanced loading mechanisms. Because we can scale
in two dimensions the many (font|-|related) math parameters also need to be dealt
with accordingly.

The end result is that for math we now only need to define two fonts per bodyfont
setup: regular and bold at the natural scale (normally 10pt) and we share these
for all sizes. As a result of this and what we describe in the next section, the
158 instances for the \LUAMETATEX\ manual can be reduced to~30.


\stopsection

\startsection[title={Text}]

Sharing instances in text mode is relatively simple, although we do have to keep
in mind that scaling is an extra axis when dealing with font features: two
neighboring glyphs with the same font id and dynamics but with different scales
are effectively from different fonts.

Another complication is that when we use font fallbacks (read:\ take missing
glyphs from another font) we no longer have a dedicated instance but use a shared
one. This in itself is not a problem but we do need to handle specified relative
scales. This was not that hard to patch in \CONTEXT\ \LMTX.

We can enforce aggressive font sharing with:

\starttyping
\enableexperiments[fonts.compact]
\stoptyping

After that we often use fewer instances. Just to give an idea, on the \LUAMETATEX\
manual we get these stats:

\starttyping
290 pages, 10.8 sec, 292M lua, 99M tex, 158 instances
290 pages,  9.5 sec, 149M lua, 35M tex,  30 instances
\stoptyping

So, we win on all fronts when we use this glyph scaling mechanism. The magic
primitive that deals with this is named \type {\glyphscale}; it accepts a number,
where \type {1200} means scaling to 20\percent\ more than normal. But it's best
not to use this primitive directly.

A specific scaled font can be defined using the \type {\definefont} command. In
\LMTX\ a regular scaler can be followed by two scale factors. The next example
demonstrates this (as can be seen, the \type {yoffset} affects the baseline):

\startbuffer
\definefont[FooA][Serif*default @ 12pt 1800 500]
\definefont[FooB][Serif*default @ 12pt 0.85 0.4]
\definefont[FooC][Serif*default @ 12pt]

\definetweakedfont[runwider] [xscale=1.5]
\definetweakedfont[runtaller][yscale=2.5,xscale=.8,yoffset=-.2ex]

{\FooA test test \runwider test test \runtaller test test}\par
{\FooB test test \runwider test test \runtaller test test}\par
{\FooC test test \runwider test test \runtaller test test}\par
\stopbuffer

\typebuffer

We also use the new \type {\definetweakedfont} command here. This example not
only shows the two scales but also introduces the offset.

\getbuffer

In compact mode this is one font. Here is another example:

\startbuffer
\definetweakedfont[squeezed][xscale=0.9]

\startlines
$a = b^2 + \sqrt{c}$
{\squeezed $a = b^2 + \sqrt{c}$}
\stoplines
\stopbuffer

\typebuffer

\getbuffer

Watch this:

\startbuffer
\startcombination[3*1]
    {\bTABLE
        \bTR \bTD foo \eTD \bTD[style=\squeezed] $x = 1$ \eTD \eTR
        \bTR \bTD oof \eTD \bTD[style=\squeezed] $x = 2$ \eTD \eTR
     \eTABLE}
    {local}
    {\bTABLE[style=\squeezed]
        \bTR \bTD $x = 1$ \eTD \bTD  $x = 3$ \eTD \eTR
        \bTR \bTD $x = 2$ \eTD \bTD  $x = 4$ \eTD \eTR
     \eTABLE}
    {global}
    {\bTABLE[style=\squeezed\squeezed]
        \bTR \bTD $x = 1$ \eTD \bTD  $x = 3$ \eTD \eTR
        \bTR \bTD $x = 2$ \eTD \bTD  $x = 4$ \eTD \eTR
     \eTABLE}
    {multiple}
\stopcombination
\stopbuffer

\typebuffer

\startlinecorrection
\getbuffer
\stoplinecorrection

An additional style parameter is also honored:

\startbuffer
\definetweakedfont[MyLargerFontA][scale=2000,style=bold]
test {\MyLargerFontA test} test
\stopbuffer

\typebuffer

This gives:

\getbuffer

Just for the record: the Latin Modern fonts, when set up to use design sizes,
will still use the specific size|-|related files.

\stopsection

\startsection[title={Hackery}]

You can use negative scale values, as is demonstrated in the following
code:

\startbuffer
\bTABLE[align=middle]
  \bTR
    \bTD a{\glyphxscale  1000 \glyphyscale  1000 bc}d \eTD
    \bTD a{\glyphxscale  1000 \glyphyscale -1000 bc}d \eTD
    \bTD a{\glyphxscale -1000 \glyphyscale -1000 bc}d \eTD
    \bTD a{\glyphxscale -1000 \glyphyscale  1000 bc}d \eTD
  \eTR
  \bTR
    \bTD \tttf +1000 +1000 \eTD
    \bTD \tttf +1000 -1000 \eTD
    \bTD \tttf -1000 -1000 \eTD
    \bTD \tttf -1000 +1000 \eTD
  \eTR
\eTABLE
\stopbuffer

\typebuffer

gives:

\startlinecorrection
\getbuffer
\stoplinecorrection

Glyphs can have offsets and these are used for implementing \OPENTYPE\ features.
However, they are also available on the \TEX\ side. Take this example where we
use the new \type {\glyph} primitive (a variant of \type {\char} that takes
keywords):

\startbuffer
\ruledhbox{
    \ruledhbox{\glyph yoffset 1ex options 0 123} % left curly brace
    \ruledhbox{\glyph xoffset .5em yoffset 1ex options "C0 123}
    \ruledhbox{oeps{\glyphyoffset 1ex \glyphxscale 800
                    \glyphyscale\glyphxscale oeps}oeps}
}
\stopbuffer

\typebuffer \getbuffer

This example demonstrates that the \type {\glyph} primitive takes quite a few
keywords: \type {xoffset}, \type {yoffset}, \type {xscale}, \type {yscale}, \type
{left}, \type {right}, \type {raise}, \type {options}, \type {font} and \type
{id} where the last two take a font identifier or font id (a positive number).
For this article it's enough to know that the option indicates that glyph
dimension should include the offset. In a moment we will see an alternative that
doesn't need that.

\startbuffer
\samplefile{jojomayer}
{\glyphyoffset .8ex
 \glyphxscale 700 \glyphyscale\glyphxscale
 \samplefile{jojomayer}}
{\glyphyscale\numexpr3*\glyphxscale/2\relax
 \samplefile{jojomayer}}
{\glyphyoffset -.2ex
 \glyphxscale 500 \glyphyscale\glyphxscale
 \samplefile{jojomayer}}
\samplefile{jojomayer}
\stopbuffer

\typebuffer

To quote Jojo Mayer:

\startnarrower
\darkred \getbuffer
\stopnarrower

Keep in mind that this can interfere badly with font feature processing which also
used offsets. It might often work out okay vertically, but less well horizontally.

The scales, as mentioned, works with pseudo|-|scales but that is sometimes a bit
cumbersome. This is why a special \type {\numericscale} primitive has been
introduced.

\startbuffer
1200 : \the\numericscale1200
1.20 : \the\numericscale1.200
\stopbuffer

\typebuffer

Both these lines produce the same integer:

\startlines\tttf
\getbuffer
\stoplines

You can do strange things with these primitives but keep in mind that you can
also waste the defaults.

\startbuffer[definition]
\def\UnKernedTeX
  {T%
   {\glyph xoffset -.2ex yoffset -.4ex `E}%
   {\glyph xoffset -.4ex options "60 `X}}
\stopbuffer

\startbuffer[example]
We use \UnKernedTeX\ and {\bf \UnKernedTeX} and {\bs \UnKernedTeX}:
the slanted version could use some more left shifting of the E.
\stopbuffer

\typebuffer[definition,example]

This gives the \TEX\ logos but of course we normally use the more official
definitions instead.

\startnarrower
    \darkred \getbuffer[definition,example]
\stopnarrower

Because offsets are (also) used for handling font features like mark and cursive
placement as well as special inter|-|character positioning, the above is
suboptimal. Here is a better alternative:

\startbuffer[definition]
\def\UnKernedTeX
  {T\glyph left .2ex raise -.4ex `E\glyph left .2ex `X\relax}
\stopbuffer

\typebuffer[definition]

The result is the same:

\startnarrower
    \darkgreen \getbuffer[definition,example]
\stopnarrower

But anyway: don't overdo it. We have dealt with such cases for decades without
these fancy new features. The next example shows margins in action:

\startlinecorrection
\bTABLE[align=middle,width=.33\textwidth]
    \bTR
        \bTD \showglyphs \multiply\glyphscale by 2 <{\darkgray \glyph                      `M}>\eTD
        \bTD \showglyphs \multiply\glyphscale by 2 <{\darkgray \glyph raise 3pt            `M}>\eTD
        \bTD \showglyphs \multiply\glyphscale by 2 <{\darkgray \glyph raise -3pt           `M}>\eTD
    \eTR
    \bTR[frame=off]
        \bTD \tttf            \eTD
        \bTD \tttf raise  3pt \eTD
        \bTD \tttf raise -3pt \eTD
    \eTR
    \bTR
        \bTD \showglyphs \multiply\glyphscale by 2 <{\darkgray \glyph left  3pt            `M}>\eTD
        \bTD \showglyphs \multiply\glyphscale by 2 <{\darkgray \glyph           right  2pt `M}>\eTD
        \bTD \showglyphs \multiply\glyphscale by 2 <{\darkgray \glyph left  3pt right  2pt `M}>\eTD
    \eTR
    \bTR[frame=off]
        \bTD \tttf left  3pt           \eTD
        \bTD \tttf           right  2pt\eTD
        \bTD \tttf left  3pt right  2pt\eTD
    \eTR
    \bTR
        \bTD \showglyphs \multiply\glyphscale by 2 <{\darkgray \glyph left -3pt            `M}>\eTD
        \bTD \showglyphs \multiply\glyphscale by 2 <{\darkgray \glyph           right -2pt `M}>\eTD
        \bTD \showglyphs \multiply\glyphscale by 2 <{\darkgray \glyph left -3pt right -2pt `M}>\eTD
    \eTR
    \bTR[frame=off]
        \bTD \tttf left -3pt            \eTD
        \bTD \tttf           right -2pt \eTD
        \bTD \tttf left -3pt right -2pt \eTD
    \eTR
\eTABLE
\stoplinecorrection

Here is another way of looking at it:

\startbuffer
\glyphscale 4000
\vl\glyph                        `M\vl\quad
\vl\glyph raise  .2em            `M\vl\quad
\vl\glyph left   .3em            `M\vl\quad
\vl\glyph             right  .2em`M\vl\quad
\vl\glyph left  -.2em right -.2em`M\vl\quad
\vl\glyph raise -.2em right  .4em`M\vl
\stopbuffer

\typebuffer

The raise as well as left and right margins are taken into account when calculating the
dimensions of a glyph.

{\getbuffer}

\stopsection

\startsection[title={Implementation}]

Discussing the implementation in the engine makes no sense here, also because
details might change. However, it is good to know that many properties travel
with the glyph nodes, for instance the scales, margins, offsets, language, script
and state properties, control over kerning, ligaturing, expansion and protrusion,
etc. The dimensions (width, height and depth) are not stored in the glyph node
but calculated from the font, scales and optionally the offsets and expansion
factor. One problem is that the more clever (and nice) solutions we cook up, the
more it might impact performance. So, I will delay some experiments till I have a
more powerful machine.

One reason for {\em not} storing the dimensions in a glyph node is that we often
copy those nodes or change character fields in the font handler and we definitely
don't want the wrong dimensions there. At that moment, offsets and margin fields
don't reflect features yet, so copying them is no big deal because at that moment
these are still zero. However, dimensions are rather character bound so every
time a character is set, we also would have to set the dimensions. Even worse,
when we can set them, the question arises if they were already set explicitly.
So, this is a can of worms we're not going to open: the basic width, height and
depth of the glyph as specified in the font is used and combined with actual
dimensions (likely already scaled according the glyph scales) in offset and
margin fields.

Now, I have to admit that especially playing with using margins to glyphs instead
of font kerns is more of an experiment to see what the consequences are than a
necessity, but what would be the joy of \TEX\ without such experiments? And as
usual, in \CONTEXT\ these will become options in the font handler that one can
enable, or not.

\stopsection

\stopchapter

\stopcomponent

% As often there is a musical timestamp to this text: discovering Julia Hofer
% (bass, came there via checking out (of course) Leland Sklar, and now i have to
% watch a bunch of Victor Wooten videos too) and Neon Vine's (play and combine
% anything) on YT. Both examples of where (imo extensive) practicing real
% instruments can bring you with todays diverse equipment. These are interesting
% times (it's now last days of 2020).
%
% The initial \definescaledfont was replaced by \definetweakedfont in the first
% week of Januari 2021, when I watched the joyful covers of Peter Gabriels
% "Steam" by Yoyoka which shows that the future belongs to the real young kids
% (so much explosive creativity!) and not to the old folks.
