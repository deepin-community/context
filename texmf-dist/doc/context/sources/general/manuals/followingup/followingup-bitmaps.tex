% language=us runpath=texruns:manuals/followingup

\startcomponent followingup-bitmaps

\environment followingup-style

\startchapter[title={Bitmap images}]

\startsection[title={Introduction}]

In \TEX\ image inclusion is traditionally handled by specials. Think of a signal
added someplace in the page stream that says:

\starttyping
\special{image: foo.png 2000 3000}
\stoptyping

Here the number for instance indicate a scale factor to be divided by 1000.
Because \TEX\ has no floating point numbers, normally one uses an integer and the
magic multiplier 1000 representing 1.000. Such a special is called a \quote
{whatsit} and is one reason why \TEX\ is so flexible and adaptive.

In \PDFTEX\ instead of a \type {\special} the command \type {\pdfximage} and its
companions are used. In \LUATEX\ this concept has been generalized to \type
{\useimageresource} which internally is not a so called whatsit (an extension
node) but a special kind of rule. This makes for nicer code as now we don't need
to check if a certain whatsit node is actually one with dimensions, while rules
already are part of calculating box dimensions, so no extra overhead in checking
for whatsits is added. In retrospect this was one of the more interesting
conceptual changes in \LUATEX.

In \LUAMETATEX\ we don't have such primitives but we do have these special rule
nodes; we're talking of subtypes and the frontend doesn't look at those details.
Depending on what the backend needs one can easily define a scanner that
implements a primitive. We already did that in \CONTEXT. More important is that
inclusion is not handled by the engine simply because there is no backend. This
means that we need to do it ourselves. There are two steps involved in this that
we will discuss below.

\stopsection

\startsection[title={Identifying}]

There is only a handful of image formats that makes sense in a typesetting
workflow. Because \PDF\ inclusion is supported (but not discussed here) one can
actually take any format as long as it converts to \PDF, and tools like graphic
magic do a decent job on that. \footnote {Although one really need to check a
converted image. When we moved to pplib, I found out that lots of converted
images in a project had invalid \PDF\ objects, but apart from a warning nothing
bad resulted from this because those objects were not used.} The main bitmap
formats that we care about are \JPEG, \JPEG2000, and \PNG. We could deal with
\JBIG\ files but I never encountered them so let's forget about them for now.

One of the problems with a built|-|in analyzer (and embedder) is that it can
crash or just abort the engine. The main reason is that when the used libraries
run into some issue, the engine is not always able to recover from it: a
converter just aborts which then cleans up (potentially messed up) memory. In
\LUATEX\ we also abort, simply because we have no clue to what extend further on
the libraries are still working as expected. We play safe. For the average user
this is quite ok as it signals that an image has to be fixed.

In a workflow that runs unattended on a server and where users push images to a
resource tree, there is a good change that a \TEX\ job fails because of some
problem with images. A crash is not really an option then. This is one reason why
converting bitmaps to \PDF\ makes much sense. Another reason is that some color
profiling might be involved. Runtime manipulations make no sense, unless there is
only one typesetting run.

Because in \LMTX\ we do the analyzing ourselves \footnote {Actually, in \MKIV\
this was also possible but not widely advertised, but we now exclusively keep
this for \LMTX.} we can recover much easier. The main reason is of course that
because we use \LUA, memory management and garbage collection happens pretty well
controlled. And crashing \LUA\ code can easily be intercepted by a \type {pcall}.

Most (extensible) file formats are based on tables that gets accessed from an
index of names and offsets into the file. This means that filtering for instance
metadata like dimensions and resolutions is no big deal (we always did that). I
can extend analyzing when needed without a substantial change in the engine that
can affect other macro packages. And \LUA\ is fast enough (and often faster) for
such tasks.

\stopsection

\startsection[title={Embeding}]

Once identified the frontend can use that information for scaling and (if needed)
reuse of the same image. Embedding of the image resource happens when a page is
shipped out. For \JPEG\ images this is actually quite simple: we only need to
create a dictionary with the right information and push the bitmap itself into
the associated stream.

For \PNG\ images it's a bit different. Unfortunately \PDF\ only supports certain
formats, for instance masks are separated and transparency needs to be resolved.
This means that there are two routes: either pass the bitmap blob to the stream,
or convert it to a suitable format supported by \PDF. In \LUATEX\ that is
normally done by the backend code, which uses a library for this. It is a typical
example of a dependency of something much larger than actually needed. In
\LUATEX\ the original poppler library used for filtering objects from a \PDF\
file as well as the \PNG\ library also have tons of code on board that relates to
manipulating (writing) data. But we don't need those features. As a side note:
this is something rather general. You decide to use a small library for a simple
task only to find out after a decade that it has grown a lot offering features
and having extra dependencies that you really don't want. Even worse: you end up
with constant updates due to fixed security (read: bug) fixes.

Passing the \PNG\ blob unchanged in itself to the \PDF\ file is trivial, but
massaging it into an acceptable form when it doesn't suit the \PDF\ specification
takes a bit more code. In fact, \PDF\ does not really support \PNG\ as format,
but it supports \PNG\ compression (aka filters).

Trying to support more complex \PNG\ files is a nice way to test if you can
transform a public specification into a program as for instance happens with
\PDF, \OPENTYPE, and font embedding in \CONTEXT. So this again was a nice
exercise in coding. After a while I was able to process the \PNG\ test suite
using \LUA. Optimizing the code came with understanding the specification.
However, for large images, especially interlaced ones, runtime was definitely not
to be ignored. It all depended on the tasks at hand:

\startitemize

\startitem
    A \PNG\ blob is compressed with \ZIP\ compression, so first it needs to be
    decompressed. This takes a bit of time (and in the process we found out that
    the \type {zlib} library used in \LUATEX\ had a bug that surfaced when a
    mostly zero byte image was uncompressed and we can then hit a filled up
    buffer condition.
\stopitem

\startitem
    The resulting uncompressed stream is itself compressed with a so called
    filter. Each row starts with a filter byte that indicates how to convert
    bytes into other bytes. The most commonly used methods are deltas with
    preceding pixels and/or pixels on a previous row. When done the filter bytes
    can go away.
\stopitem

\startitem
    Sometimes an image uses 1, 2 or 4 bits per pixel, in which case the rows
    needs to be expanded. This can involve a multiplication factor per pixel (it
    can also be an index in a palette).
\stopitem

\startitem
    An image can be interlaced which means that there are seven parts of the
    image that stepwise build up the whole. In professional workflows with high
    res images interlacing makes no sense as transfer over the internet is not an
    issue and the overhead due to reassembling the image and the potentially
    larger file size (due to independent compression of the seven parts) are not
    what we want either.
\stopitem

\startitem
    There can be an image mask that needs to be separated from the main blob. A
    single byte gray scale image then has two bytes per pixel, and a double byte
    pixel has four bytes of information. An \RGB\ image has three bytes per pixel
    plus an alpha byte, and in the case of double byte pixels we get eight bytes
    per pixel.
\stopitem

\startitem
    Finally the resulting blob has to be compressed again. The current amount of
    time involved in that suggests that there is room for improvement.
\stopitem

\stopitemize

The process is controlled by number of rows and columns, the number of bytes per
pixel (one or two) and the color space which effectively means one or three
bytes. These numbers get fed into the filter, deinterlacer, expander and|/|or
mask separator. In order to speed up the embedding these basic operations can be
assisted by a helpers written in \CCODE. Because \LUA\ is quite good with
strings, we pass strings and get back strings. So, most of the logic stays at the
\LUA\ end.

\stopsection

\startsection[title=Conclusion]

Going for a library|-|less solution for bitmap inclusion is quite doable and in
most cases as efficient. Because we have a pure \LUA\ implementation for testing
and an optimized variant for production, we can experiment as we like. A positive
side effect is that we can more robustly intercept bad images and inject a
placeholder instead.

\stopsection

\stopchapter

\stopcomponent
