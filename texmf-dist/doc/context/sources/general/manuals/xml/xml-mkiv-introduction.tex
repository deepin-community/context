% language=us runpath=texruns:manuals/xml

\environment xml-mkiv-style

\startcomponent xml-mkiv-introduction

\startchapter[title={Introduction}]

This manual presents the \MKIV\ way of dealing with \XML. Although the
traditional \MKII\ streaming parser has a charming simplicity in its control, for
complex documents the tree based \MKIV\ method is more convenient. It is for this
reason that the old method has been removed from \MKIV. If you are familiar with
\XML\ processing in \MKII, then you will have noticed that the \MKII\ commands
have \type {XML} in their name. The \MKIV\ commands have a lowercase \type {xml}
in their names. That way there is no danger for confusion or a mixup.

You may wonder why we do these manipulations in \TEX\ and not use \XSLT\ (or
other transformation methods) instead. The advantage of an integrated approach is
that it simplifies usage. Think of not only processing the document, but also
using \XML\ for managing resources in the same run. An \XSLT\ approach is just as
verbose (after all, you still need to produce \TEX\ code) and probably less
readable. In the case of \MKIV\ the integrated approach is also faster and gives
us the option to manipulate content at runtime using \LUA. It has the additional
advantage that to some extend we can handle a mix of \TEX\ and \XML\ because we
know when we're doing one or the other.

This manual is dedicated to Taco Hoekwater, one of the first \CONTEXT\ users, and
also the first to use it for processing \XML. Who could have thought at that time
that we would have a more convenient way of dealing with those angle brackets.
The second version for this manual is dedicated to Thomas Schmitz, a power user
who occasionally became victim of the evolving mechanisms.

\blank

\startlines
Hans Hagen
\PRAGMA
Hasselt NL
2008\endash2016
\stoplines

\stopchapter

\stopcomponent
