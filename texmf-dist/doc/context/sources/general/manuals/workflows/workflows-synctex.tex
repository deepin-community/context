% language=uk

\environment workflows-style

\startcomponent workflows-xml

\startchapter[title=\SYNCTEX]

\startsection[title=Introduction]

Some users like the \SYNCTEX\ feature that is built in the \TEX\ engines.
Personally I never use it because it doesn't work well with the kind of documents
I maintain. If you have one document source, and don't shuffle around (reuse)
text too much it probably works out okay but that is not our practice. Here I
will describe how you can enable a more \CONTEXT\ specific \SYNCTEX\ support so
that aware \PDF\ viewers can bring you back to the source.

\stopsection

\startsection[title=What we want]

The \SYNCTEX\ method roughly works as follows. Internally \TEX\ constricts linked
lists of glyphs, kerns, glue, boxes, rules etc. These elements are called nodes.
Some nodes carry information about the file and line where they were created. In
the backend this information gets somehow translated in a (sort of) verbose tree
that describes the makeup in terms of boxes, glue and kerns. From that
information the \SYNCTEX\ parser library, hooked into a \PDF\ viewer, can go back
from a position on the screen to a line in a file. One would expect this to be a
relative simple rectangle based model, but as far as I can see it's way more
complex than that. There are some comments that \CONTEXT\ is not supported well
because it has a layered page model, which indicates that there are some
assumptions about how macro packages are supposed to work. Also the used
heuristics not only involve some specific spot (location) but also involve the
corners and edges. It is therefore not so much a (simple) generic system but a
mechanism geared for a macro package like \LATEX.

Because we have a couple of users who need to edit complex sets of documents,
coded in \TEX\ or \XML, I decided to come up with a variant that doesn't use the
\SYNCTEX\ machinery but manipulates the few \SYNCTEX\ fields directly \footnote {This
is something that in my opinion should have been possible right from the start
but it's too late now to change the system and it would not be used beyond
\CONTEXT\ anyway.} and eventually outputs a straightforward file for the editor.
Of course we need to follow some rules so that the editor can deal with it. It
took a bit of trial and error to get the right information in the support file
needed by the viewer but we got there.

The prerequisites of a decent \CONTEXT\ \quotation {click on preview and goto
editor} are the following:

\startitemize

\startitem
    It only makes sense to click on text in the text flow. Headers and footers
    are often generated from structure, and special typographic elements can
    originate in macros hooked into commands instead of in the source.
\stopitem

\startitem
    Users should not be able to reach environments (styles) and other files
    loaded from the (normally read|-|only) \TEX\ tree, like modules. We don't
    want accidental changes in such files.
\stopitem

\startitem
    We not only have \TEX\ files but also \XML\ files and these can normally
    flush in rather arbitrary ways. Although the concept of lines is sort of
    lost in such a file, there is still a relation between lines and the snippets
    that make out the content of an \XML\ node.
\stopitem

\startitem
    In the case of \XML\ files the overhead related to preserving line
    numbers should be minimal and have no impact on loading and memory when
    these features are not used.
\stopitem

\startitem
    The overhead in terms of an auxiliary file size and complexity as well
    as producing that file should be minimal. It should be easy to turn on and
    off these features. (I'd never turn them on by default.)
\stopitem

\stopitemize

It is unavoidable that we get more run time but I assume that for the average user
that is no big deal. It pays off when you have a workflow when a book (or even a
chapter in a book) is generated from hundreds of small \XML\ files. There is no
overhead when \SYNCTEX\ is not used.

In \CONTEXT\ we don't use the built|-|in \SYNCTEX\ features, that is: we let
filename and line numbers be set but often these are overloaded explicitly. The
output file is not compressed and constructed by \CONTEXT. There is no benefit in
compression and the files are probably smaller than default \SYNCTEX\ anyway.

\stopsection

\startsection[title=Commands]

Although you can enable this mechanism with directives it makes sense to do it
using the following command.

\starttyping
\setupsynctex[state=start]
\stoptyping

The advantage of using an explicit command instead of some command line option is
that in an editor it's easier to disable this trickery. Commenting that line will
speed up processing when needed. This command can also be given in an environment
(style). On the command line you can say

\starttyping
context --synctex somefile.tex
\stoptyping

A third method is to put this at the top of your file:

\starttyping
% synctex=yes
\stoptyping

Often an \XML\ files is very structured and although probably the main body of
text is flushed as a stream, specific elements can be flushed out of order. In
educational documents flushing for instance answers to exercises can happen out of
order. In that case we still need to make sure that we go to the right spot in
the file. It will never be 100\% perfect but it's better than nothing. The
above command will also enable \XML\ support.

If you don't want a file to be accessed, you can block it:

\starttyping
\blocksynctexfile[foo.tex]
\stoptyping

Of course you need to configure the viewer to respond to the request for
editing. In Sumatra combined with SciTE the magic command is:

\starttyping
c:\data\system\scite\wscite\scite.exe "%f" "-goto:%l"
\stoptyping

Such a command is independent of the macro package so you can just consult the
manual or help info that comes with a viewer, given that it supports this linking
back to the source at all.

\stopsection

\startsection[title=Methods]

Contrary to the native \SYNCTEX\ we only deal with text which gives reasonable
efficient output. If you enable tracing (see next section) you can what has
become clickable. Instead of words you can also work with ranges, which not only
gives less runtime but also much smaller \type {.synctex} files. Just try:

\starttyping
\setupsynctex[state=start,method=min]
\stoptyping

to get words clickable and

\starttyping
\setupsynctex[state=start,method=max]
\stoptyping

to get the more efficient ranges. The overhead for \type {min} is some 10 percent
while \type {max} slows down around 5 percent.

\stopsection

\startsection[title=Tracing]

In case you want to see what gets synced you can enable a tracker:

\starttyping
\enabletrackers[system.synctex.visualize]
\enabletrackers[system.synctex.visualize=real]
\stoptyping

The following tracker outputs some status information about \XML\ flushing. Such
trackers only make sense for developers.

\starttyping
\enabletrackers[system.synctex.xml]
\stoptyping

% At the cost of some extra overhead, the next (experimental) directive can be used
% when the accuracy is not optimal.
%
% \starttyping
% \enabledirectives[system.synctex.details]
% \stoptyping

\stopsection

\startsection[title=Warning]

Don't turn on this feature when you don't need it. This is one of those mechanism
that hits performance badly.

Depending on needs the functionality can be improved and|/|or extended. Of course
you can always use the traditional \SYNCTEX\ method but don't expect it to behave
as described here.

\stopsection

\stopchapter

\stopcomponent
