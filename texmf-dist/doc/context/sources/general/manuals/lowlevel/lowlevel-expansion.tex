% language=us runpath=texruns:manuals/lowlevel

% This is work in progress and after an initial draft got extended because of the
% 2021 meeting. It will hopefully improve over time.

\usemodule[system-tokens]

\environment lowlevel-style

\startdocument
  [title=expansion,
   color=middleyellow]

\startsectionlevel[title=Preamble]

This short manual demonstrates a couple of properties of the macro language. It
is not an in|-|depth philosophical expose about macro languages, tokens,
expansion and such that some \TEX ies like. I prefer to stick to the practical
aspects. Occasionally it will be technical but you can just skip those paragraphs
(or later return to them) when you can't follow the explanation. It's often not
that relevant. I won't talk in terms of mouth, stomach and gut the way the \TEX
book does and although there is no way to avoid the word \quote {token} I will do
my best to not complicate matters by too much token speak. Examples show best
what we mean.

\stopsectionlevel

\startsectionlevel[title={\TEX\ primitives}]

The \TEX\ language provides quite some commands and those built in are called
primitives. User defined commands are called macros. A macro is a shortcut to a
list of primitives and|/|or macro calls. All can be mixed with characters that
are to be typeset somehow.

\starttyping[option=TEX]
\def\MyMacro{b}

a\MyMacro c
\stoptyping

When \TEX\ reads this input the \type {a} gets turned into a glyph node with a
reference to the current font set and the character \type {a}. Then the parser
sees a macro call, and it will enter another input level where it expands this
macro. In this case it sees just an \type {b} and it will give this the same
treatment as the \type {a}. The macro ends, the input level decrements and the
\type {c} gets its treatment.

Before we move on to more examples and differences between engines, it is good to
stress that \type {\MyMacro} is not a primitive command: we made our command
here. The \type {b} actually can be seen as a sort of primitive because in this
macro it gets stored as so called token with a primitive property. That primitive
property can later on be used to determine what to do. More explicit examples of
primitives are \type {\hbox}, \type {\advance} and \type {\relax}. It will be
clear that \CONTEXT\ extends the repertoire of primitive commands with a lot of
macro commands. When we typeset a source using module \type {m-scite} the
primitives come out dark blue.

The amount of primitives differs per engine. It all starts with \TEX\ as written
by Don Knuth. Later \ETEX\ added some more primitives and these became official
extensions adopted by other variants of \TEX. The \PDFTEX\ engine added quite
some and as follow up on that \LUATEX\ added more but didn't add all of \PDFTEX.
A few new primitives came from \OMEGA\ (\ALEPH). The \LUAMETATEX\ engine drops a
set of primitives that comes with \LUATEX\ and adds plenty new ones. The nature
of this engine (no backend and less frontend) makes that we need to implement
some primitives as macros. But the basic set is what good old \TEX\ comes with.

Internally these so called primitives are grouped in categories that relate to
their nature. They can be directly expanded (a way of saying that they get
immediately interpreted) or delayed (maybe stored for later usage). They can
involve definitions, calculations, setting properties and values or they can
result in some typesetting. This is what makes \TEX\ confusing to new users: it
is a macro programming language, an interpreter but at the same time an executor
of typesetting instructions.

A group of primitives is internally identified as a command (they have a \type
{cmd} code) and the sub commands are flagged by their \type {chr} code. This
sounds confusing but just thing of the fact that most of what we input are
characters and therefore they make up most sub commands. For instance the \quote
{letter \type {cmd}} is used for characters that are seen as letters that can be
used in the name of user commands, can be typeset, are valid for hyphenation
etc.\ The letter related \type {cmd} can have many \type {chr} codes (all of
\UNICODE). I'd like to remark that the grouping is to a large extend functional,
so sometimes primitives that you expect to be similar in nature are in different
groups. This has to do with the fact that \TEX\ needs to be a able to determine
efficiently if a primitive is operating (or forbidden) in horizontal, vertical
and|/|or math mode.

There are more than 150 internal \type {cmd} groups. if we forget about the
mentioned character related ones, some, have only a few sub commands (\type
{chr}) and others many more (just consider all the \OPENTYPE\ math spacing
related parameters). A handful of these commands deal with what we call macros:
user defined combinations of primitives and other macros, consider them little
programs. The \type {\MyMacro} example above is an example. There are differences
between engines. In standard \TEX\ there are \type {\outer} and \type {\long}
commands, and most engines have these. However, in \LUAMETATEX\ the later to be
discussed \type {\protected} macros have their own specific \quote {call \type
{cmd}}. Users don't need to bother about this.

So, when from now on we talk about primitives, we mean the built in, hard coded
commands, and when we talk about macros we mean user commands. Although
internally there are less \type {cmd} categories than primitives, from the
perspective of the user they are all unique. Users won't consult the source
anyway but when they do they are warned. Also, when in \LUAMETATEX\ you use the
low level interfacing to \TEX\ you have to figure out these subtle aspects
because there this grouping does matter.

Before we continue I want to make clear that expansion (as discussed in this
document) can refer to a macro being expanded (read: its meaning gets injected
into the input, so the engine kind of sidetracks from what is was doing) but also
to direct consequences of running into a primitive. However, users only need to
consider expansion in the perspective of macros. If a user has \type {\advance}
in the input it immediately gets done. But when it's part of a macro definition
it only is executed when the macro expands. A good check in (traditional) \TEX\
is to compare what happens in \type {\def} and \type {\edef} which is why we will
use these two in the upcoming examples. You put something in a macro and then
check what \type {\meaning} or \type {\show} reports.

Now back to user defined macros. A macro can contain references to macros so in
practice the input can go several levels up and some applications push back a lot
so this is why your \TEX\ input stack can be configured to be huge.

\starttyping[option=TEX]
\def\MyMacroA{ and }
\def\MyMacroB{1\MyMacroA 2}

a\MyMacroA b
\stoptyping

When \type {\MyMacroB} is defined, its body gets three so called tokens: the
character token \type {1} with property \quote {other}, a token that is a
reference to the macro \type {\MyMacroB}, and a character token \type {2}, also
with property \quote {other} The meaning of \type {\MyMacroA} is five tokens:
a reference to a space token, then three character tokens with property \quote
{letter}, and finally a space token.

\starttyping[option=TEX]
\def \MyMacroA{ and }
\edef\MyMacroB{1\MyMacroA 2}

a\MyMacroA b
\stoptyping

In the second definition an \type {\edef} is used, where the \type {e} indicates
expansion. This time the meaning gets expanded immediately. So we get effectively the same
as in:

\starttyping[option=TEX]
\def\MyMacroB{1 and 2}
\stoptyping

Characters are easy: they just expand to themselves or trigger adding a glyph
node, but not all primitives expand to their meaning or effect.

\startbuffer
\def\MyMacroA{\scratchcounter = 1 }
\def\MyMacroB{\advance\scratchcounter by 1}
\def\MyMacroC{\the\scratchcounter}

\MyMacroA a
\MyMacroB b
\MyMacroB c
\MyMacroB d
\MyMacroC
\stopbuffer

\typebuffer[option=TEX]

\scratchcounter0 \getbuffer

\startlines \tt
\meaning\MyMacroA
\meaning\MyMacroB
\meaning\MyMacroC
\stoplines

Let's assume that \type {\scratchcounter} is zero to start with and use \type
{\edef's}:

\startbuffer
\edef\MyMacroA{\scratchcounter = 1 }
\edef\MyMacroB{\advance\scratchcounter by 1}
\edef\MyMacroC{\the\scratchcounter}

\MyMacroA a
\MyMacroB b
\MyMacroB c
\MyMacroB d
\MyMacroC
\stopbuffer

\typebuffer[option=TEX]

\scratchcounter0 \getbuffer

\startlines \tt
\meaning\MyMacroA
\meaning\MyMacroB
\meaning\MyMacroC
\stoplines

So, this time the third macro has its meaning frozen, but we can
prevent this by applying a \type {\noexpand} when we do this:

\startbuffer
\edef\MyMacroA{\scratchcounter = 1 }
\edef\MyMacroB{\advance\scratchcounter by 1}
\edef\MyMacroC{\noexpand\the\scratchcounter}

\MyMacroA a
\MyMacroB b
\MyMacroB c
\MyMacroB d
\MyMacroC
\stopbuffer

\typebuffer[option=TEX]

\scratchcounter0 \getbuffer

\startlines \tt
\meaning\MyMacroA
\meaning\MyMacroB
\meaning\MyMacroC
\stoplines

Of course this is a rather useless example but it serves its purpose: you'd better
be aware what gets expanded immediately in an \type {\edef}. In most cases you
only need to worry about \type {\the} and embedded macros (and then of course
their meanings).

\def\MyShow{\quotation {\strut \inlinebuffer \expandafter \typ \expandafter
{\the\scratchtoks}\strut}}

You can also store tokens in a so-called token register. Here we use a predefined
scratch register:

\startbuffer
\def\MyMacroA{ and }
\def\MyMacroB{1\MyMacroA 2}
\scratchtoks {\MyMacroA}
\stopbuffer

\typebuffer[option=TEX]

The content of \type {\scratchtoks} is: \MyShow, so no expansion has happened
here.

\startbuffer
\def\MyMacroA{ and }
\def\MyMacroB{1\MyMacroA 2}
\scratchtoks \expandafter {\MyMacroA}
\stopbuffer

\typebuffer[option=TEX]

Now the content of \type {\scratchtoks} is: \MyShow, so this time expansion has
happened.

\startbuffer
\def\MyMacroA{ and }
\def\MyMacroB{1\MyMacroA 2}
\scratchtoks \expandafter {\MyMacroB}
\stopbuffer

\typebuffer[option=TEX]

Indeed the macro gets expanded but only one level: \MyShow. Compare this with:

\startbuffer
\def\MyMacroA{ and }
\edef\MyMacroB{1\MyMacroA 2}
\scratchtoks \expandafter {\MyMacroB}
\stopbuffer

\typebuffer[option=TEX]

The trick is to expand in two steps with an intermediate \type {\edef}: \MyShow. Later we will see that other
engines provide some more expansion tricks. The only way to get some grip on
expansion is to just play with it.

The \type {\expandafter} primitive expands the token (which can be a macro) standing after
the next next one and then injects its meaning into the stream. So:

\starttyping[option=TEX]
\expandafter \MyMacroA \MyMacroB
\stoptyping

works okay. In a normal document you will never need this kind of hackery: it
only happens in a bit more complex macros. Here is an example:

\startbuffer[a]
\scratchcounter 1
\bgroup
\advance\scratchcounter 1
\egroup
\the\scratchcounter
\stopbuffer

\typebuffer[a][option=TEX]

\startbuffer[b]
\scratchcounter 1
\bgroup
\advance\scratchcounter 1
\expandafter
\egroup
\the\scratchcounter
\stopbuffer

\typebuffer[b][option=TEX]

The first one gives \inlinebuffer[a], while the second gives \inlinebuffer[b].

% \let
% \futurelet
% \afterassignment
% \aftergroup

\stopsectionlevel

\startsectionlevel[title={\ETEX\ primitives}]

In this engine a couple of extensions were added and later on \PDFTEX\ added some
more. We only discuss a few that relate to expansion. There is however a pitfall
here. Before \ETEX\ showed up, \CONTEXT\ already had a few mechanism that also
related to expansion and it used some names for macros that clash with those in
\ETEX. This is why we will use the \type {\normal} prefix here to indicate the
primitive. \footnote {In the meantime we no longer have a low level \type
{\protected} macro so one can use the primitive}.

\startbuffer
\def\MyMacroA{a}
\def\MyMacroB{b}
\normalprotected\def\MyMacroC{c}
\edef\MyMacroABC{\MyMacroA\MyMacroB\MyMacroC}
\stopbuffer

\typebuffer[option=TEX] \getbuffer

These macros have the following meanings:

\startlines \tt
\meaning\MyMacroA
\meaning\MyMacroB
\meaning\MyMacroC
\meaning\MyMacroABC
\stoplines

In \CONTEXT\ you will use the \type {\unexpanded} prefix instead, because that one
did something similar in older versions of \CONTEXT. As we were early adopters of
\ETEX, this later became a synonym to the \ETEX\ primitive.

\startbuffer
\def\MyMacroA{a}
\def\MyMacroB{b}
\normalprotected\def\MyMacroC{c}
\normalexpanded{\scratchtoks{\MyMacroA\MyMacroB\MyMacroC}}
\stopbuffer

\typebuffer[option=TEX] \getbuffer

Here the wrapper around the token register assignment will expand the three
macros, unless they are protected, so its content becomes \MyShow. This saves
either a lot of more complex \type {\expandafter} usage or the need to use an intermediate
\type {\edef}. In \CONTEXT\ the \type {\expanded} macro does something simpler
but it doesn't expand the first token as this is meant as a wrapper around a command,
like:

\starttyping[option=TEX]
\expanded{\chapter{....}} % a ConTeXt command
\stoptyping

where we do want to expand the title but not the \type {\chapter} command (not
that this would happen actually because \type {\chapter} is a protected command.)

The counterpart of \type {\normalexpanded} is \type {\normalunexpanded}, as in:

\startbuffer
\def\MyMacroA{a}
\def\MyMacroB{b}
\normalprotected\def\MyMacroC{c}
\normalexpanded {\scratchtoks
    {\MyMacroA\normalunexpanded {\MyMacroB}\MyMacroC}}
\stopbuffer

\typebuffer[option=TEX] \getbuffer

The register now holds \MyShow: three tokens, one character token and two
macro references.

Tokens can represent characters, primitives, macros or be special entities like
starting math mode, beginning a group, assigning a dimension to a register, etc.
Although you can never really get back to the original input, you can come pretty
close, with:

\startbuffer
\detokenize{this can $ be anything \bgroup}
\stopbuffer

\typebuffer[option=TEX]

This (when typeset monospaced) is: {\tt \inlinebuffer}. The detokenizer is like
\type {\string} applied to each token in its argument. Compare this to:

\startbuffer
\normalexpanded {
    \normaldetokenize{10pt}
}
\stopbuffer

\typebuffer[option=TEX]

We get four tokens: {\tt\inlinebuffer}.

\startbuffer
\normalexpanded {
    \string 1\string 0\string p\string t
}
\stopbuffer

\typebuffer[option=TEX]

So that was the same operation: {\tt\inlinebuffer}, but in both cases there is a
subtle thing going on: characters have a catcode which distinguishes them. The
parser needs to know what makes up a command name and normally that's only
letters. The next snippet shows these catcodes:

\startbuffer
\normalexpanded {
    \noexpand\the\catcode`\string 1 \noexpand\enspace
    \noexpand\the\catcode`\string 0 \noexpand\enspace
    \noexpand\the\catcode`\string p \noexpand\enspace
    \noexpand\the\catcode`\string t \noexpand
}
\stopbuffer

\typebuffer[option=TEX]

The result is \quotation {\tt\inlinebuffer}: two characters are marked as \quote
{letter} and two fall in the \quote {other} category.

\stopsectionlevel

\startsectionlevel[title={\LUATEX\ primitives}]

This engine adds a little to the expansion repertoire. First of all it offers a
way to extend token lists registers:

\startbuffer
\def\MyMacroA{a}
\def\MyMacroB{b}
\normalprotected\def\MyMacroC{b}
\scratchtoks{\MyMacroA\MyMacroB}
\stopbuffer

\typebuffer[option=TEX] \getbuffer

The result is: \MyShow.

\startbuffer
\toksapp\scratchtoks{\MyMacroA\MyMacroB}
\stopbuffer

\typebuffer[option=TEX] \getbuffer

We're now at: \MyShow.

\startbuffer
\etoksapp\scratchtoks{\MyMacroA\space\MyMacroB\space\MyMacroC}
\stopbuffer

\typebuffer[option=TEX] \getbuffer

The register has this content: \MyShow, so the additional context got expanded in
the process, except of course the protected macro \type {\MyMacroC}.

There is a bunch of these combiners: \type {\toksapp} and \type {\tokspre} for
local appending and prepending, with global companions: \type {\gtoksapp} and
\type {\gtokspre}, as well as expanding variant: \type {\etoksapp}, \type
{\etokspre}, \type {\xtoksapp} and \type {\xtokspre}.

These are not beforehand more efficient that using intermediate expanded macros
or token lists, simply because in the process \TEX\ has to create tokens lists
too, but sometimes they're just more convenient to use. In \CONTEXT\ we actually
do benefit from these.

\stopsectionlevel

\startsectionlevel[title={\LUAMETATEX\ primitives}]

We already saw that macro's can be defined protected which means that

\startbuffer
           \def\TestA{A}
\protected \def\TestB{B}
          \edef\TestC{\TestA\TestB}
\stopbuffer

\typebuffer[option=TEX] \getbuffer

gives this:

\startlines
\type{\TestC} : {\tttf \meaningless\TestC}
\stoplines

One way to get \type {\TestB} expanded it to prefix it with \type {\expand}:

\startbuffer
           \def\TestA{A}
\protected \def\TestB{B}
          \edef\TestC{\TestA\TestB}
          \edef\TestD{\TestA\expand\TestB}
\stopbuffer

\typebuffer[option=TEX] \getbuffer

We now get:

\startlines
\type{\TestC} : {\tttf \meaningless\TestC}
\type{\TestD} : {\tttf \meaningless\TestD}
\stoplines

There are however cases where one wishes this to happen automatically, but that
will also make protected macros expand which will create havoc, like switching fonts.

\startbuffer
               \def\TestA{A}
\protected     \def\TestB{B}
\semiprotected \def\TestC{C}
              \edef\TestD{\TestA\TestB\TestC}
              \edef\TestE{\normalexpanded{\TestA\TestB\TestC}}
              \edef\TestF{\semiexpanded  {\TestA\TestB\TestC}}
\stopbuffer

\typebuffer[option=TEX] \getbuffer

This time \type {\TestC} looses its protection:

\startlines
\type{\TestA} : {\tttf \meaningless\TestA}
\type{\TestB} : {\tttf \meaningless\TestB}
\type{\TestC} : {\tttf \meaningless\TestC}
\type{\TestD} : {\tttf \meaningless\TestD}
\type{\TestE} : {\tttf \meaningless\TestE}
\type{\TestF} : {\tttf \meaningless\TestF}
\stoplines

Actually adding \type {\fullyexpanded} would be trivial but it makes not much
sense to add the overhead (at least not now). This feature is experimental
anyway so it might go away when I see no real advantage from it.

When you store something in a macro or token register you always need to keep an
eye on category codes. A dollar in the input is normally treated as math shift, a
hash indicates a macro parameter or preamble entry. Characters like \quote {A}
are letters but \quote {[} and \quote {]} are tagged as \quote {other}. The \TEX\
scanner acts according to these codes. If you ever find yourself in a situation
that changing catcodes is no option or cumbersome, you can do this:

\starttyping[option=TEX]
\edef\TestOA{\expandtoken\othercatcode `A}
\edef\TestLA{\expandtoken\lettercatcode`A}
\stoptyping

In both cases the meaning is \type {A} but in the first case it's not a letter
but a character flagged as \quote {other}.

A whole new category of commands has to do with so called local control. When
\TEX\ scans and interprets the input, a process takes place that is called
tokenizing: (sequences of) characters get a symbolic representation and travel
through the system as tokens. Often they immediately get interpreted and are then
discarded. But when for instance you define a macro they end up as a linked list
of tokens in the macro body. We already saw that expansion plays a role. In most
cases, unless \TEX\ is collecting tokens, the main action is dealt with in the so-called
main loop. Something gets picked up from the input but can also be pushed
back, for instance because of some lookahead that didn't result in an action.
Quite some time is spent in pushing and popping from the so-called input stack.

When we are in \LUA, we can pipe back into the engine but all is collected till
we're back in \TEX\ where the collected result is pushed into the input. Because
\TEX\ is a mix of programming and action there basically is only that main loop.
There is no real way to start a sub run in \LUA\ and do all kind of things
independent of the current one. This makes sense when you consider the mix: it
would get too confusing.

However, in \LUATEX\ and even better in \LUAMETATEX, we can enter a sort of local
state and this is called \quote {local control}. When we are in local control a
new main loop is entered and the current state is temporarily forgotten: we can for
instance expand where one level up expansion was not done. It sounds complicated
an indeed it is complicated so examples have to clarify it.

\starttyping[option=TEX]
1 \setbox0\hbox to 10pt{2} \count0=3 \the\count0 \multiply\count0 by 4
\stoptyping

This snippet of code is not that useful but illustrates what we're dealing with:

\startitemize

\startitem
    The \type {1} gets typeset. So, characters like that are seen as text.
\stopitem

\startitem
    The \type {\setbox} primitive triggers picking up a register number, then
    goes on scanning for a box specification and that itself will typeset a
    sequence of whatever until the group ends.
\stopitem

\startitem
    The \type {count} primitive triggers scanning for a register number (or
    reference) and then scans for a number; the equal sign is optional.
\stopitem

\startitem
    The \type {the} primitive injects some value into the current input stream
    and it does so by entering a new input level.
\stopitem

\startitem
    The \type {multiply} primitive picks up a register specification and
    multiplies that by the next scanned number. The \type {by} is optional.
\stopitem

\stopitemize

We now look at this snippet again but with an expansion context:

\startbuffer[def]
\def \TestA{1 \setbox0\hbox{2} \count0=3 \the\count0}
\stopbuffer

\startbuffer[edef]
\edef\TestB{1 \setbox0\hbox{2} \count0=3 \the\count0}
\stopbuffer

\typebuffer[def] [option=TEX]
\typebuffer[edef][option=TEX]

\getbuffer[def]
\getbuffer[edef]

These two macros have a slightly different body. Make sure you see the
difference before reading on.

\luatokentable\TestA

\luatokentable\TestB

We now introduce a new primitive \type {\localcontrolled}:

\startbuffer[edef]
\edef\TestB{1 \setbox0\hbox{2} \count0=3 \the\count0}
\stopbuffer

\startbuffer[ldef]
\edef\TestC{1 \setbox0\hbox{2} \localcontrolled{\count0=3} \the\count0}
\stopbuffer

\typebuffer[edef][option=TEX]
\typebuffer[ldef][option=TEX]

\getbuffer[edef]
\getbuffer[ldef]

Again, watch the subtle differences:

\luatokentable\TestB

\luatokentable\TestC

Another example:

\startbuffer[edef]
\edef\TestB{1 \setbox0\hbox{2} \count0=3 \the\count0}
\stopbuffer

\startbuffer[ldef]
\edef\TestD{\localcontrolled{1 \setbox0\hbox{2} \count0=3 \the\count0}}
\stopbuffer

\typebuffer[edef][option=TEX]
\typebuffer[ldef][option=TEX]

\getbuffer[edef]\getbuffer[ldef]\quad{\darkgray\leftarrow\space Watch how the results end up here!}

\luatokentable\TestB

\luatokentable\TestD

We can use this mechanism to define so called fully expandable macros:

\startbuffer[def]
\def\WidthOf#1%
  {\beginlocalcontrol
   \setbox0\hbox{#1}%
   \endlocalcontrol
   \wd0 }
\stopbuffer

\startbuffer[use]
\scratchdimen\WidthOf{The Rite Of Spring}

\the\scratchdimen
\stopbuffer

\typebuffer[def][option=TEX]
\typebuffer[use][option=TEX]

\getbuffer[def]\getbuffer[use]

When you want to add some grouping, it quickly can become less pretty:

\startbuffer[def]
\def\WidthOf#1%
  {\dimexpr
      \beginlocalcontrol
        \begingroup
          \setbox0\hbox{#1}%
          \expandafter
        \endgroup
      \expandafter
      \endlocalcontrol
      \the\wd0
   \relax}
\stopbuffer

\startbuffer[use]
\scratchdimen\WidthOf{The Rite Of Spring}

\the\scratchdimen
\stopbuffer

\typebuffer[def][option=TEX]
\typebuffer[use][option=TEX]

\getbuffer[def]\getbuffer[use]

A single token alternative is available too and its usage is like this:

\startbuffer
 \def\TestA{\scratchcounter=100 }
\edef\TestB{\localcontrol\TestA \the\scratchcounter}
\edef\TestC{\localcontrolled{\TestA} \the\scratchcounter}
\stopbuffer

\typebuffer[option=TEX] \getbuffer

The content of \type {\TestB} is \quote {\tttf\meaningless\TestB} and of course
the \type {\TestC} macro gives \quote {\tttf\meaningless\TestC}.

We now move to the \LUA\ end. Right from the start the way to get something into
\TEX\ from \LUA\ has been the print functions. But we can also go local
(immediate). There are several methods:

\startitemize[packed]
\startitem via a set token register \stopitem
\startitem via a defined macro \stopitem
\startitem via a string \stopitem
\stopitemize

Among the things to keep in mind are catcodes, scope and expansion (especially in
when the result itself ends up in macros). We start with an example where we go via
a token register:

\startbuffer[set]
\toks0={\setbox0\hbox{The Rite Of Spring}}
\toks2={\setbox0\hbox{The Rite Of Spring!}}
\stopbuffer

\typebuffer[set][option=TEX]

\startbuffer[run]
\startluacode
tex.runlocal(0) context("[1: %p]",tex.box[0].width)
tex.runlocal(2) context("[2: %p]",tex.box[0].width)
\stopluacode
\stopbuffer

\typebuffer[run][option=TEX]

\start \getbuffer[set,run] \stop

We can also use a macro:

\startbuffer[set]
\def\TestA{\setbox0\hbox{The Rite Of Spring}}
\def\TestB{\setbox0\hbox{The Rite Of Spring!}}
\stopbuffer

\typebuffer[set][option=TEX]

\startbuffer[run]
\startluacode
tex.runlocal("TestA") context("[3: %p]",tex.box[0].width)
tex.runlocal("TestB") context("[4: %p]",tex.box[0].width)
\stopluacode
\stopbuffer

\typebuffer[run][option=TEX]

\start \getbuffer[set,run] \stop

A third variant is more direct and uses a (\LUA) string:

\startbuffer[run]
\startluacode
tex.runstring([[\setbox0\hbox{The Rite Of Spring}]])

context("[5: %p]",tex.box[0].width)

tex.runstring([[\setbox0\hbox{The Rite Of Spring!}]])

context("[6: %p]",tex.box[0].width)
\stopluacode
\stopbuffer

\typebuffer[run][option=TEX]

\start \getbuffer[run] \stop

A bit more high level:

\starttyping[option=LUA]
context.runstring([[\setbox0\hbox{(Here \bf 1.2345)}]])
context.runstring([[\setbox0\hbox{(Here \bf   %.3f)}]],1.2345)
\stoptyping

Before we had \type {runstring} this was the way to do it when staying in \LUA\
was needed:

\startbuffer[run]
\startluacode
token.setmacro("TestX",[[\setbox0\hbox{The Rite Of Spring}]])
tex.runlocal("TestX")
context("[7: %p]",tex.box[0].width)
\stopluacode
\stopbuffer

\typebuffer[run][option=TEX]

\start \getbuffer[run] \stop

\startbuffer[run]
\startluacode
tex.scantoks(0,tex.ctxcatcodes,[[\setbox0\hbox{The Rite Of Spring!}]])
tex.runlocal(0)
context("[8: %p]",tex.box[0].width)
\stopluacode
\stopbuffer

\typebuffer[run][option=TEX]

\start \getbuffer[run] \stop

The order of flushing matters because as soon as something is not stored in a
token list or macro body, \TEX\ will typeset it. And as said, a lot of this relates
to pushing stuff into the input which is stacked. Compare:

\startbuffer[run]
\startluacode
context("[HERE 1]")
context("[HERE 2]")
\stopluacode
\stopbuffer

\typebuffer[run][option=TEX]

\start \getbuffer[run] \stop

with this:

\startbuffer[run]
\startluacode
tex.pushlocal() context("[HERE 1]") tex.poplocal()
tex.pushlocal() context("[HERE 2]") tex.poplocal()
\stopluacode
\stopbuffer

\typebuffer[run][option=TEX]

\start \getbuffer[run] \stop

You can expand a macro at the \LUA\ end with \type {token.expandmacro} which has
a peculiar interface. The first argument has to be a string (the name of a macro)
or a userdata (a valid macro token). This macro can be fed with parameters by
passing more arguments:

\starttabulate[|||]
\NC string \NC serialized to tokens \NC \NR
\NC true   \NC wrap the next string in curly braces \NC \NR
\NC table  \NC each entry will become an argument wrapped in braces \NC \NR
\NC token  \NC inject the token directly \NC \NR
\NC number \NC change control to the given catcode table \NC \NR
\stoptabulate

There are more scanner related primitives, like the \ETEX\ primitive
\type {\detokenize}:

\startbuffer[run]
[\detokenize {test \relax}]
\stopbuffer

\typebuffer[run][option=TEX]

This gives: {\tttf \getbuffer[run]}. In \LUAMETATEX\ we also have complementary
primitive(s):

\startbuffer[run]
[\tokenized   catcodetable \vrbcatcodes {test {\bf test} test}]
[\tokenized                             {test {\bf test} test}]
[\retokenized              \vrbcatcodes {test {\bf test} test}]
\stopbuffer

\typebuffer[run][option=TEX]

The \type {\tokenized} takes an optional keyword and the examples above give: {\tttf
\getbuffer[run]}. The \LUATEX\ primitive \type {\scantextokens} which is a
variant of \ETEX's \type {\scantokens} operates under the current catcode regime
(the last one honors \type {\everyeof}). The difference with \type {\tokenized}
is that this one first serializes the given token list (just like \type
{\detokenize}). \footnote {The \type {\scan*tokens} primitives now share the same
helpers as \LUA, but they should behave the same as in \LUATEX.}

With \type {\retokenized} the catcode table index is mandatory (it saves a bit of
scanning and is easier on intermixed \type {\expandafter} usage. There
often are several ways to accomplish the same:

\startbuffer[run]
\def\MyTitle{test {\bf test} test}
\detokenize               \expandafter{\MyTitle}: 0.46\crlf
\meaningless                           \MyTitle : 0.47\crlf
\retokenized              \notcatcodes{\MyTitle}: 0.87\crlf
\tokenized   catcodetable \notcatcodes{\MyTitle}: 0.93\crlf
\stopbuffer

\typebuffer[run][option=TEX]

\getbuffer[run]

Here the numbers show the relative performance of these methods. The \type
{\detokenize} and \type {\meaningless} win because they already know that a
verbose serialization is needed. The last two first serialize and then
reinterpret the resulting token list using the given catcode regime. The last one
is slowest because it has to scan the keyword.

There is however a pitfall here:

\startbuffer[run]
\def\MyText {test}
\def\MyTitle{test \MyText\space test}
\detokenize               \expandafter{\MyTitle}\crlf
\meaningless                           \MyTitle \crlf
\retokenized              \notcatcodes{\MyTitle}\crlf
\tokenized   catcodetable \notcatcodes{\MyTitle}\crlf
\stopbuffer

\typebuffer[run][option=TEX]

The outcome is different now because we have an expandable embedded macro call.
The fact that we expand in the last two primitives is also the reason why they are
\quote {slower}.

\getbuffer[run]

To complete this picture, we show a variant than combines much of what has been
introduced in this section:

\startbuffer[run]
\semiprotected\def\MyTextA {test}
\def\MyTextB {test}
\def\MyTitle{test \MyTextA\space \MyTextB\space test}
\detokenize               \expandafter{\MyTitle}\crlf
\meaningless                           \MyTitle \crlf
\retokenized              \notcatcodes{\MyTitle}\crlf
\retokenized              \notcatcodes{\semiexpanded{\MyTitle}}\crlf
\tokenized   catcodetable \notcatcodes{\MyTitle}\crlf
\tokenized   catcodetable \notcatcodes{\semiexpanded{\MyTitle}}
\stopbuffer

\typebuffer[run][option=TEX]

This time compare the last four lines:

\getbuffer[run]

Of course the question remains to what extend we need this and eventually will
apply in \CONTEXT. The \type {\detokenize} is used already. History shows that
eventually there is a use for everything and given the way \LUAMETATEX\ is
structured it was not that hard to provide the alternatives without sacrificing
performance or bloating the source.

% tex.quitlocal
%
% tex.expandmacro   : string|userdata + [string|true|table|userdata|number]*
% tex.expandasvalue : kind + string|userdata + [string|true|table|userdata|number]*
% tex.runstring     : [catcode] + string + expand + grouped
% tex.runlocal      : function|number(register)|string(macro)|userdata(token) + expand + grouped
% mplib.expandtex   : mpx + kind + string|userdata + [string|true|table|userdata|number]*

\stopsectionlevel

\startsectionlevel[title=Dirty tricks]

When I was updating this manual Hans vd Meer and I had some discussions about
expansion and tokenization related issues when combining of \XML\ processing with
\TEX\ macros where he did some manipulations in \LUA. In these mixed cases you
can run into catcode related problems because in \XML\ you want for instance a
\type {#} to be a hash mark (other character) and not an parameter identifier.
Normally this is handled well in \CONTEXT\ but of course there are complex cases
where you need to adapt.

Say that you want to compare two strings (officially we should say token lists)
with mixed catcodes. Let's also assume that you want to use the normal \type
{\if} construct (which was part of the discussion). We start with defining
a test set. The reason that we present this example here is that we use
commands discussed in previous sections:

\startbuffer[run]
               \def\abc{abc}
\semiprotected \def\xyz{xyz}
              \edef\pqr{\expandtoken\notcatcodes`p%
                        \expandtoken\notcatcodes`q%
                        \expandtoken\notcatcodes`r}

1: \ifcondition\similartokens{abc} {def}YES\else NOP\fi (NOP) \quad
2: \ifcondition\similartokens{abc}{\abc}YES\else NOP\fi (YES)

3: \ifcondition\similartokens{xyz} {pqr}YES\else NOP\fi (NOP) \quad
4: \ifcondition\similartokens{xyz}{\xyz}YES\else NOP\fi (YES)

5: \ifcondition\similartokens{pqr} {pqr}YES\else NOP\fi (YES) \quad
6: \ifcondition\similartokens{pqr}{\pqr}YES\else NOP\fi (YES)
\stopbuffer

\typebuffer[run][option=TEX]

So, we have a mix of expandable and semi expandable macros, and also a mix of
catcodes. A naive approach would be:

\startbuffer[def]
\permanent\protected\def\similartokens#1#2%
  {\iftok{#1}{#2}}
\stopbuffer

\typebuffer[def][option=TEX]

but that will fail on some cases:

\pushoverloadmode \startpacked \tttf \getbuffer[def,run]\stoppacked \popoverloadmode

So how about:

\startbuffer[def]
\permanent\protected\def\similartokens#1#2%
  {\iftok{\detokenize{#1}}{\detokenize{#2}}}
\stopbuffer

\typebuffer[def][option=TEX]

That one is even worse:

\pushoverloadmode \startpacked \tttf \getbuffer[def,run]\stoppacked \popoverloadmode

We need to expand so we end up with this:

\startbuffer[def]
\permanent\protected\def\similartokens#1#2%
  {\normalexpanded{\noexpand\iftok
     {\noexpand\detokenize{#1}}
     {\noexpand\detokenize{#2}}}}
\stopbuffer

\typebuffer[def][option=TEX]

Better:

\pushoverloadmode \startpacked \tttf \getbuffer[def,run]\stoppacked \popoverloadmode

But that will still not deal with the mildly protected macro so in the end we
have:

\startbuffer[def]
\permanent\protected\def\similartokens#1#2%
  {\semiexpanded{\noexpand\iftok
     {\noexpand\detokenize{#1}}
     {\noexpand\detokenize{#2}}}}
\stopbuffer

\typebuffer[def][option=TEX]

Now we're good:

\pushoverloadmode \startpacked \tttf \getbuffer[def,run]\stoppacked \popoverloadmode

Finally we wrap this one in the usual \type {\doifelse...} macro:

\startbuffer[def]
\permanent\protected\def\doifelsesimilartokens#1#2%
  {\ifcondition\similartokens{#1}{#2}%
     \expandafter\firstoftwoarguments
   \else
     \expandafter\secondoftwoarguments
   \fi}
\stopbuffer

\typebuffer[def][option=TEX]

so that we can do:

\starttyping[option=TEX]
\doifelsesimilartokens{pqr}{\pqr}{YES}{NOP}
\stoptyping

A companion macro of this is \type {\wipetoken} but for that one you need to look
into the source.

\stopsectionlevel

\stopdocument

% \aftergroups
% \aftergrouped
%
%     \starttyping
%           \def\foo{foo}
% \protected\def\oof{oof}
%
% \csname foo\endcsname
% \csname oof\endcsname
% \csname \foo\endcsname
% \begincsname \oof\endcsname % error in luametatex, but in texexpand l 477 we can block an error
%
% \ifcsname  foo\endcsname yes\else nop\fi
% \ifcsname  oof\endcsname yes\else nop\fi
% \ifcsname \foo\endcsname yes\else nop\fi
% \ifcsname \oof\endcsname yes\else nop\fi % nop in luametatex
% \stoptyping
