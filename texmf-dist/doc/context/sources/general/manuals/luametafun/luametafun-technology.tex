% language=us runpath=texruns:manuals/luametafun

\environment luametafun-style

\startcomponent luametafun-technology

\startchapter[title={Technology}]

The \METAPOST\ library that we use in \LUAMETATEX\ is a follow up on the library
used in \LUATEX\ which itself is a follow up on the original \METAPOST\ program
that again was a follow up on Don Knuths \METAFONT, the natural companion to
\TEX.

When we start with John Hobbies \METAPOST\ we see a graphical engine that
provides a simple but powerful programming language meant for making graphics,
not the freehand kind, but the more systematic ones. The output is \POSTSCRIPT\
but a simple kind that can easily be converted to \PDF. \footnote {For that
purpose I wrote a converter in the \TEX\ language for \PDFTEX, and even within
the limitations of \TEX\ at that time (fonts, number of registers, memory) it
worked out quite well.} It's output is very accurate and performance is great.

As part of the \LUATEX\ development project Taco Hoekwater turned \METAPOST\ into
\MPLIB, a downward compatible library where \METAPOST\ became a small program
using that library. But there is more: there are (when enabled) backends that
produce \PNG\ or \SVG, but when used these also add dependencies on moving
targets. The library by default uses the so called scaled numbers: floats that
internally are long integers. But it can also work in doubles, decimal and binary
and especially the last two create a dependency on libraries. It is good to
notice that as in the original \METAPOST\ the \POSTSCRIPT\ output handling is
visible all over the source. Also, the way \TYPEONE\ fonts are handled has been
extended, for instance by providing access to shapes.

At some point a \LUA\ interface got added that made it possible to call out to
the \LUA\ instance used in \LUATEX, so the three concepts: \TEX, \METAPOST\ and
\LUA\ can combine forces. A snippet of code can be run, and a result can be piped
back. Although there is some limited access to \METAPOST\ internals, the normal
way to go is by serializing \METAPOST\ data to the \LUA\ end and let \METAPOST\
scan the result using \type {scantokens}.

The library in \LUAMETATEX\ is a bit different. Of course it has the same core
graphic engine, but there is no longer a backend. In \CONTEXT\ \MKIV\ the
\POSTSCRIPT\ (and other) backends were not used anyway because it operates on the
exported \LUA\ representation of the result. Combined with the \type {prescript}
and \type {postscript} features introduced in the library that provides all we
need to make interesting extensions to the graphical engine (color, shading,
image inclusion, text, etc). The \METAPOST\ font support features are also not
used because we need support for \OPENTYPE\ and even in \MKII\ (for \PDFTEX\ and
\XETEX) we used a different approach to fonts.

It is for that reason that the library we use in \LUAMETATEX\ is a leaner version
of its ancestor. As mentioned, there is no backend code, only the \LUA\ export,
which saves a lot, and there are no traces of font support left, which also drops
many lines of code. We forget about the binary number model because it needs a
large library that also occasionally changes, but one can add it if needed. This
means that there are no dependencies except for decimal but that library is
relatively small and doesn't change at all. It also means that the resulting
\MPLIB\ library is much smaller, but it's still a substantial component in
\LUAMETATEX. Internally I use the future version number 3. The original
\METAPOST\ program is version 1, so the library got version 2, and that one
basically being frozen (it's in bug|-|fix mode) means that it will stick to that.

Another difference is that from the \LUA\ end one has access to several scanners
and also has possibilities to efficiently push back results to the engine.
Running scripts can also be done more efficient. This permits a rather efficient
(in terms of performance and memory usage) way to extend the language and add for
instance key|/|value based interfaces. There are some more additions, like for
instance pre- and postscripts to clip, boundary and group objects. Internals can
be numeric, string and boolean. One can use \UTF\ input although that has also be
added to the ancestor. Some redundant internal input|/|output remapping has been
removed and we are more tolerant to newlines in return values from \LUA. Error
messages have been normalized, internal documentation cleaned up a bit. A few
anomalies have been fixed too. All in- and output is now under \LUA\ control.
Etcetera. The (now very few) source files are still \CWEB\ files but the
conversion to \CCODE\ is done with a \LUA\ script that uses (surprise) the
\LUAMETATEX\ engine as \LUA\ processor. This give a bit nicer \CCODE\ output for
when we view it in e.g.\ Visual Studio too (normally the \CWEB\ output is not
meant to be seen by humans).

Keep in mind that it's still \METAPOST\ with all it provided, but some has to be
implemented in macros or in \LUA\ via callbacks. The simple fact that the
original library is the standard and is also the core of \METAPOST\ most of these
changes and additions cannot be backported to the original, but that is no big
deal. The advantage is that we can experiment with new features without
endangering users outside the \CONTEXT\ bubble. The same is true for the \LUA\
interface, which already is upgraded in many aspects.

\stopchapter

\stopcomponent
