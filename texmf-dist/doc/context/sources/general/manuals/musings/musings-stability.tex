% language=uk

\environment musings-style

\startcomponent musings-stability

\startchapter[title={Stability}]

\startsubject[title=Introduction]

How stable is \CONTEXT ? This question is hard to answer. For instance \MKII\
hasn't changed for years and seems to work quite well: no changes equals
stability. Those who use it can do with what it offers. The potentially sensitive
dependencies on for instance fonts are probably absent because there is not much
development in the 8 bit fonts arena. As long as these are available we're okay,
in fact, \OPENTYPE\ fonts are more a moving target and therefore less stable.

What do we mean by stable? The fundamental differences between an 8 bit engine
(and fonts) and an \UNICODE\ aware engine able to handle \OPENTYPE\ fonts is
substantial which is why we dropped some functionality and added some relevant
new. One can consider that a problem but in practice using fonts has become
easier so no one is hurt by it. Here we need to keep in mind that \PDFTEX\ is
really stable: it uses fonts and technology that doesn't change. On the other
hand \XETEX\ and \LUATEX\ follow new trends. Thereby \XETEX\ uses libraries,
which introduces a dependency and instability, while \LUATEX\ assumes solutions
in \LUA\ which means that users and macro writers can tweak and thereby also
introduce instability (but at least one can adapt that code).

Due to the way the user interface is set up, it is unlikely that \CONTEXT\ will
change. But the fact that we now have \LUA\ available means that many commands
have been touched. Most behave compatible, some have more functionality, and of
course we have a \LUA\ interface. We include a lot of support code which also
lessens dependencies.

The user input is normally \TEX\ but when you use \XML\ the move to \MKIV\ meant
that we dropped the \MKII\ way of dealing with it in favour of a completely new
mechanism. I get the impression that those using \XML\ don't regret that change.
Talking of stability the \MKIV\ \XML\ interface is typically a mechanism that is
stable and might change little. We can add new trickery but the old stays as it
is.

If we look at the output, there is \DVI\ and \PDF. In \MKII\ the \DVI\ could
become \POSTSCRIPT. As there are different \DVI\ post|-|processors the backend
code was using a plug|-|in model. Contrary to other macro packages there was only
one so called format that could adapt itself to the required (engine specific)
output. A \CONTEXT\ run has always been managed by a wrapper so users were not
bothered much by what \TEX\ engine they used and|/|or what backend was triggered.
This changed with \MKIV\ where we use just \LUATEX, always produce \PDF\ and
optionally can export \XML. But again the run is managed by a wrapper, which
incidentally is written in \LUA\ and thereby avoids dependencies on for instance
\PERL, \RUBY\ or \PYTHON, which are moving targets, use libraries and additional
user code, and thereby are potentially instable too.

The \PDF\ code that is produced is a mix of what the engine spits out and what
the macro package injects. The code is normally rather simple. This means that
it's no big deal to support the so called standards. It also means that we can
support advanced interactivity and other features but these also depends on the
viewers used. So, stability here is more fluent, for instance because the \PDF\
standard evolves and|/|or we need to adapt to viewers. Special demands like
tagged \PDF\ have been supported right from the start but how that evolves
depends mostly on input from users who need it. Again, that is less important
(and crucial) for stability than the rendering capabilities.

The fact that we use \LUA\ creates a dependency on that language but the reason
that we use it is {\em because} it is so stable. We follow the updates and so far
that worked out well. Now, say that we had a frozen version of \CONTEXT\ 2010 and
\LUATEX\ 1.09 that uses \LUA\ 5.3, would that work? First of all, in 2010
\LUATEX\ itself was evolving so the answer is probably \quotation {no}, unless
one adds a few compatibility patches. I'm not going to try it. The change from
5.1 to 5.2 to 5.3 was not really a problem I think and the few issues could be
dealt with easily. If you want long term stability and use a lot of \LUA\ code
you can take it into account when coding. Avoiding external libraries is a good
start.

Fonts are more than before moving targets. So, if you want stability there you
should save them with your document source. The processing of them has evolved
and has been improved over time. By now it's rather stable. More recent code can
catch more issues and fixes are relatively easy. But it's an area that you always
need to check when you update an old distribution. The same is true for language
related hyphenation patterns and script specific support. The community is no
longer leading in the math department either (\OPENTYPE\ math is a \MICROSOFT\
invention). But, the good news is that the \TEX\ ecosystem is always fast to
adapt and can also often provide more functionality.

Vertical spacing, in fact spacing in general is an aera that can always be
improved, so there is where you can expect changed. The same is true for side
floats or mechanisms where content is somehow attached to other moving content,
for instance marginal notes.

But code dealing with fonts, color, scripts, structure, and specific features
that once written don't need more, will not change that much. As mentioned for
fonts, like any resource, we also depend on third parties. Colors can relate to
standards, but their main properties are unchanged. Support for specific scripts
can (and will) be improved due to user input and demands so there the users also
influence stability. Structure doesn't really influence the overall rendering,
but the way you set it up does, but that's user styling. Of course during the
transition from \MKII\ to \MKIV\ and the evolution of \LUATEX\ things could be
broken, but fixing something structural seldom relates to rendering. If for
instance we improve the interpretation of \BIBTEX\ input , which can be real
messy, that involves data processing, nor rendering. When we improve support for
the \APA\ standard, which is complex, it might involve rendering but then that's
asked for and expected. One cannot do better than the input permits.

\stopsubject

\startsubject[title=Publishers]

When discussing stability and especially stability as requirement we need to look
at the way \CONTEXT\ is used. So let's look at a few scenarios. Say that a
publisher gets a camera ready book from an author in \PDF\ format. In that
case the author can do all tweaks needed. Now say that the publisher also wants
the source code in a format that makes reuse possible.

But let's face reality. Will that publisher really reformat the document in \PDF\
again? It's very unlikely. First of all the original \PDF\ can be kept, and
second, a reformat only makes sense after updating the content or going for a
completely different layout. It's basically a new book then. In that case literal
similarity of output is irrelevant. It is a cheap demand without much substance.

When the source is used for a different purpose the tool used to make the \PDF\
is irrelevant. In that case the coding of the source can matter. If it is in some
dialect of \TEX, fine, one has to convert it anyway (to suit the other usage). If
there is an \XML\ export available, fine too as it can be transformed, given that
the structure is rich enough, something that is unlikely to have been checked
when the original was archived. Then there could have been the demand for a
document in some other format and who can guarantee stability of the tools used
there? Just look at how \MICROSOFT\ Word evolved, or for that matter, its
competitors. On the average \TEX\ is more stable as one can snapshot a \TEX\ tree
and run binaries for years, if needed, in a virtual machine.

So, I don't think that a publisher is of any relevance in the discussion about
stability. Even if we can clearly define what a publisher is, I doubt if
publishers themselves can be considered long term stable organizations. Not
today. I'm not sure if (especially the large) publishers really deserve a place
in the discussion about stability but I'm willing to discuss that when I run into
one.

The main problem that an author can face when being confronted with the stability
issue this way is that the times are long gone that publishers have a clue about
what \TEX\ is, how it evolved and how it always had to and did adapt to changing
requirements. If you're lucky you will run into someone who does know all this.
They're normally a bit older and have seen the organization from any angles and
therefore are fun to work with.

But even then, rendering issues are often not high on their agenda. Outsourcing
often has become the modus operandi which basically brings us to the second group
involved in this discussion: suppliers.

\stopsubject

\startsubject[title=Suppliers]

I don't know many suppliers other than the ones we ran into over a few decades.
At least where I live the departments that are responsible for outsourcing
typesetting like to deal with only a few large suppliers, interestingly because
they assume that they are stable. However, in my experience hardly any of those
seem to have survived. (Of course one can wonder if long term commitment really
is that important in a world where companies change so fast.) This is somewhat
obscured by the fact that publishers themselves merge, reorganize, move people
around, etc. so who can check on the stability of suppliers. It is definitely a
fact that at least recently hardly any of them played a rol of any relevance in
the development of stable tools. In the past the membership of \TEX\ user groups
contained people working at publishers and suppliers but that has changed.

Let's focus on the suppliers that somehow use \TEX\ and let's consider two kind
of suppliers: small ones, one were only a few people work, and large ones. The
small ones depend on stable \TEX\ distributions, like \TEX Live where they can
get the resources from: styles, fonts, patterns, binaries. If they get the
authors \TEX\ files they need to have that access. They have to rework that input
into what the customer demands and that likely involves tweaks. So, maybe they
have developed their own additional code. For that code, stability is their own
responsibility. Did they tweak core code of a macro package? Fine, but you might
have it coming when you update. You cannot expect the evolving free meal world to
stick to your commercial needs. A supplier can play safe and somehow involve the
developers of macro packages or consult them occasionally, but does that really
happen often? Interesting is that a few times that I was asked for input it was
also wrapped in obscurity, as if some holy grail of styling was involved, while
it's quite likely that the developer of a macro package can write such a style
(or extra code) easily and probably also better. There really is not that much
unique code around.

Small suppliers can be on mailing lists where they can contribute, get feedback,
provide testing, etc. They are part of a process and as such have some influence
on stability. If they charge by the page, then a change in their tools can be
reflected in what they charge. Basically redoing a book (or so) after a decade is
doing a new job. And adapting to some new options in a package, as part of a
typesetting job is probably no big deal. Is commercial really more stable than
open source free software? Probably not, except from open source software
developers whose real objective is to eventually sell their stuff to some company
(and cash) and even accept it to be ditched. Small suppliers are more flexible.

The large suppliers are a different group. They often guard their secrets and
stay in the dark. They probably seldom share (fundamental) code and information.
If they are present in a community it can be for marketing reasons. If at some
point a large supplier would demand stability, then my first response would be:
sure I can make you a stable setup and maybe even provide intermediate patches
but put your money where your mouth is. But that never happened and I've come to
the conclusion that we can safely ignore that group. The \TEX\ user groups create
distributions and have for instance funded font development and it are the common
users who paid for that, not the scale ones. To some extent this is actually good
because large (software related) organizations often have special agendas that
can contradict what we aim at in the long term.

From the authors perspective there is a dilemma here. When you submit to a
publisher who outsources, it can be a demand to deliver in a specific \TEX\
format. Often a \PDF\ comes with the source then, so that the intended rendering
is known. Then that source goes to a supplier who then (quite likely) redoes a
lot of the coding in some stable subset, maybe even in a very old version of the
macro package. If I were such an author I'd render the document in \quote {as
stupid as possible mode} because you gain nothing by spending time on the looks.
So, stability within the package that you use is easy and translation from one to
another probably also. It's best to check beforehand what will happen with your
source and let stability, if mentioned, be their problem. After all they get paid
for it.

Suppliers seldom know \CONTEXT. An interesting question is if they really know
the alternatives well, apart from the bit they use. A well structured \CONTEXT\
source (or probably any source) is often easy to convert to another format. You
can assume that a supplier has tools for that (although we're often surprised
about the poor quality of tools used). Often the strict demand for some kind of
format is an excuse for lack of knowledge. Unfortunately you need a large author
base to change that attitude.

\stopsubject

\startsubject[title=Authors]

Before we move to some variants of the above, first I will look at stability from
the authors perspective. When a book is being written the typesetting more or
less happens as part of the process. The way it looks can influence the way you
write and vise versa. Once the book is done it can go in print and, unless you
were using beta versions of \CONTEXT\ and updated frequently. Normally you will
try to work in a stable setup. Of course when a user asks for additional features
while working on a project, he or she should also accept other beta features
and side effects.

After a few years an author might decide to update the book. The worst that can
happen is that the code doesn't run with the latest \CONTEXT. This is not so
likely because commands are upward compatible. However, the text might come out a
bit different, for instance because different fonts or patterns are used. But on
the average paragraphs will come out the same in \TEX. You can encounter
differences in the vertical spacing and page breaks, because that is where
improvements are still possible. If you use conceptually and implementation wise
complex mechanism like side floats, you can also run into compatibility issues.
But all these don't really matter much because the text will be updated anyway
and fine|-|tuning of page breaks (if at all) happens at the end. The more you try
to compete with desk top publishing, and the more tweaks you apply, the greater
is the risk that you introduce instability. It is okay for a one|-|time job, but
when you come back to it after a decade, be prepared for surprises.

Even if you stick to the original coding, it makes sense to sacrifice some of that
stability if new mechanisms have become available. For instance, if you use
\METAPOST, better ways to solve your problem might have become available. Or if
you document is 15 years old, a move from \MKII\ to \MKIV\ is a valid option,
in which case you might also consider using the latest fonts.

Of course, when you made a style where you patched core code, you can expect
problems, because anything not explicitly mentioned in the interface definition
files is subjected to change. But you probably see that coming anyway.

So, is an author (or stand alone user) really dependent on stability? Probably
less than thought. In fact, the operating system, internet and browsers,
additional tools: all change over time and one adapts. It's something one can
live with. Just see how people adapt to phones, tablets, social media, electric
cars, etc. As long as the document processes and reasonable output is generated
it's fine. And that is always what we aim at! After all we need to be able to use
it ourselves, don't we?

\stopsubject

\startsubject[title=Projects]

Although it is often overlooked as valid alternative in rendering in large scale
projects, \CONTEXT\ is perfect as component in a larger whole. Something goes
in, something comes out. In a long term project one can just install a minimal
distribution, write styles, and run it for ages. Use a virtual machine and
we're talking decades without any change. And, when one updates, it's easy to check
if all still works. Often the demands and styles are simple and predictable. It's
way more likely that a hard coded solution in some large programming environment has
stability issues than that the \CONTEXT\ bit has.

If \CONTEXT\ is used in for instance documentation of (say) software, again there
is no real issue. Such documents are simple, evolve and therefore have no stable
page flow, and updating \CONTEXT\ is not needed if the once decided upon coding
is stable. You don't need the latest features. We've written styles and setups for
such tasks and indeed they run for ages.

It can make me smile to see how much effort sometimes goes in low quality
rendering where \CONTEXT\ could do a way better job with far less investment in
time and money but where using some presumed stable toolkit is used instead, one
that comes with expensive licensing, from companies that come and go but shine in
marketing. (A valid question is to what extent the quality of and care for
documentation reflects the core products that a company produces, at least under
the hood.)

The biggest hurdle in setting up a decent efficient workflow is that it has to be
seen as a project: proper analysis, proper planning, prototyping and testing,
etc. You invest first and gain later. When dealing with paper many publishers
still think in price per page and have problems seeing that a stable mostly
automated flow in the end can result in a ridiculous low price per page,
especially in typesetting on demand.

\stopsubsubject

\startsubject[title=Hybrids]

Last I will mention a setup that we sometimes are involved in. An author writes
books and uses \TEX. The publisher is okay with that and adds some quality
assurance but in the end the product comes from the author. Maybe images are
oursourced (not always for the better) but these can be handled easily. It can be that
a copy|-|editor is involved and that person also then has to use \TEX\ of course,
or feedback to the author.

Publishers, and this really depends on knowledgeable persons, which as said can
be fun to work with, can look beyond paper and also decide for additional
materials, for instance web pages, interactive exercises, etc. In that case
either \CONTEXT\ input has to be available as \XML\ (an export) or (often better)
\XML\ is the starting point for multiple output. Contrary to what is believed,
there are authors out there who have no problem coding in \XML\ directly. They
think in structured content and reuse! The fact that they can hit a button in the
editor and see the result in \PDF\ helps a lot. It just works.

Here stability is either achieved by simply not updating during a project. There
are however cases where an update is needed, for instance because demands
changed. An example is a project where \ASCIIMATH\ is used which is a moving
target. Of course one can update just that module, and often that works, but not
when a module uses some new improved core helpers. Another example is additional
proofing options.

The budget of such projects seldom permit patching an existing distribution, so
we then just update to the latest but not after checking if the used style works
okay. There is no author involvement in this. Depending on the workflow, it can
even be that the final rendering which involves fine tuning (side) float placement
or page breaks (often educational documents have special demands) is done by us
using special directives.

Such hybrid workflows are quite convenient for all parties. The publisher works
with the author who likes using these tools, the author can do her or his thing
in the preferred way, and we do what we're best in: supporting this. And it
scales up pretty well too if needed, without much costs for the publishers.

\stopsubject

\startsubject[title=Conclusion]

So what can we conclude with respect to the demand for stability? First of course
that it's important that our files keep running well. So, functionality should be
stable. Freezing a distribution will make sure that during project you don't run
into issues. Many \CONTEXT\ users update frequently in order to benefit from the
latest additions. Most will not be harmed by this, but when something really
breaks it's users like those on the \CONTEXT\ support list (who often also
contribute in helping out other users) that are listened to first. Publishers
demands play no role in this, if only because they also play no role in
typesetting, and if they want to they should also contribute. The same is true
for large suppliers. We're talking of free software often written without any
compensation so these parties have no say in the matter unless they pay for it.
It's small suppliers, authors and general users that matter most. If \CONTEXT\ is
part of a workflow that we support, of course stability is guaranteed quite well,
and those paying for that never have an issue with better solutions popping up.
In fact, \CONTEXT\ is often just a tool then, one that does the job and questions
about stability don't matter much in practice, as long as it does the job well.

The main engine we use, \LUATEX, will be quite stable from version 1.10 and we'll
try to make sure that newer versions are capable of running an older \CONTEXT,
which is easier when no fundamental changes happen in the engine. Maybe a stripped
down version of \LUATEX\ for \CONTEXT\ can facilitate that objective even more.

Users themselves can try to stick to standard \CONTEXT\ features. The more tricks
you apply, the less stable your future might be. Most mechanism are not evolving
but some, like those that deal with columns, might become better over time. But
typesetting in columns is often a one|-|shot adventure anyway (and who needs
columns in the future).

Of one thing users can be sure. There will never be a \CONTEXT\ professional or
\CONTEXT\ enterprise. There is only one variant. All users get the same
functionality and policies don't change suddenly. There will be no lock in to some
cloud or web based service either. Of course one can hire us for support of any
kind but that's independent of the distributed package. There is support by users
for users on mailing lists and other media. That itself can also guard stability.

But, always keep in mind that stability and progress, either of not driven by the
environment that we operate in, can be in conflict.

\stopsubject

\stopchapter

\stopcomponent
