% language=us runpath=texruns:manuals/musings

\startcomponent musings-hownotto

\environment musings-style

\startchapter[title={How not to install \CONTEXT}]

\startalign[flushleft]

Installing LuaMetaTeX can be a complex process that requires some technical
expertise, but the following steps should give you a general idea of what is
involved:

\startitemize[n]

\startitem
    First, you need to ensure that you have a recent version of the Lua
    programming language installed on your system. You can download the latest
    version of Lua from the official website at http://www.lua.org/download.html.
    \footnote {The \LUA\ code needed is part of the source tree that can be
    downloaded from GitHub or websites.}
\stopitem

\startitem
    Next, you need to download the latest version of the MetaTeX distribution,
    which includes the LuaMetaTeX engine, from the official ConTeXt Garden
    website at https://wiki.contextgarden.net/ConTeXt_Standalone. \footnote
    {There is no \METATEX, although we sometimes joke about it.}
\stopitem

\startitem
    Once you have downloaded the MetaTeX distribution, extract the files to a
    directory on your system. \footnote {So here one is stuck.}
\stopitem

\startitem
    You can then run the LuaMetaTeX engine by opening a command prompt or
    terminal window and navigating to the directory where you extracted the
    MetaTeX files. From there, you can run the command "luametatex" followed by
    the name of the TeX file you want to process. \footnote {Shouldn't it be
    compiled first? And even then it needs some format, so one needs \type
    {context} and \type {mtxrun}.}
\stopitem

\startitem
    To make it easier to use LuaMetaTeX with your favorite text editor, you may
    also want to install a TeX distribution such as TeX Live or MiKTeX, which
    includes support for LuaMetaTeX. These distributions typically include a
    graphical user interface that makes it easier to manage your TeX installation
    and configure your system for use with LuaMetaTeX. \footnote {Indeed
    installing \TEXLIVE\ is easier, as is installing the smaller reference
    installation which uses \LUAMETATEX\ as its own installer. As far as we know,
    \MIKTEX\ doesn't include \LMTX. And yes, consulting the documentation might
    be best.}
\stopitem

\stopitemize

It's worth noting that the exact steps for installing LuaMetaTeX may vary
depending on your operating system and the specific TeX distribution you are
using. For more detailed instructions, you may want to consult the official
documentation for LuaMetaTeX and the TeX distribution you are using. \footnote
{So here is the way out of the proposed mess.}

\start \blank
\bf by ChatGPT
\footnote {Queried by Mikael Sundqvist.}
\footnote {Which (at least here) is more about well formed sentences than about
verified content. We can only hope that new \TEX\ users are able to recognize
a fake.}
\stop

\stopalign

\stoptext
