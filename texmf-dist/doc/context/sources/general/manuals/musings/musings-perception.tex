% language=us runpath=texruns:manuals/musings

\definefontfeature[ligatures][liga=yes,mode=node]

\startcomponent musings-perception

\environment musings-style

\startchapter[title=Advertising \TEX]

I can get upset when I hear \TEX ies boast about the virtues of \TEX\ compared to
for instance Microsoft Word. Not that I feel responsible for defending a program
that I never use(d) but attacking something for no good reason makes not much
sense to me. It is especially annoying when the attack is accompanied by a
presentation that looks pretty bad in design and typography. The best
advertisements for \TEX\ should of course come from outside the \TEX\ community,
by people impressed by its capabilities. How many \TEX ies can really claim that
Word is bad when they never tried to make something in it with a similar learning
curve as they had in \TEX\ or the same amount of energy spent in editing and
perfecting a word|-|processor|-|made document.

In movies where computer technology plays a role one can encounter weird
assumptions about what computers and programs can do. Run into a server room,
pull one disk out of a \RAID-5 array and get all information from it. Connect
some magic device to a usb port of a phone and copy all data from it in seconds.
Run a high speed picture or fingerprint scan on a computer (probably on a remote
machine) and show all pictures flying by. Okay, it's not so far from other
unrealistic aspects in movies, like talking animals, so maybe it is just a
metaphor for complexity and speed. When zapping channels on my television I saw
\in{figure}[fig:tex-in-movie] and as the media box permits replay I could make a
picture. I have no clue what the movie was about or what movie it was so a
reference is lacking here. Anyway it's interesting that seeing a lot of \TEX\
code flying by can impress someone: the viewer, even if no \TEX ie will ever see
that on the console unless in some error or tracing message and even then it's
hard to get that amount. So, the viewer will never realize that what is seen is
definitely not what a \TEX ie wants to see.

\startplacefigure[title={\TEX\ in a movie},reference=fig:tex-in-movie]
    \externalfigure[tex-in-movie.jpg][height=8cm]
\stopplacefigure

So, as that kind of free advertisement doesn't promote \TEX\ well, what of an
occasional mentioning of \TEX\ in highly|-|regarded literature? When reading
\quotation {From bacteria to Bach and back, the evolution of minds} by Daniel
Dennett I ran into the following:

\startquotation
In Microsoft Word, for instance, there are the typographical operations of
superscript and subscript, as illustrated by

\startnarrower
base\high{power}
\stopnarrower

and

\startnarrower
human\low{female}
\stopnarrower

But try to add another superscript to base\high{power}\emdash it {\em should}
work, but it doesn't! In mathematics, you can raise powers to powers to powers
forever, but you can't get Microsoft Word to display these (there are other
text|-|editing systems, such as TeX, that can). Now, are we sure that human
languages make use of true recursion, or might some or all of them be more like
Microsoft Word? Might our interpretation of grammars as recursive be rather an
elegant mathematical idealization of the actual \quotation {moving parts} of a
grammar?
\stopquotation

Now, that book is a wonderfully interesting read and the author often refers to
other sources. When one reads some reference (with a quote) then one assumes that
what one reads is correct, and I have no reason to doubt Dennett in this. But
this remark about \TEX\ has some curious inaccuracies. \footnote {Of course one
can wonder in general that when one encounters such an inaccuracy, how valid
other examples and conclusions are. However, consistency in arguments and
confirmation by other sources can help to counter this.}

First of all a textual raise or lower is normally not meant to be recursive.
Nesting would have interesting consequences for the interline space so one will
avoid it whenever possible. There are fonts that have superscript and subscript
glyphs and even \UNICODE\ has slots for a bunch of characters. I'm not sure what
Word does: take the special glyph or use a scaled down copy?

Then there is the reference to \TEX\ where we can accept that the \quotation {E}
is not lowered but just kept as a regular \quotation {e}. Actually the mentioning
of nested scripts refers to typesetting math and that's what the superscripts and
subscripts are for in \TEX. In math mode however, one will normally raise or
lower symbols and numbers, not words: that happens in text mode.

While Word will use the regular text font when scripting in text mode, a \TEX\
user will either have to use a macro to make sure that the right size (and font)
is used, or one can revert to math mode. But how to explain that one has to enter
math and then explicitly choose the right font? Think of this:

\startbuffer
efficient\high{efficient} or
efficient$^{\text{efficient}}$ or \par
{\bf efficient\high{efficient} or
efficient$^{\text{efficient}}$}
\stopbuffer

\typebuffer

Which gives (in Cambria)

\getbuffer

Now this,

\startbuffer
efficient\high{efficient\high{efficient}} or
efficient$^{\text{efficient$^{\text{efficient}}$}}$ or \par
{\bf efficient\high{efficient\high{efficient}} or
efficient$^{\text{efficient$^{\text{efficient}}$}}$}
\stopbuffer

\typebuffer

will work okay but the math variant is probably quite frightening at a glance for
an average Word user (or beginner in \TEX) and I can understand why someone would
rather stick to click and point.

\getbuffer

Oh, and it's tempting to try the following:

\startbuffer
efficient{\addff{f:superiors}efficient}
\stopbuffer

\typebuffer

but that only works with fonts that have such a feature, like Cambria:

\blank {\switchtobodyfont[cambria]\getbuffer} \blank

To come back to Dennett's remark: when typesetting math in Word, one just has to
switch to the math editing mode and one can have nested scripts! And, when using
\TEX\ one should not use math mode for text scripts. So in the end in both
systems one has to know what one is doing, and both systems are equally capable.

The recursion example is needed in order to explain how (following recent ideas
from Chomsky) for modern humans some recursive mechanism is needed in our
wetware. Now, I won't go into details about that (as I can only mess up an
excellent explanation) but if you want to refer to \TEX\ in some way, then
expansion \footnote{Expanding macros actually works well with tail recursion.} of
(either combined or not) snippets of knowledge might be a more interesting model
than recursion, because much of what \TEX\ is capable of relates to expansion.
But I leave that to others to explore. \footnote {One quickly starts thinking of
how \cs {expandafter}, \type {noexpand}, \type {unexpanded}, \type {protected}
and other primitives can be applied to language, understanding and also
misunderstanding.}

Now, comparing \TEX\ to Word is always kind of tricky: Word is a text editor with
typesetting capabilities and \TEX\ is a typesetting engine with programming
capabilities. Recursion is not really that relevant in this perspective. Endless
recursion in scripts makes little sense and even \TEX\ has its limits there: the
\TEX\ math engine only distinguishes three levels (text, script and scriptscript)
and sometimes I'd like to have a level more. Deeper nesting is just more of
scriptscript unless one explicitly enforces some style. So, it's recursive in the
sense that there can be many levels, but it also sort of freezes at level three.

\startplacefigure[title={Nicer than \TEX},reference=fig:nicer-than-tex]
    \externalfigure[mathematics.png][width=\textwidth]
\stopplacefigure

I love \TEX\ and I like what you can do with it and it keeps surprising me. And
although mathematics is part of that, I seldom have to typeset math myself. So, I
can't help that \in {figure} [fig:nicer-than-tex] impresses me more. It even has
the so|-|familiar|-|to|-|\TEX ies dollar symbols in it: the poem \quotation
{Poetry versus Orchestra} written by Hollie McNish, music composed by Jules
Buckley and artwork by Martin Pyper (I have the \DVD\ but you can also find it on
\YOUTUBE). It reminds me of Don Knuth's talk at a \TUG\ meeting. In \TUGBOAT\
31:2 (2010) you can read Don's announcement of his new typesetting engine i\TEX:
\quotation {Output can be automatically formatted for lasercutters, embroidery
machines, \THREED\ printers, milling machines, and other \CNC\ devices \unknown}.
Now that is something that Word can't do!

\stopcomponent
