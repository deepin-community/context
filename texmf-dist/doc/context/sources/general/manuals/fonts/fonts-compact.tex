
% language=us runpath=texruns:manuals/fonts

\continuewhenlmtxmode

\startcomponent fonts-compact

\environment fonts-environment

\definetypeface
  [narrowtt] [tt]
  [mono] [modern-condensed] [default] [features=none]

\startchapter[title=Compact][color=darkgray]

\startsection[title=Introduction]

The title of this chapter is somewhat misleading but in the end a lot boils down
to efficient but flexible usage as offered in \CONTEXT\ \LMTX\ and \LUAMETATEX.
This also means that what is discussed here does not apply to \MKIV.

\stopsection

\startsection[title=Compact mode]

\startsubsection[title=What \TEX\ needs]

A traditional \TEX\ approach to fonts is that you explicitly load a font
that then gets an identifier which is used to trigger its usage. In plain \TEX\
you find these lines:

\startcolumns[n=3]
\starttyping
\font\tenrm     cmr10
\font\preloaded cmr9
\font\preloaded cmr8
\font\sevenrm   cmr7
\font\preloaded cmr6
\font\fiverm    cmr5

\font\teni      cmmi10
\font\preloaded cmmi9
\font\preloaded cmmi8
\font\seveni    cmmi7
\font\preloaded cmmi6
\font\fivei     cmmi5

\font\tensy     cmsy10
\font\preloaded cmsy9
\font\preloaded cmsy8
\font\sevensy   cmsy7
\font\preloaded cmsy6
\font\fivesy    cmsy5

\font\tenex     cmex10

\font\preloaded cmss10
\font\preloaded cmssq8

\font\preloaded cmssi10
\font\preloaded cmssqi8

\font\tenbf     cmbx10
\font\preloaded cmbx9
\font\preloaded cmbx8
\font\sevenbf   cmbx7
\font\preloaded cmbx6
\font\fivebf    cmbx5

\font\tentt     cmtt10
\font\preloaded cmtt9
\font\preloaded cmtt8

\font\preloaded cmsltt10

\font\tensl     cmsl10
\font\preloaded cmsl9
\font\preloaded cmsl8

\font\tenit     cmti10
\font\preloaded cmti9
\font\preloaded cmti8
\font\preloaded cmti7
\stoptyping
\stopcolumns

There are a few more later on in that file. Although users seldom see such low
level definitions it shows a few interesting aspects. First of all, we see that
there are different fonts for different sizes. These can be triggered with
commands like \type {\tenbf}. We also see that some fonts get the name \type
{\preloaded} but because later one we have:

\starttyping
\let\preloaded\undefined
\stoptyping

that command is no longer meaningful. The fonts are loaded but not accessible.
However, when you define it again, the already loaded variant will be used which
might save some runtime. Back to the sizes: all is centered around a ten point
body font. The smaller sizes are for math and footnotes, the larger ones for
section titles and such. The different sizes are optimized for that size and
expected to be printed at that size. Of course looking at a document on a phone,
epub device, high resolution monitor or beamer image, spoils that concept:
instead of being optimized for some size it then becomes a variant with different
properties. A seven point glyph at indeed seven point in a math superscript makes
sense but when blown up to 16 point there is less need for a different shape
unless one wants (for instance) to run the scripts narrower. The danger of ink
filling shapes at small sizes is not present when we scale up.

Anyway, the engine loads a font and when triggered it will use the dimensions
that it provides: character properties like widths, heights, depths, italic
corrections, kerns, ligatures, math variants and extensibles as well as font
properties like spacing. If you load a ten point font at fifteen point the
original gets scaled and a copy is used with these new properties. If you use
400 sizes, you get 400 copies.

\stopsubsection

\startsubsection[title=How \LUATEX\ works]

Because \LUATEX\ is a wide engine loading a font not only takes more time but
using one also consumes more memory. In principle there should be no real
difference in the amount of fonts loaded unless the macro package does a poor
job. In what we call base mode, there is no difference with the traditional
approach: \TEX\ needs the usual properties so, assuming that callbacks are used
for font management, only these have to be passed to the engine. In what we call
node mode in \CONTEXT\ rendering is delegated to \LUA. The engine is not involved
in for instance ligature building and kerning.

Where in a traditional engine small caps and old style variants are bound to
specific fonts, using them involves loading a new instance. However, in \CONTEXT\
we can avoid that by using dynamic features. This dynamic mechanism is part of
node mode and is a convenient way to deal with small caps and oldstyle shapes. So
here we can actually save some memory due to less instances but of course at the
cost of some more complexity. But we still have dedicated instances for say eight
point and ten point.

\stopsubsection

\startsubsection[title=What \LUAMETATEX\ can do]

In this engine we can do scaling of glyphs on demand. This makes it possible to
only define a ten point instance and scale for instance an eight point size from
that instance. This can come at the cost of more runtime due to more calculations
but in the end experiments demonstrated that runtime can become less. Memory
consumption of course is less anyway.

This scaling feature has quite some impact on the front|-|end because dimensions
that relate to the current font might have to be scaled too. It is definitely
something to take into account at the \LUA\ end. In addition to scaling we also
have some other manipulations. Although these are taken into account in the
frontend, it is the backend that eventually has to take care of it.

\starttabulate
\NC \type {\glyphscale } \NC scales the glyphs in two directions \NC \NR
\NC \type {\glyphxscale} \NC scales the glyphs horizontally, aka extending \NC \NR
\NC \type {\glyphyscale} \NC scales the glyphs vertically, aka squeezing \NC \NR
\NC \type {\glyphslant } \NC transforms the glyphs such that it becomes oblique\NC \NR
\NC \type {\glyphweight} \NC boldens the glyphs \NC \NR
\stoptabulate

Apart from slanting all these primitives result in different metrics. The weight is
somewhat tricky because here we need some guesswork.

When we use compact font mode in \CONTEXT\ \LMTX, the same font is loaded once
and scaled on demand using the above control options. Of course this need some
management, for instance because when setting the scale an already active scale
might effectively get rescaled.

% \unprotect
% \meaning\font_scale_defined_x
% \protect
% \glyphscale\dimexpr\xtextface*\plushundred/\onepoint\relax
% \gyphrescale\xtextface\relax

\stopsubsection

\stopsection

\startsection[title=Design sizes]

\startsubsection[title=Introduction]

The first \TEX\ distributions had an important characteristic: the fonts that
came with them were tuned for a certain size and we could speak of them having
\quote {design sizes}. There were good reasons for this.

\startitemize

\startitem
    The typeset result was supposed to be printed and we were not talking desktop
    printers here, but printing presses. A book normally came in a ten point
    design; after all, how many type does a typesetter has available in lead or
    film.
\stopitem

\startitem
    When math is involved there are two levels of script which means that there
    are seven and five point sizes used too. At these smaller sizes some details of a
    glyph have to be exaggerated and other bits and pieces have to become more
    open in order not to become a blob of black ink.
\stopitem

\startitem
    A section header or maybe a bit larger chapter title got its own size, often a
    bit heavier and maybe running somewhat wider.
\stopitem

\startitem
    Combined, this kind of usage results in a predictable way of presenting the
    result to the user. Unless the reader uses a magnifying glass, what you see is
    what you got.
\stopitem

\stopitemize

Compare this to todays rendering and presentation. What is a ten point on a
mobile phone, 1920x1080 14 inch laptop screen, a 27 inch 4K desktop monitor, or
beamed on the wall presentation with a 3 meter diameter? And let's not bring
these low res epub devices into the discussion. It is of course also a matter of
taste but it is not uncommon to hear \TEX ies claim virtues of fonts and the
rendering with \TEX\ (take expansion) while at the same time observing that they
ignore aspects of typesetting that would really make their documents look nice.

Where in the good old \TEX\ approach, rooted in traditional typesetting and
printing, the concept of \quote {design size} made sense, I personally think that
in today's rendering and usage it has no real meaning. A basic ten point font can
show up in any size and is not adapted on-the-fly which in turn would demand
on-the-fly typesetting with definitely a different look and feel due to different
line breaks, and we don't even consider how scaling (bitmap or vector) images fit
into such adaptations. It might work in browsers but not on more traditional
designed and optimized published content.

There is a sort of tradition that a font comes in \quote {normal}, \quote {bold},
\quote {italic} and \quote {bolditalic} shapes. It can make sense to come up with
a bit lighter and more heavy variant, but some designers can't make up their mind
and come with a whole range, so we see \quote {thin}, \quote {demi-bold}, \quote
{medium} and \quote {medium-bold}, and whatever fits the repertoire. The choice
of what looks best is delegated to the user. In the past there definitely was a
commercial aspect to this, after all each set has to be bought, and when it's not
clear what to use you end up with dozens of fonts, of which most are never used.
Today the drive probably comes from the possibilities that font design programs
offer. And when variable fonts showed up we suddenly saw fonts come in many
weights, again putting the burden of what looks best on users instead of what is
intended on designers. As a side effect, the distinction between designs becomes
vague and there is a danger that in the end users no longer really care what font
they use because they can make it look like any other anyway.

I like to argue that these \quote {new} fonts that come in what some call design
sizes has little to do with design sizes in the traditional sense. The variants
more reflect usage, like \quote {display} or \quote {heading}. As mentioned
above, one can indeed consider them to be designed for (say) larger than default
sizes, but then the word \quote {size} no longer hold.

Say that in a traditional \TEX\ document one has the running text in ten point. A
section header can use the twelve point and a chapter heading fourteen (\TEX\ has
these 14.4pt, 17.3pt sizes). In print they really are like that but but when
watched on a device the ten point can effectively be a eight point (epub) or
twenty point (monitor) so there size has become meaningless because the font is
not seen at the size it was designed for. When we choose a different one (say
\quote {display}) we do so because of other properties than size. This means that
instead of \quote {design size} we can better talk of \quote {usage}.

The advantage of talking \quote {usage} is that we don't need to think size. In a
traditional \TEX\ setup using design sizes you end up with (difficult) choices.
If the document is in eleven point, do we scale down the twelve point or scale up
the ten point, assuming that we have ten and twelve point designs optimized for
the running text. And how about math when we go down to eight point in footnotes,
do we really use five point script and three point script script or do we go for
six point and five or maybe four point. Nowadays we can decide to just scale
relative, and sticking to what was common in the days of lead typesetters with
fixed sizes is not needed. Maybe a gradual evolution meant that we stuck to
concepts a bit too much. So in todays \TEX\ setup we can decide to just define a
basic setup (no design sizes at all) and do that multiple times depending on
usage: kicking in a different set of shapes for \quote {display} and \quote
{heading} when available. There is no reason to use a shape meant for display for
a sixteen point running text because we scale the ten point and then use the
display font at (say) twenty four point: it is usage and not scale that matters
here. And an occasional ten point display font used at ten point in a ten point
running text can have some use, but we make sure to trigger it with the right
denomination: we mix a \quote {display} design into a \quote {regular} stream at
the same size.

It is of course also a matter of perception and taste but as with font expansion
(aka hz) it could be annoying to see two different two's in $2^2$, something at
10 points might look better at small sizes but worse as larger. In {figures}
[fig:designsizes:1] \in {and} [fig:designsizes:2] you can see how it looks when
we scale a design size beyond its intended size. Especially the second stylistic
variant (meant for script script) can stand out.

\def\SampleOne#1#2{\switchtobodyfont[#1]\scale[scale=#2000]{\im{2^{2^2}}}}

\startbuffer
\startcombination[6*3]
    {\SampleOne{lucida} {1}} {}
    {\SampleOne{lucida} {2}} {}
    {\SampleOne{lucida} {3}} {}
    {\SampleOne{lucida} {4}} {}
    {\SampleOne{lucida} {5}} {}
    {\SampleOne{lucida} {6}} {}
    {\SampleOne{pagella}{1}} {}
    {\SampleOne{pagella}{2}} {}
    {\SampleOne{pagella}{3}} {}
    {\SampleOne{pagella}{4}} {}
    {\SampleOne{pagella}{5}} {}
    {\SampleOne{pagella}{6}} {}
    {\SampleOne{modern} {1}} {}
    {\SampleOne{modern} {2}} {}
    {\SampleOne{modern} {3}} {}
    {\SampleOne{modern} {4}} {}
    {\SampleOne{modern} {5}} {}
    {\SampleOne{modern} {6}} {}
\stopcombination
\stopbuffer

\startplacefigure[title={Design sizes beyond their design size (1).},reference=fig:designsizes:1]
    \getbuffer
\stopplacefigure

\def\SampleOne#1#2{\switchtobodyfont[#1]\scale[scale=#2000]{\im{2\char\getmathcharone`2 \char\getmathchartwo`2}}}

\startplacefigure[title={Design sizes beyond their design size (2).},reference=fig:designsizes:2]
    \getbuffer
\stopplacefigure

With arbitrary scaling \quote {design size} no longer is a meaningful concept but
(intended) usage might be; we could just drop the \quote {sizes} and stick to
\quote {design} or maybe go for \quote {design variants} or simply \quote
{usage}. So, that is what in \CONTEXT\ we will carry on. That still leaves the
user the horrible task of figuring out all these weights, because in the worst
case all the usage sets (3-5) combined with weights (3-8) will give us (9-40)
setups to come up with and then choose from. Of course we can decide that the
regular shape is what the designer came up with and that the rest is just
marketing and bit of technical show-off. And if it's all too much work for a
user, the regular shapes a good choice anyway. I wonder if readers really care
that much, especially when they read on devices instead of proper high resolution
print.

\stopsubsection

\startsubsection[title=A system]

Mainly because, as shown above, \TEX\ shipped with Computer Modern in several
design sizes the low level font system is designed to deal with that. This means
that setups for \MKII\ look more complex than those for \MKIV\ and \LMTX,
although deep down the same mechanism is active.

Most fonts come with normal, bold, italic and bolditalic variants and for such a
basic set of four, defining a typescript is relatively easy. If we don't need to
access the specific fonts by name (like \typ {MyFontSerif}) we can just map
(file)names onto \type {Serif} directly. If the \quote {select font} mechanism is
used (\typ {\definefontfamily}) one doesn't even make typescripts but that
assumes that we can reliably resolve fontnames.

So how, in the perspective of the mentioned (usage specific) font variants does
one set up a system. I will use Iwona as an example because that font comes in
different variants. For that font we have the following choices:

\startlines
iwona      iwona-light      iwona-medium      iwona-heavy
iwona-cond iwona-light-cond iwona-medium-cond iwona-heavy-cond
\stoplines

\startbuffer[iwona-use]
\usebodyfont[iwona]
\usebodyfont[iwona-heavy]
\usebodyfont[iwona-cond]
\stopbuffer

Say that we know that we use, we can instruct \CONTEXT\ to preload these so that
we don't get local definitions and loading when we use one of them grouped, as in
a section title.

\typebuffer[iwona-use]

These names are a bit too bound to the fontname so we introduce some abstraction:

\startbuffer[iwona-abstract]
\aliasbodyfont[mainfont] [iwona]
\aliasbodyfont[titlefont][iwona-heavy]
\aliasbodyfont[notefont] [iwona-cond]
\stopbuffer

\typebuffer[iwona-abstract]

\startbuffer[iwona-demo]
\setupbodyfont[mainfont]
\setuphead[chapter][style=\switchtobodyfont[titlefont]\tfd]
\setuphead[section][style=\switchtobodyfont[titlefont]\tfc]
\setupnotes[bodyfont={notefont,small}]

\starttext

\startchapter[title=Demo]
\startsection[title=Demo]
\samplefile {tufte} \footnote{\samplefile{tufte}}
\stopsection
\stopchapter

\stoptext
\stopbuffer

\typebuffer[iwona-demo]

We see the result in \in {figure} [fig:iwona:demo] and when that example is
processed the log tells that quite some fonts got loaded: 18. The majority is
traditional math fonts that get assembled into a pseudo \OPENTYPE\ font.

\startplacefigure[default=page,title={An example font setup},reference=fig:iwona:demo]
    \scale[width=\textwidth]{\typesetbuffer[iwona-use,iwona-abstract,iwona-demo]}
\stopplacefigure

The log also mentions: {\tt 28 instances, 24 shared in backend, 2 common vectors,
22 common hashes, load time 0.615 seconds}. When we start the document with:

\starttyping
\enableexperiments[fonts.compact]
\stoptyping

We can see that less instances are loaded: {\tt 6 instances, 2 shared in backend,
2 common vectors, 0 common hashes, load time 0.154 seconds}. We also save half a
second. All this is because we now enabled scaling on-the-fly.

\stopsubsection

\startsubsection[title=Relations]

There is a lot of history behind the font system. We started out with design
sizes and features distributed over fonts. So, at the lowest level we had to
handle for instance small caps and oldstyle via specific definitions. A tenpoint
definition looked similar to a twelvepoint but with different font names. On top
of that there were more general (and abstract) definitions as well as size
specific mappings to larger or smaller variants of the fonts used at that size.

With a single design size a setup is much simpler but it still uses that granular
low level mechanism. In all of this, typescripts play an important role: they are
recipes that relate symbolic names to files, and use these symbolic names in
assemblies of fonts. Eventually a combination of serif, sans, mono and math is
defined, even if most fonts don't have these four available. In that case we
combine different designs.

We could come up with a new system and it could even be a bit faster but we also
need to keep compatibility in mind. It is not trivial to come up with something
better (read: more \LMTX-ish) that is backward compatible. The default setup for
a bodyfont is a rather safe one but not always the best for every body font.
Without complicatign the existing mechanisms too much some extensions in for
instance the bodyfont environment system were possible, for instance:

% \setupbodyfontenvironment
%   [metric]
%   [
%     small=0.7,
%     interlinespace=1.4,
%   ]

\starttyping
\definebodyfontenvironment[iwona]      [default][metric]
\definebodyfontenvironment[iwona-heavy][default][metric]
\definebodyfontenvironment[iwona-cond] [default][metric]
\stoptyping

where \type {metric} is a predefined setup and \type {default} indicates that all
unset sizes use that one. You can define your own or adapt the predefined ones.

\starttyping
\setupbodyfontenvironment
  [metric]
  [small=0.7]
\stoptyping

or

\starttyping
\setupbodyfontenvironment
  [metric]
  [interlinespace=1.4]
\stoptyping

for a line height relative to the bodyfont size. Although this seldom happens
you can be even mor granular. Say that you have

\starttyping
\definetypeface [DejavuA] [rm] [serif] [dejavu] [default]
\definetypeface [DejavuA] [ss] [sans]  [dejavu] [default]
\definetypeface [DejavuA] [tt] [mono]  [dejavu] [default]
\definetypeface [DejavuA] [mm] [math]  [dejavu] [default]
\stoptyping

We can now make the sans a bit smaller:

\starttyping
\definebodyfontenvironment
  [DejavuA] [ss] [default]
  [text=0.95, ...]
\stoptyping

and a specific bodyfont size even more:

\starttyping
\definebodyfontenvironment
  [DejavuA] [ss] [8pt]
  [text=0.85, ...]
\stoptyping

Of course this kind of setups can best be achieved with an isolated, stepwise
written environment file because testing it on a huge and versatile document
source is not much fun.

\stopsubsection

% TODO: set tx to be some other bodyfont but with same interlinespace

\stopsection

\stopchapter

\stopcomponent
